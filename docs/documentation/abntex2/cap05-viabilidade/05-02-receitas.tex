\section{Receitas}

Com base nos custos mensais e totais da Seção \ref{sec:custos}, constata-se que o projeto não é financeiramente viável para a entidade parceira, pois --- mesmo parcelando --- ela teria que arcar com todas as despesas do projeto descritas.

Assim sendo, será considerada uma eventual adaptação do sistema para um modelo \gls{saas} a fim de analisar as receitas que a aplicação geraria. A Tabela \ref{tab:receitas-saas} apresenta os valores das mensalidades dos planos definidos para a aplicação.

\begin{table}[htbp]
	\centering
	\caption{Projeção de receitas mensais - SaaS para salões de beleza}
	\label{tab:receitas-saas}
	\begin{tabular}{lrr}
		\toprule
		\textbf{Plano} & \multicolumn{2}{r}{\textbf{Valor Mensal (R\$)} }\\
		\midrule
		\multicolumn{1}{l}{\textbf{Planos de Assinatura}} & & \\
		\quad Básico & & 59,90 \\
		\quad Profissional & & 99,90 \\
		\bottomrule
	\end{tabular}
	\fonte{Produzido pelos autores}
\end{table}

A precificação dos planos foi feita baseando-se nos preços praticados pelos concorrentes identificados na Seção \ref{sec:analise-concorrencia}, além de considerar os custos operacionais e tentar oferecer um preço atrativo para estimular possíveis clientes a usarem o sistema.