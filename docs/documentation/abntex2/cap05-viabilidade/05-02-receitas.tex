\section{Receitas}

Com base nos custos mensais e totais da Seção \ref{sec:custos}, constata-se que o projeto não é financeiramente viável para a entidade parceira, pois --- mesmo parcelando --- ela teria que arcar com todas as despesas do projeto descritas.

Assim sendo, será considerada uma eventual adaptação do sistema para um modelo \gls{saas}, a fim de analisar as receitas que a aplicação poderia gerar. A Tabela \ref{tab:receitas-saas} apresenta os valores das mensalidades e as principais funcionalidades de cada plano.

A precificação dos planos foi definida com base nos preços praticados pelos concorrentes identificados na Seção \ref{sec:analise-concorrencia}, considerando também os custos operacionais e o público-alvo do sistema. 

\begin{table}[htbp]
	\centering
	\caption{Projeção de receitas mensais - SaaS para salões de beleza}
	\label{tab:receitas-saas}
	\begin{tabular}{p{3cm}p{7cm}r}
		\toprule
		\textbf{Plano} & \textbf{Funcionalidades e Diferenciais} & \textbf{Valor Mensal (R\$)} \\
		\midrule
		\textbf{Básico} & Indicado para pequenos salões e profissionais autônomos. Inclui gestão de agendamentos, serviços, profissionais e turnos, além do envio de notificações \textit{in-app} para clientes e profissionais sobre os agendamentos. Permite ainda a avaliação de serviços pelos clientes e o cadastro de até 5 profissionais por salão. & 59,90 \\
		\addlinespace
		\textbf{Profissional} & Voltado a salões de médio e grande porte. Inclui todas as funcionalidades do plano Básico e adiciona recursos avançados como registro de pagamentos, gestão de permissões de acesso, criação de campanhas de desconto, envio de notificações personalizadas e por e-mail, geração de relatórios visuais e suporte a número ilimitado de profissionais. & 99,90 \\
		\bottomrule
	\end{tabular}
	\fonte{Produzido pelos autores}
\end{table}

O \textbf{Plano Básico} busca atender profissionais que desejam digitalizar suas operações de forma acessível, oferecendo um conjunto essencial de funcionalidades para o controle de agendamentos, serviços e equipe. 

Já o \textbf{Plano Profissional}, destinado a salões com maior volume de atendimentos, amplia significativamente as possibilidades de gestão, automação e análise, justificando o valor superior ao agregar funções estratégicas voltadas à expansão e tomada de decisão.
