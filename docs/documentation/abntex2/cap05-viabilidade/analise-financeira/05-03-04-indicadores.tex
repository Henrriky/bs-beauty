\subsection{Indicadores financeiros}

Para complementar a análise dos cenários apresentados, foram utilizadas métricas financeiras amplamente empregadas em estudos de viabilidade econômica, como o \textbf{Retorno sobre o Investimento (ROI)} e o \textbf{Período de Retorno (Payback Period)}.

O \textbf{ROI (Return on Investment)} mede a relação entre o lucro obtido e o investimento inicial, indicando o percentual de retorno gerado pelo projeto. Sua fórmula é:

\[
ROI = \frac{\text{Lucro Líquido}}{\text{Investimento Total}} \times 100
\]

O \textbf{Payback Period} representa o tempo necessário para recuperar o investimento inicial a partir das receitas acumuladas, sendo calculado como:

\[
\text{Payback} = \frac{\text{Investimento Inicial}}{\text{Receita Líquida Média Mensal}}
\]

\subsubsection{Cenário realista}

Considerando o cenário realista, em que o aumento mensal é de 15 clientes no plano básico e 10 no profissional, ao fim de 12 meses, o lucro acumulado obtido somente pelas mensalidades é de aproximadamente R\$ 59 mil.  
Com investimento inicial de R\$ 60.840 (Tabela \ref{tab:custo-total-projeto}), o ROI estimado é de cerca de 108\% e o \textit{payback} ocorre antes de completar um ano.

\subsubsection{Cenário otimista}

No cenário otimista, com crescimento de 30 clientes no básico e 25 no profissional por mês, ao fim de 12 meses, o lucro acumulado pelas mensalidades é de R\$ 253 mil, resultando em ROI de aproximadamente 416\% e confirmando o rápido retorno do investimento. O \textit{payback} ocorre em cerca de 3 meses.

\subsubsection{Cenário pessimista}

No cenário pessimista, com aumento de apenas 5 clientes no plano básico e 3 no profissional por mês, ao fim de 12 meses, o projeto apresenta prejuízo de R\$ 34 mil, refletindo ROI negativo e ausência de payback no período analisado.
