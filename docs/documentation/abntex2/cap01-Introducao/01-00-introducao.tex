%este arquivo contém todas as seções do capítulo de introdução

Segundo levantamento do Sebrae \cite{Sebrae_2025}
, em 2024 o setor de beleza no Brasil movimentou aproximadamente US\$\,27\,bilhões, colocando o país entre os cinco maiores mercados do mundo nesse ramo. Esse volume financeiro trouxe uma série de novidades e gerou, consequentemente, novas demandas. Diante de tantas mudanças e inovações, tornou-se indispensável que os empreendedores se adaptem rapidamente às tendências.

Conforme publicação do hub \emph{Beauty Fair} (Maior evento no setor da beleza no Brasil), até pouco tempo os profissionais autônomos precisavam deslocar-se até a residência de seus clientes para atendê-los ou firmar parcerias prestando serviços dentro de estabelecimentos de terceiros. Com o surgimento dos coworkings de beleza, esse cenário vem se transformando. A própria \emph{Beauty Fair} esclarece que um coworking de beleza é um espaço compartilhado que oferece infraestrutura para que profissionais da área possam trabalhar e colaborar. Trata-se de um local no qual cabeleireiros, maquiadores, esteticistas, manicures, massoterapeutas e demais especialistas podem alugar um posto de trabalho, dividir recursos e alcançar potenciais clientes \cite{BeautyFair}.

Reportagem online na Gazeta do Povo destaca que o ambiente coworking vem se consolidando como um dos modelos de negócio que mais crescem no Brasil, oferecendo ao profissional autônomo flexibilidade, troca de experiências e uma infraestrutura completa sem burocracia nem custos inesperados \cite{gazeta-coworking}. Nesse ambiente, o prestador de serviços tem o benefício de não precisar arcar com despesas de instalação ou manutenção de um salão próprio; basta utilizar o espaço, atender seus clientes e agendar a próxima sessão, preservando o controle sobre seus horários e ganhos.

À medida que esse formato de trabalho se expande, aumenta também a necessidade de maximizar a autonomia e a rentabilidade de cada profissional. Portanto, surge o desafio de gerir agendas, espaços e custos de forma ágil e intuitiva, evitando conflitos de reserva ou falhas de cobrança. Este projeto propõe-se a desenvolver uma aplicação web que atenda exatamente a essa demanda.

\section{Objetivos}

A aplicação web BS Beauty foi desenvolvida especialmente para gerenciar um salão de beleza que opera em modelo coworking, sob a gestão de nossa parceira de extensão Bruna. Seu objetivo principal é otimizar os processos internos e centralizar o agendamento de serviços, atendendo tanto às demandas da gestora quanto às necessidades dos profissionais autônomos.

\noindent\textbf{Para os Clientes Finais:} A plataforma possibilita o agendamento de serviços de forma intuitiva e flexível. Os clientes poderão escolher profissionais específicos ou optar pelo melhor horário disponível, visualizando facilmente a lista de prestadores, seus serviços, preços, tempo de execução e agendas atualizadas.

\noindent\textbf{Para os Profissionais Autônomos:} O sistema BS Beauty tem como propósito reforçar a autonomia dos profissionais sobre sua agenda e finanças. A aplicação permite bloquear horários, editar preços e a duração dos serviços, além de acompanhar os agendamentos realizados (sejam eles do dia, futuros ou passados) e visualizar relatórios detalhados com a receita gerada pelos serviços prestados.

\noindent\textbf{Para a Gestora:} Nossa parceira, Bruna, terá acesso a funcionalidades exclusivas que incluem análise de gastos (como luz e água), gerenciamento do aluguel ou comissão de cada profissional, visualização do fluxo de agendamentos em períodos específicos, envio de mensagens de marketing e promoções aos clientes, e acesso a relatórios financeiros detalhados. Ademais, a gestora poderá incluir ou remover profissionais da plataforma conforme a necessidade.
% ----------------------------------------------------------
\section{Problema e Solução Proposta}

RASCUNHO

A gestão de um local por empreendedores pequenos é frequentemente um desafio. Surgem demandas e processos que muitas vezes começam de froma manual, até que algo começa a dar errado e ve-se a necessidade de uma solução digital, a fim de diminuir erros lógicos e o esforço colocado na administração do negócio. No nosso caso, o objetivo geral da nossa aplicação é suprir as necessidades de um salão de beleza em modelo coworking que, como explicado anteriormete, é um modeol de trabalho recente (desde a pandemia) que visa atender as necessidades de diferentes autonomos, e não uma equipe com um objetivo em comum. O problema é como gerenciar a ocupação de cada profissional no local de trabalho além de controlar finanças e clientes. 

Nossa parceira, Bruna, já utilizava um sistema pra gerenciamento de seu salão, chamado Avec. Porém, apesar de ela já ter uma solução digitalizada, o sistema ainda deixava desejar em alguns aspectos (citar os pontos de insatifasção da bruna). 

Por isso, nossa solução é criar uma aplicação web que continue com todas as funcionalidades que já funcionam para nossa parceira Bruna (exemplificar), adicionar outras que fazem falta (exemplificar) e modificar requisitos funcionais ou não funcionais cuja ideia é boa, mas não funciona direito (como login).
% ----------------------------------------------------------
\section{Justificativa}
graficos com numeros, expor a relevância da solução - extensao e importancia
% ----------------------------------------------------------
\section{Análise da Concorrência}

Foi conduzida uma pesquisa de mercado centrada em plataformas brasileiras que combinam agendamento on-line e gestão financeira para espaços de beleza no modelo coworking. Deste levantamento emergiram três empresas que servirão de referência nesta análise: uma já amplamente consolidada no mercado nacional — embora atue além do universo coworking — e outras duas que, apesar de conhecidas, ainda estão em expansão, mas com foco mais relacionado ao da nossa proposta, o que as torna concorrentes que merecem maior atenção estratégica.

\subsection{Trinks}

%inicio de figura
\begin{figure}[htb]
	\centering
	\includegraphics[width=0.5\textwidth]{cap01-Introducao/Images/1.4.1_Trinks}
	\caption{Logo plataforma Trinks}
	\label{fig:Trinks}
\end{figure}

Trinks é uma plataforma já bem consolidada no mercado de gestão de negócios de beleza, com soluções
personalizadas para barbearias, salões de beleza e clínicas de estética. Criada em 2012, é hoje a
plataforma de gestão para beleza com a maior base instalada do país, englobando aproximadamente
2,8\,milhões de usuários e mais de 40\,mil estabelecimentos, sediada no Rio de Janeiro. A plataforma começou como um empreendimento de consultoria em software personalizado, mas logo identificou uma oportunidade no mercado da beleza e mudou de nicho. Em 2024, foi adquirida pelo grupo Stone, o que alavancou ainda mais funcionalidades do aplicativo, como o autoatendimento.
Atualmente, a Trinks oferece software de back-office (conjunto de módulos internos que controlam o funcionamento do negócio como finanças, estoque, comissões e relatórios), marketplace B2C e meios de pagamento próprios (Trinks Pay), funcionando praticamente como um “ERP + iFood” para salões e barbearias. Existe um
plano grátis que engloba apenas 150 agendamentos por mês, e os planos pagos variam de R\$ 59 a R\$ 249/mês \cite{Trinks}.

Além dos serviços comuns, seus principais diferenciais são:

\begin{itemize}
	\item Ponto de venda (PDV) completo: integração TEF, Pix e split de comissão, atendendo desde
	MEIs até redes com exigência de NFC-e e SAT/ECF;
	\item Marketplace \textit{Trinks.com}, que gera maior fluxo de clientes, expõe o salão ao
	público final e permite pagamento antecipado;
	\item Estrutura em nuvem madura, com SLA de 99{,}9\,\% e aplicativos nativos para
	iOS/Android.
\end{itemize}

Apesar dos grandes benefícios, identificamos algumas brechas do ponto de vista do negócio da nossa
parceira de extensão, Bruna:

\begin{itemize}
	\item A interface pode ser considerada “poluída” para clientes iniciantes, devido ao grande
	número de funcionalidades;
	\item Há pouco foco no aluguel de estações típico do coworking, exigindo ajustes manuais de
	comissão;
	\item Maior parte das funcionalidades estão presentes apenas nos planos superiores.
\end{itemize}

\subsection{Gendo}

%inicio de figura
\begin{figure}[htb]
	\centering
	\includegraphics[width=0.4\textwidth]{cap01-Introducao/Images/1.4.2_Gendo}
	\caption{Logo plataforma Gendo}
	\label{fig:Gendo}
\end{figure}

Lançado em 2017 e sediado em Curitiba-PR, o Gendo se posiciona como um \emph{hub}\footnote{Hub: plataforma centralizada que integra agenda, PDV, finanças e pagamentos em um único ambiente, funcionando como “nó” que organiza os fluxos de dados do negócio.} de gestão 100\,\% em nuvem para negócios além do setor da beleza, como estética, saúde, bem-estar, pet-shop e mais recentemente, espaços em formato coworking. 
Atualmente mantém mais de 10 mil assinantes, com maior penetração nas regiões Sul e Sudeste do Brasil. Foi criado no modelo SaaS com o intuito de oferecer prontamente agenda on-line, automação de lembretes (e-mail/WhatsApp), módulo financeiro completo e integrações com gateways de pagamento (Stone, Cielo e Mercado Pago). Atualmente, os planos são somente pagos e variam de R\$ 32 a R\$ 293/mês, após 14 dias de teste gratuito \cite{Gendo}.

Seus principais diferenciais são:
\begin{itemize}
	\item Caixa do profissional: Módulo pensado para coworking, possibilitando débito automático de aluguel de estação e visualização dos ganhos de cada profissional;
	\item Aplicativo Gendo Pro (iOS e Android): permite ao profissional ver a agenda, acompanhar comissões, pedir saques e registrar fotos de antes e depois dos serviços;
	\item Relatórios instantâneos: exibem ticket médio, previsão de faturamento e dados de cancelamentos, com opção de exportar para Excel.
\end{itemize}


Já os maiores pontos de melhoria identificados são:
\begin{itemize}
	\item Base instalada ainda pequena, limitando o efeito de rede junto a grandes franquias;
	\item Dependência de gateways externos, o que adiciona custo extra ao \emph{split} \footnote{Split é a divisão automática do pagamento entre salão e profissional que, se feita por um gateway externo, gera uma taxa extra.};
	\item Relatórios fiscais avançados disponíveis apenas no plano Premium.
\end{itemize}

\subsection{Avec}

\begin{figure}[htb]
	\centering
	\includegraphics[width=0.4\textwidth]{cap01-Introducao/Images/1.4.3_Avec}
	\caption{Logo plataforma Avec}
	\label{fig:Avec}
\end{figure}

A Avec é, hoje, a principal concorrente do nosso projeto, pois a entidade parceira que motivou este trabalho utiliza essa plataforma para gerenciar seu salão de beleza em modelo coworking. Por esse motivo, ela foi adotada como referência: buscamos manter as funcionalidades que já funcionam bem na Avec e, ao mesmo tempo, acrescentar ou aprimorar recursos que ainda fazem falta para a nossa parceira.

Lançada em 2014 e sediada em São Paulo-SP, a Avec se apresenta como solução``360º'' para salões, barbearias, esmaltarias, spas e estúdios de tatuagem. A plataforma integra software de gestão, um sistema próprio de pagamentos (\emph{Avec Pay}) e um marketplace B2C que encaminha novos clientes aos estabelecimentos. Segundo a empresa, mais de 40 mil negócios utilizam o serviço no
Brasil. Também desenvolvida no modelo SaaS, a ferramenta oferece agenda on\-/line multiprofissional com confirmações via WhatsApp ou SMS, ponto de venda completo com TEF, Pix e \emph{split} interno de comissões, além de módulo financeiro integrado. Dispõe ainda de uma carteira digital empregada em pacotes pré\=/pagos, gift-cards
e cashback, e de dois aplicativos: o \emph{Avec}, voltado ao cliente final, e o \emph{Avec Pro}, destinado aos profissionais. Há um plano gratuito ``\emph{Avec Go}'' que inclui funções básicas e cobra apenas a taxa transacional, enquanto os planos pagos variam de R\$ 77 a
R\$249 por mês \cite{Avec}.


Com base no feedback da nossa entidade parceira, destacam-se quatro funcionalidades que
a plataforma \emph{Avec} executa bem:
\begin{itemize}
	\item \emph{Split} instantâneo de comissões, dispensando gateways externos;
	\item Marketplace B2C e aplicativo do cliente, que ampliam a visibilidade do salão e aumentam os agendamentos on-line;
	\item App \emph{Avec Pro} (iOS/Android), no qual o profissional acompanha agenda,
	comissões, saques e registra fotos de “antes e depois” dos serviços.
\end{itemize}

As principais brechas identificadas são:
\begin{itemize}
	\item módulos fiscais avançados (NF-e e SAT) disponíveis apenas nos planos superiores;
	\item dependência do hardware e das tarifas do próprio \emph{Avec Pay} para uso pleno do sistema;
	\item custos adicionais para envios em massa de SMS/WhatsApp em campanhas de marketing;
	\item instabilidade recorrente: o domínio eventualmente fica fora do ar.
\end{itemize}

%seção 1.4 quadro comparativo

\subsection{Quadro comparativo}

\begin{quadro}[htb]
	\caption{\label{frame:comparativo_concorrência}Comparação entre as plataformas concorrentes e a aplicação proposta}
	\footnotesize
	\setlength{\tabcolsep}{4pt}
	\begin{tabular}{|p{6.8cm}|c|c|c|c|}
		\hline
		\textbf{Recurso}                                   & \textbf{Trinks} & \textbf{Gendo} & \textbf{Avec} & \textbf{BS Beauty}\\ \hline 
		Aplicação \textit{web}  & — & \checkmark & \checkmark & \checkmark \\ \hline
		Foco no coworking  & — & \checkmark & — & \checkmark \\ \hline
		Controle de acesso para gestão  & — & \checkmark  & \checkmark & \checkmark \\ \hline
		Agendamento de serviços 100\% on-line & \checkmark & \checkmark & \checkmark & \checkmark \\ \hline
		Controle de conflitos de agenda                    & \checkmark & \checkmark & \checkmark & \checkmark \\ \hline
		Avaliação pós-serviço  & \checkmark & \checkmark  & \checkmark  & \checkmark \\ \hline
		Plataforma do \textit{cliente}                               & \checkmark & \checkmark & \checkmark & \checkmark \\ \hline
		Plataforma do \textit{profissional}                                        & \checkmark & \checkmark & \checkmark & \checkmark \\ \hline
		Confirmação automática (WhatsApp / SMS / e-mail)                    & \checkmark & \checkmark & \checkmark & \checkmark \\ \hline
		Cálculo de \emph{Split} de comissão      & \checkmark & \checkmark & \checkmark & — \\ \hline
		Pagamento on-line                 & \checkmark & — & \checkmark & \ — \\ \hline
		Marketplace B2C                         & \checkmark & — & \checkmark & \ — \\ \hline
		Marketing integrado (envio em massa de SMS/WhatsApp/e-mail)  & \checkmark & — & \checkmark & \checkmark \\ \hline
		Gift-card                     & \checkmark & — & \checkmark & \checkmark \\ \hline
		Programa de indicação  & \checkmark & — & \checkmark & \checkmark \\ \hline
		Relatório financeiro em tempo real    & \checkmark & \checkmark & \checkmark & \checkmark \\ \hline
		Login simplificado com integração Google            & \checkmark & \checkmark &  — & \checkmark \\ \hline
		Fotos “antes e depois” anexadas ao histórico do cliente             & — & \checkmark & \checkmark & \checkmark \\ \hline
		Plano gratuito disponível                                           & \checkmark & — & \checkmark & \checkmark \\ \hline
		Lista de aniversariantes para promoções  & \checkmark & \checkmark & — & \checkmark \\ \hline
		
	\end{tabular}
	\fonte{Produzido pelos autores}
\end{quadro}

