%este arquivo contém todas as seções do capítulo de introdução

Segundo levantamento do Sebrae \cite{Sebrae_2025}
, em 2024 o setor de beleza no Brasil movimentou aproximadamente US\$\,27\,bilhões, colocando o país entre os cinco maiores mercados do mundo nesse ramo. Esse volume financeiro trouxe uma série de novidades e gerou, consequentemente, novas demandas. Diante de tantas mudanças e inovações, tornou-se indispensável que os empreendedores se adaptem rapidamente às tendências.

Conforme publicação do hub \emph{Beauty Fair} (Maior evento no setor da beleza no Brasil), até pouco tempo os profissionais autônomos precisavam deslocar-se até a residência de seus clientes para atendê-los ou firmar parcerias prestando serviços dentro de estabelecimentos de terceiros. Com o surgimento dos coworkings de beleza, esse cenário vem se transformando. A própria \emph{Beauty Fair} esclarece que um coworking de beleza é um espaço compartilhado que oferece infraestrutura para que profissionais da área possam trabalhar e colaborar. Trata-se de um local no qual cabeleireiros, maquiadores, esteticistas, manicures, massoterapeutas e demais especialistas podem alugar um posto de trabalho, dividir recursos e alcançar potenciais clientes \cite{BeautyFair}.

Reportagem online na Gazeta do Povo destaca que o ambiente coworking vem se consolidando como um dos modelos de negócio que mais crescem no Brasil, oferecendo ao profissional autônomo flexibilidade, troca de experiências e uma infraestrutura completa sem burocracia nem custos inesperados \cite{gazeta-coworking}. Nesse ambiente, o prestador de serviços tem o benefício de não precisar arcar com despesas de instalação ou manutenção de um salão próprio; basta utilizar o espaço, atender seus clientes e agendar a próxima sessão, preservando o controle sobre seus horários e ganhos.

À medida que esse formato de trabalho se expande, aumenta também a necessidade de maximizar a autonomia e a rentabilidade de cada profissional. Portanto, surge o desafio de gerir agendas, espaços e custos de forma ágil e intuitiva, evitando conflitos de reserva ou falhas de cobrança. Este projeto propõe-se a desenvolver uma aplicação web que atenda exatamente a essa demanda.

A aplicação web BS Beauty foi desenvolvida especialmente para gerenciar um salão de beleza que opera em modelo coworking, sob a gestão de nossa parceira de extensão Bruna. Seu objetivo principal é otimizar os processos internos e centralizar o agendamento de serviços, atendendo tanto às demandas da gestora quanto às necessidades dos profissionais autônomos.