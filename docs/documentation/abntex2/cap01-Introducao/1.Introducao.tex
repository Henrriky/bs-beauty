%este arquivo contém todas as seções do capítulo de introdução

\section{Objetivos}
% ----------------------------------------------------------
\section{Problema e Solução Proposta}
% ----------------------------------------------------------
\section{Justificativa}
graficos com numeros, expor a relevância da solução - extensao e importancia
% ----------------------------------------------------------
\section{Análise da Concorrência}

Foi conduzida uma pesquisa de mercado centrada em plataformas brasileiras que combinam agendamento on-line e gestão financeira para espaços de beleza no modelo coworking. Deste levantamento emergiram três empresas que servirão de referência nesta análise: uma já amplamente consolidada no mercado nacional — embora atue além do universo coworking — e outras duas que, apesar de conhecidas, ainda estão em expansão, mas com foco mais relacionado ao da nossa proposta, o que as torna concorrentes que merecem maior atenção estratégica.

\subsection{Trinks}

Trinks é uma plataforma já bem consolidada no mercado de gestão de negócios de beleza, com soluções
personalizadas para barbearias, salões de beleza e clínicas de estética. Criada em 2012, é hoje a
plataforma de gestão para beleza com a maior base instalada do país, englobando aproximadamente
2,8\,milhões de usuários e mais de 40\,mil estabelecimentos, sediada no Rio de Janeiro. A plataforma começou como um empreendimento de consultoria em software personalizado, mas logo identificou uma oportunidade no mercado da beleza e mudou de nicho. Em 2024, foi adquirida pelo grupo Stone, o que alavancou ainda mais funcionalidades do aplicativo, como o autoatendimento.
Atualmente, a Trinks oferece software de back-office (conjunto de módulos internos que controlam o funcionamento do negócio como finanças, estoque, comissões e relatórios), marketplace B2C e meios de pagamento próprios (Trinks Pay), funcionando praticamente como um “ERP + iFood” para salões e barbearias. Existe um
plano grátis que engloba apenas 150 agendamentos por mês, e os planos pagos variam de R\$ 59 a R\$ 249/mês \cite{Trinks}.

Além dos serviços comuns — agenda on-line com fila de espera, encaixes, confirmação automática por
WhatsApp/SMS, histórico de clientes e relatórios de funcionários (funcionalidade essencial para
gerar insights e impulsionar o negócio) — seus principais diferenciais são:

\begin{itemize}
	\item Ponto de venda (PDV) completo: integração TEF, Pix e split de comissão, atendendo desde
	MEIs até redes com exigência de NFC-e e SAT/ECF;
	\item Marketplace \textit{Trinks.com}, que gera maior fluxo de clientes, expõe o salão ao
	público final e permite pagamento antecipado;
	\item Estrutura em nuvem madura, com SLA de 99{,}9\,\% e aplicativos nativos para
	iOS/Android.
\end{itemize}

Apesar dos grandes benefícios, identificamos algumas brechas do ponto de vista do negócio da nossa
parceira de extensão, Bruna:

\begin{itemize}
	\item A interface pode ser considerada “poluída” para clientes iniciantes, devido ao grande
	número de funcionalidades;
	\item Há pouco foco no aluguel de estações típico do coworking, exigindo ajustes manuais de
	comissão;
	\item Os planos superiores podem ficar caros para microempreendedores.
\end{itemize}

Informações para o quadro comparativo:\\

- plataforma online com apps de diferentes acessos para profissionais e clientes\\
- gestão de agenda online\\
- relatório de receitas e despesas\\
- relatório de serviços por cliente\\
- integração com whatsapp: disparo de lembrete, confirmação, mensagem de aviversário etc.\\
- pesquisa de satisfação \\
- possui serviços  (coworking ou não) para salão de beleza, barbearia e clínicas de estética \\
- integração com google e redes sociais\\
- mais de 12 anos no mercado \\
- aproximadamente 2.8 milhões de usuários\\
- foi comprada pelo grupo stone\\
- sistema de pagamento integrado\\
- programa de indicação


• Funcionalidades-chave
– Agenda on-line com fila de espera, encaixes e confirmação automática por WhatsApp/SMS\\
– PDV fiscal completo: NFC-e, SAT/ECF, integração TEF, Pix e split de comissão\\
– Controle detalhado de estoque, metas de venda, contratos de serviço e cartões-presente\\
– Marketplace Trinks.com, que expõe o salão ao público final e permite pagamento antecipado\\
– App “Profissional” para agenda pessoal, extrato de comissões e histórico de clientes\\

• Modelos de preço (2025)\\
– Plano Gratuito (até 150 agendamentos/mês, sem NF-e)\\
– Planos pagos de RS 59 a RS 249/mês, que liberam módulos fiscais e relatórios avançados\\
– Trinks Pay: taxas a partir de 1,48  (débito) e 2,88  (crédito), com opção de antecipação\\

• Pontos fortes\\
– Maior efeito de rede: marketplace gera fluxo de novos clientes para o salão\\
– Módulo fiscal robusto, atendendo desde MEIs até redes com exigência de SAT e NFC-e\\
– Estrutura cloud madura, com SLA de 99,9  e apps nativos para iOS/Android\\


\subsection{Gendo}

Lançado em 2017 e sediado em Curitiba-PR, o Gendo se posiciona como um \emph{hub}\footnote{Hub: plataforma centralizada que integra agenda, PDV, finanças e pagamentos em um único ambiente, funcionando como “nó” que organiza os fluxos de dados do negócio.} de gestão 100\,\% em nuvem para negócios além do setor da beleza, como estética, saúde, bem-estar, pet-shop e mais recentemente, espaços em formato coworking. 
Atualmente mantém mais de 10 mil assinantes, com maior penetração nas regiões Sul e Sudeste do Brasil. Foi criado no modelo SaaS com o intuito de oferecer prontamente agenda on-line, automação de lembretes (e-mail/WhatsApp), módulo financeiro completo e integrações com gateways de pagamento (Stone, Cielo e Mercado Pago). Atualmente, os planos são somente pagos e variam de R\$ 32 a R\$ 293/mês, após 14 dias de teste gratuito \cite{Gendo}.

Seus principais diferenciais são:
\begin{itemize}
	\item Caixa do profissional: Módulo pensado para coworking, possibilitando débito automático de aluguel de estação e visualização dos ganhos de cada profissional;
	\item Caixa de vendas on-line: emite nota fiscal eletrônica, aceita Pix e repassa o valor das vendas no cartão de forma automática;
	\item Aplicativo Gendo Pro (iOS e Android): permite ao profissional ver a agenda, acompanhar comissões, pedir saques e registrar fotos de antes e depois dos serviços;
	\item Relatórios instantâneos: exibem ticket médio, previsão de faturamento e dados de cancelamentos, com opção de exportar para Excel.
\end{itemize}


Já os maiores pontos de melhoria identificados são:
\begin{itemize}
	\item Base instalada ainda pequena, limitando o efeito de rede junto a grandes franquias;
	\item Dependência de gateways externos, o que adiciona custo extra ao \emph{split} \footnote{Split é a divisão automática do pagamento entre salão e profissional que, se feita por um gateway externo, gera uma taxa extra.};
	\item Relatórios fiscais avançados disponíveis apenas no plano Premium.
\end{itemize}

informações para quadro comparativo:\\

- app para agendamento\\
- sistema de fidelidade\\
- lembrete via e-mail ou wpp\\
-interface de agenda simples e "muito editavel"\\
- comanda eletronica para clientes\\
- lembrete de contas a pagar para gerente do salão\\
- relatórios financeiros com gráficos de previsão de faturamento\\
- pagamento online\\
-venda de pacotes online\\
-lista de aniversariantes\\
-programa de desconto\\
cálculo de comissões 9caso comissionado e não autonomo)\\
- fotos antes e depois do serviço
-visualização de pendência de pagamentos de clientes
- avaliação final de cliente
-cadastro de fornecedor
-relatório de abandono de clientes\\


• Funcionalidades-chave
– Agenda on-line multi­profissional, com bloqueio de recursos (cadeiras, salas) e confirmação automática por WhatsApp ou SMS
– “Caixa do profissional”: extrato individual que credita comissões e debita despesas fixas (aluguel, materiais) sem planilhas externas
– PDV em nuvem com NFC-e/SAT, Pix integrado e repasse automático de cartão via parceiros adquirentes
– Módulo financeiro: contas a pagar/receber, fluxo de caixa diário e DRE simplificado
– Relatórios em tempo real (faturamento por profissional, ticket médio, no-show) e exportação para Excel
– App “Gendo Pro” (Android/iOS) para que cada profissional acompanhe agenda, comissões e solicitações de saque

• Modelos de preço (2025)
– Teste gratuito de 14 dias, sem limite de usuários
– Planos Basic, Plus e Premium de RS 59 a RS 199/mês, variando por quantidade de usuários, NF-e e marketing
– Gateway externo: taxas a partir de 1,49  (débito) e 2,99  (crédito) + RS 0,40 por transação; split opcional (adição de 0,5 p.p.)

• Pontos fortes
– Módulo “Caixa do profissional” pensado para coworking, possibilitando débito automático de aluguel de estação
– Interface clean, mobile-first, com curva de aprendizado curta para equipes pequenas
– Roadmap aberto e releases mensais, mostrando evolução constante do produto
– Suporte híbrido (chat + WhatsApp) com SLA abaixo de 5 min em horário comercial

• Possíveis brechas
– Base instalada ainda limitada, o que reduz efeito de rede e volume de feedbacks de grandes franquias
– Sem adquirência própria; depende de gateways externos, acrescentando taxa extra ao split
– Relatórios fiscais avançados (centro de custo, contabilidade) disponíveis apenas no plano Premium
– Ausência de marketplace B2C para captação de novos clientes, algo que concorrentes como Trinks oferecem

\subsection{Avec}

A Avec é, hoje, a principal concorrente do nosso projeto, pois a entidade parceira que motivou este trabalho utiliza essa plataforma para gerenciar seu salão de beleza em modelo coworking. Por esse motivo, ela foi adotada como referência: buscamos manter as funcionalidades que já funcionam bem na Avec e, ao mesmo tempo, acrescentar ou aprimorar recursos que ainda fazem falta para a nossa parceira.

Lançada em 2014 e sediada em São Paulo-SP, a Avec se apresenta como solução``360º'' para salões, barbearias, esmaltarias, spas e estúdios de tatuagem. A plataforma integra software de gestão, um sistema próprio de pagamentos (\emph{Avec Pay}) e um marketplace B2C que encaminha novos clientes aos estabelecimentos. Segundo a empresa, mais de 40 mil negócios utilizam o serviço no
Brasil. Também desenvolvida no modelo SaaS, a ferramenta oferece agenda on\-/line multiprofissional com confirmações via WhatsApp ou SMS, ponto de venda completo com TEF, Pix e \emph{split} interno de comissões, além de módulo financeiro integrado. Dispõe ainda de uma carteira digital empregada em pacotes pré\=/pagos, gift-cards
e cashback, e de dois aplicativos: o \emph{Avec}, voltado ao cliente final, e o \emph{Avec Pro}, destinado aos profissionais. Há um plano gratuito ``\emph{Avec Go}'' que inclui funções básicas e cobra apenas a taxa transacional, enquanto os planos pagos variam de R\$ 77 a
R\$249 por mês \cite{Avec}.


Com base no feedback da nossa entidade parceira, destacam-se quatro funcionalidades que
a plataforma \emph{Avec} executa particularmente bem:
\begin{itemize}
	\item \emph{Split} instantâneo de comissões, dispensando gateways externos;
	\item marketplace B2C e aplicativo do cliente, que ampliam a visibilidade do salão e geram agendamentos on-line;
	\item programa de fidelidade “Clube Avec”, cujo sistema de carteira digital reúne pacotes, gift-card e cashback totalmente integrados ao caixa;
	\item app \emph{Avec Pro} (iOS/Android), no qual o profissional acompanha agenda,
	comissões, saques e registra fotos de “antes e depois” dos serviços.
\end{itemize}

As principais brechas identificadas são:
\begin{itemize}
	\item módulos fiscais avançados (NF-e e SAT) disponíveis apenas nos planos superiores;
	\item dependência do hardware e das tarifas do próprio \emph{Avec Pay} para uso pleno do sistema;
	\item custos adicionais para envios em massa de SMS/WhatsApp em campanhas de marketing;
	\item instabilidade recorrente: o domínio eventualmente fica fora do ar.
\end{itemize}


informações para quadro comparativo:

- app para clientes e profissionais\\
- gerenciamento de agenda\\
-relatórios financeiros\\
- pagamento de comissão por avecPay\\

Fundada em 2014 com sede em São Paulo, a Avec se apresenta como uma solução “360º” para salões, barbearias, esmaltarias e spas. Reúne software de gestão, adquirência própria (Avec Pay) e um marketplace B2C que direciona clientes para os estabelecimentos. A empresa divulga atender algo em torno de 10 a 12 mil negócios no Brasil.

• Funcionalidades-chave
– Agenda on-line multi­profissional com confirmação automática por WhatsApp ou SMS
– PDV completo: TEF, Pix, split de comissão e integração com maquininhas próprias
– Controle de estoque, fluxo de caixa e contas a pagar/receber
– Carteira digital e programa de fidelidade “Clube Avec” (pacotes pré-pago, gift-card, cash-back)
– App “Avec Pro” para que cada profissional acompanhe sua própria agenda e comissões
– Painel de marketing (e-mail, SMS, push) e relatórios de desempenho em tempo real

• Modelos de preço (2025)
– Plano gratuito “Avec Go” com funções básicas e cobrança somente de taxa transacional
– Planos pagos entre RS 69 e RS 249/mês, variando por número de usuários e módulos extras
– Taxas Avec Pay a partir de 1,49 (débito) e 2,99 (crédito), com opção de antecipação

• Pontos fortes
– Ecossistema integrado de pagamentos, que simplifica o repasse de comissões
– Marketplace próprio capaz de gerar demanda adicional para o salão
– Interface moderna, mobile-first, e onboarding rápido para pequenos negócios

• Possíveis brechas
– Módulos fiscais avançados (NF-e, SAT) presentes apenas nos planos superiores
– Menor ênfase em locação de estações, usual no modelo coworking, exigindo ajustes manuais
– Dependência de hardware e das taxas do próprio Avec Pay para pleno aproveitamento da plataforma

\subsection{Quadro comparativo}

\begin{quadro}[htb]
	\caption{\label{frame:comparativo_concorrência}Comparação entre as plataformas concorrentes e a aplicação proposta}
	\footnotesize
	\setlength{\tabcolsep}{4pt}
	\begin{tabular}{|p{6.8cm}|c|c|c|c|}
		\hline
		\textbf{Recurso}                                   & \textbf{Trinks} & \textbf{Gendo} & \textbf{Avec} & \textbf{BS Beauty}\\ \hline
		App do \textit{cliente} (iOS/Android)                               & \checkmark & \checkmark & \checkmark & \checkmark \\ \hline
		App do \textit{profissional}                                        & \checkmark & \checkmark & \checkmark & \checkmark \\ \hline
		Agenda on--line multiprofissional                                   & \checkmark & \checkmark & \checkmark & \checkmark \\ \hline
		Confirmação automática (WhatsApp / SMS / e-mail)                    & \checkmark & \checkmark & \checkmark & \checkmark \\ \hline
		PDV fiscal (NFC-e, SAT/ECF, TEF)                                    & \checkmark & \checkmark & \checkmark\textsuperscript{*} & \checkmark \\ \hline
		\emph{Split} de comissão em tempo real                              & \checkmark & \checkmark & \checkmark & \checkmark \\ \hline
		Sistema próprio de pagamentos (sem gateway externo)                 & \checkmark & — & \checkmark & \checkmark \\ \hline
		Marketplace B2C (captação de novos clientes)                        & \checkmark & — & \checkmark & \checkmark \\ \hline
		Módulo “Caixa do profissional” para coworking                       & — & \checkmark & — & \checkmark \\ \hline
		Carteira digital (pacotes, gift-card, cashback)                     & \checkmark & — & \checkmark & \checkmark \\ \hline
		Programa de indicação / fidelidade                                  & \checkmark & — & \checkmark & \checkmark \\ \hline
		Venda on-line de serviços (checkout + NF-e)                         & \checkmark & \checkmark & \checkmark & \checkmark \\ \hline
		Módulo financeiro (contas a pagar / receber, fluxo de caixa)        & \checkmark & \checkmark & \checkmark & \checkmark \\ \hline
		Relatórios e analytics em tempo real                                & \checkmark & \checkmark & \checkmark & \checkmark \\ \hline
		Relatórios fiscais avançados (NF-e, NFC-e, SAT)                     & \checkmark & \checkmark\textsuperscript{*} & \checkmark\textsuperscript{*} & \checkmark \\ \hline
		Marketing integrado (envio em massa de SMS/WhatsApp/e-mail)         & \checkmark & — & \checkmark & \checkmark \\ \hline
		Integrações externas (Google, redes sociais, calendário)            & \checkmark & — & \checkmark & \checkmark \\ \hline
		Gestão de estoque e cadastro de fornecedores                        & \checkmark & — & \checkmark & \checkmark \\ \hline
		Fotos “antes e depois” anexadas ao histórico do cliente             & — & \checkmark & \checkmark & \checkmark \\ \hline
		Plano gratuito disponível                                           & \checkmark & — & \checkmark & \checkmark \\ \hline
	\end{tabular}
	\fonte{Produzido pelos autores}
\end{quadro}

\begin{table}[htb]
	\centering
	\caption{Legenda dos símbolos de disponibilidade}
	\label{tab:legenda_comparativo_concorrência}
	
	\begin{tabular}{|c|p{9cm}|}
		\hline
		\textbf{Símbolo} & \textbf{Significado} \\ \hline
		\checkmark       & Recurso disponível na plataforma \\ \hline
		—                & Recurso inexistente / não suportado \\ \hline
		\textsuperscript{*} & Recurso disponível apenas em planos superiores \\ \hline
	\end{tabular}
	
	\vspace{0.5em}
	{\footnotesize Fonte: Produzido pelos autores\par}
\end{table}

