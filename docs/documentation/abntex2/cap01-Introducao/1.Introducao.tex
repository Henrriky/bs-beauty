%este arquivo contém todas as seções do capítulo de introdução

\section{Objetivos}
% ----------------------------------------------------------
\section{Problema e Solução Proposta}
% ----------------------------------------------------------
\section{Justificativa}
graficos com numeros, expor a relevância da solução - extensao e importancia
% ----------------------------------------------------------
\section{Análise da Concorrência}

Foi conduzida uma pesquisa de mercado centrada em plataformas brasileiras que combinam agendamento on-line e gestão financeira para espaços de beleza no modelo coworking. Deste levantamento emergiram três empresas que servirão de referência nesta análise: uma já amplamente consolidada no mercado nacional — embora atue além do universo coworking — e outras duas que, apesar de conhecidas, ainda estão em expansão, mas com foco mais relacionado ao da nossa proposta, o que as torna concorrentes que merecem maior atenção estratégica.

\subsection{Trinks}

%https://negocios.trinks.com/ 

Criada em 2012, a Trinks é hoje a plataforma de gestão para beleza com a maior base instalada do país (aproximadamente 25 mil estabelecimentos). Oferece software de back-office, marketplace B2C e meios de pagamento próprios (Trinks Pay), funcionando praticamente como um “ERP + iFood” para salões e barbearias.

- plataforma online com apps de diferentes acessos para profissionais e clientes\\
- gestão de agenda online\\
- relatório de receitas e despesas\\
- relatório de serviços por cliente\\
- integração com whatsapp: disparo de lembrete, confirmação, mensagem de aviversário etc.\\
- pesquisa de satisfação \\
- possui serviços  (coworking ou não) para salão de beleza, barbearia e clínicas de estética \\
- integração com google e redes sociais\\
- mais de 12 anos no mercado \\
- aproximadamente 2.8 milhões de usuários\\
- foi comprada pelo grupo stone\\
- sistema de pagamento integrado\\
- programa de indicação

• Sede: Rio de Janeiro – RJ

• Funcionalidades-chave
– Agenda on-line com fila de espera, encaixes e confirmação automática por WhatsApp/SMS
– PDV fiscal completo: NFC-e, SAT/ECF, integração TEF, Pix e split de comissão
– Controle detalhado de estoque, metas de venda, contratos de serviço e cartões-presente
– Marketplace Trinks.com, que expõe o salão ao público final e permite pagamento antecipado
– App “Profissional” para agenda pessoal, extrato de comissões e histórico de clientes

• Modelos de preço (2025)
– Plano Gratuito (até 150 agendamentos/mês, sem NF-e)
– Planos pagos de RS 59 a RS 249/mês, que liberam módulos fiscais e relatórios avançados
– Trinks Pay: taxas a partir de 1,48  (débito) e 2,88  (crédito), com opção de antecipação

• Pontos fortes
– Maior efeito de rede: marketplace gera fluxo de novos clientes para o salão
– Módulo fiscal robusto, atendendo desde MEIs até redes com exigência de SAT e NFC-e
– Estrutura cloud madura, com SLA de 99,9  e apps nativos para iOS/Android

• Possíveis brechas
– Interface considerada “poluída” por alguns usuários iniciantes
– Pouco foco no aluguel de estações típico do coworking; requer ajustes de comissão manuais
– Planos superiores podem ficar caros para microempreendedores individuais


\subsection{Gendo}

%referencias: https://www.gendo.com.br/ 

- app para agendamento\\
- sistema de fidelidade\\
- lembrete via e-mail ou wpp\\
-interface de agenda simples e "muito editavel"\\
- comanda eletronica para clientes\\
- lembrete de contas a pagar para gerente do salão\\
- relatórios financeiros com gráficos de previsão de faturamento\\
- pagamento online\\
-venda de pacotes online\\
-lista de aniversariantes\\
-programa de desconto\\
cálculo de comissões 9caso comissionado e não autonomo)\\
- fotos antes e depois do serviço
-visualização de pendência de pagamentos de clientes
- avaliação final de cliente
-cadastro de fornecedor
-relatório de abandono de clientes\\
Lançado em 2017, o Gendo se posiciona como um “hub” de gestão para salões, barbearias, clínicas de estética e, mais recentemente, espaços de beleza em formato coworking. O produto é 100 por cento cloud, inclui app mobile para profissionais e integrações com gateways de pagamento (Stone, Cielo e Mercado Pago). A empresa divulga algo entre 2 000 e 3 500 licenças ativas, crescendo sobretudo nas regiões Sul e Sudeste.

• Sede: Curitiba – PR

• Funcionalidades-chave
– Agenda on-line multi­profissional, com bloqueio de recursos (cadeiras, salas) e confirmação automática por WhatsApp ou SMS
– “Caixa do profissional”: extrato individual que credita comissões e debita despesas fixas (aluguel, materiais) sem planilhas externas
– PDV em nuvem com NFC-e/SAT, Pix integrado e repasse automático de cartão via parceiros adquirentes
– Módulo financeiro: contas a pagar/receber, fluxo de caixa diário e DRE simplificado
– Relatórios em tempo real (faturamento por profissional, ticket médio, no-show) e exportação para Excel
– App “Gendo Pro” (Android/iOS) para que cada profissional acompanhe agenda, comissões e solicitações de saque

• Modelos de preço (2025)
– Teste gratuito de 14 dias, sem limite de usuários
– Planos Basic, Plus e Premium de RS 59 a RS 199/mês, variando por quantidade de usuários, NF-e e marketing
– Gateway externo: taxas a partir de 1,49  (débito) e 2,99  (crédito) + RS 0,40 por transação; split opcional (adição de 0,5 p.p.)

• Pontos fortes
– Módulo “Caixa do profissional” pensado para coworking, possibilitando débito automático de aluguel de estação
– Interface clean, mobile-first, com curva de aprendizado curta para equipes pequenas
– Roadmap aberto e releases mensais, mostrando evolução constante do produto
– Suporte híbrido (chat + WhatsApp) com SLA abaixo de 5 min em horário comercial

• Possíveis brechas
– Base instalada ainda limitada, o que reduz efeito de rede e volume de feedbacks de grandes franquias
– Sem adquirência própria; depende de gateways externos, acrescentando taxa extra ao split
– Relatórios fiscais avançados (centro de custo, contabilidade) disponíveis apenas no plano Premium
– Ausência de marketplace B2C para captação de novos clientes, algo que concorrentes como Trinks oferecem

\subsection{Avec}

% referencia: https://negocios.avec.app/ 

- app para clientes e profissionais\\
- gerenciamento de agenda\\
-relatórios financeiros\\
- pagamento de comissão por avecPay\\

• Sede: São Paulo – SP
• Visão geral: Fundada em 2014, a Avec se apresenta como uma solução “360º” para salões, barbearias, esmaltarias e spas. Reúne software de gestão, adquirência própria (Avec Pay) e um marketplace B2C que direciona clientes para os estabelecimentos. A empresa divulga atender algo em torno de 10 a 12 mil negócios no Brasil.

• Funcionalidades-chave
– Agenda on-line multi­profissional com confirmação automática por WhatsApp ou SMS
– PDV completo: TEF, Pix, split de comissão e integração com maquininhas próprias
– Controle de estoque, fluxo de caixa e contas a pagar/receber
– Carteira digital e programa de fidelidade “Clube Avec” (pacotes pré-pago, gift-card, cash-back)
– App “Avec Pro” para que cada profissional acompanhe sua própria agenda e comissões
– Painel de marketing (e-mail, SMS, push) e relatórios de desempenho em tempo real

• Modelos de preço (2025)
– Plano gratuito “Avec Go” com funções básicas e cobrança somente de taxa transacional
– Planos pagos entre RS 69 e RS 249/mês, variando por número de usuários e módulos extras
– Taxas Avec Pay a partir de 1,49 (débito) e 2,99 (crédito), com opção de antecipação

• Pontos fortes
– Ecossistema integrado de pagamentos, que simplifica o repasse de comissões
– Marketplace próprio capaz de gerar demanda adicional para o salão
– Interface moderna, mobile-first, e onboarding rápido para pequenos negócios

• Possíveis brechas
– Módulos fiscais avançados (NF-e, SAT) presentes apenas nos planos superiores
– Menor ênfase em locação de estações, usual no modelo coworking, exigindo ajustes manuais
– Dependência de hardware e das taxas do próprio Avec Pay para pleno aproveitamento da plataforma

\subsection{Quadro comparativo}
% Conteúdo do subtópico 1.4.4
exemplo de quadro:
\begin{quadro}[htb]
	\caption{\label{quadro_exemplo}Exemplo de quadro}
	\begin{tabular}{|c|c|c|c|}
		\hline
		\textbf{Pessoa} & \textbf{Idade} & \textbf{Peso} & \textbf{Altura} \\ \hline
		Marcos & 26    & 68   & 178    \\ \hline
		Ivone  & 22    & 57   & 162    \\ \hline
		...    & ...   & ...  & ...    \\ \hline
		Sueli  & 40    & 65   & 153    \\ \hline
	\end{tabular}
	\fonte{Autor.}
\end{quadro}