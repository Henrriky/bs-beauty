\section{Problema e Solução Proposta}

A gestão de um salão por pequenos empreendedores é frequentemente desafiadora. Ademais, demandas surgem e muitas vezes são realizadas manualmente. Portanto, quando alguma etapa falha, evidencia‐se a necessidade de uma solução digital capaz de reduzir erros e diminuir o esforço administrativo.

Por isso, o objetivo geral do projeto é suprir as necessidades de um salão de beleza em modelo \emph{coworking} de forma ágil. Como explicado anteriormente, esse modelo de trabalho é recente (popularizado após a pandemia de \gls{covid} em 2020) e atende diferentes profissionais autônomos (relacionados à gerente por locação ou comissão), não uma equipe com objetivo comum. Desta forma, o problema central é gerenciar a ocupação de cada profissional no espaço de trabalho, além de controlar as finanças e a agenda de clientes.

Nossa parceira Bruna já utilizava um sistema digital para gerenciamento do salão. Contudo, apesar dos benefícios trazidos pela solução, o sistema apresentava pontos insatisfatórios, sendo o principal deles a instabilidade da plataforma, que gerava insatisfação e perda de clientes.

Nossa solução consiste em criar uma aplicação \emph{web} que mantenha todas as funcionalidades que já atendem bem a Bruna como o agendamento \emph{on-line} e pesquisa de satisfação. Além disso, a plataforma incluirá funções ainda ausentes e ajustará requisitos funcionais e não funcionais cuja concepção é adequada, mas apresenta falhas, como o \emph{login} instável, senhas excessivamente complexas e erros recorrentes na troca de senha.

Em síntese, a solução proposta é uma plataforma com \emph{login} simplificado (integrado ao \gls{sso} - \emph{Single Sign-On} \footnote{Single Sign-On é um sistema que permite usar um único nome de usuário e senha para acessar vários serviços diferentes, sem precisar criar contas ou lembrar várias senhas.} do Google) e agendamento fácil e transparente para os clientes (incluindo todos os serviços e atributos necessários para uma melhor decisão). Também contará com política de não comparecimento, agenda totalmente controlada pelos profissionais, notificações de agendamento e cancelamento para clientes e profissionais, lista de aniversariantes, desconto por frequência e retenção de dados em conformidade com a \gls{lgpd}. Além disso, a gerente terá acesso à relatórios financeiros e \emph{dashboards} com métricas de produtividade.
