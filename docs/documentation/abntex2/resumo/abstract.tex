	\begin{resumo}[ABSTRACT]
	\begin{otherlanguage*}{english}
		
		This integrated extension project presents the development of the BS Beauty web application, aimed at optimizing the management and scheduling of beauty services in a coworking environment under the responsibility of the manager, our extension partner. As part of a competitive initiative, the developed digital solution aims to improve service and foster customer loyalty by reducing the time spent on administrative tasks and ensuring an intuitive user interface. Therefore, the system centralizes schedules, prevents scheduling conflicts and provides automatic notifications, financial reports and performance dashboards. In order to meet our partner’s demands, constant communication was essential for requirements gathering, competitor analysis and definition of business rules. For project phase management, the agile framework Scrum was adopted, however the planning and task control was formalized in ProjectLibre, a project management software that allows managing schedules and allocating resources to tasks. The application architecture was designed in layers, detailed in component and deployment diagrams. In parallel, the test plan was developed, documentation was standardized and financial viability was assessed in realistic, optimistic and pessimistic scenarios. As a result, the developed application strengthens the theoretical knowledge of the undergraduate course, aligns it with market demands and promotes technological innovation in the beauty sector by supporting the coworking model.
		
		\textbf{Keywords:} web application. online scheduling. beauty coworking. service management. Scrum.
		
	\end{otherlanguage*}
\end{resumo}