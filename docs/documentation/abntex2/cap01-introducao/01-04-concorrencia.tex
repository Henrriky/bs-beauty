\section{Análise da Concorrência}
\label{sec:analise-concorrencia}
Foi conduzida uma pesquisa de mercado centrada em plataformas brasileiras que combinam agendamento \emph{on-line} e gestão financeira para espaços de beleza no modelo \emph{coworking}. Deste levantamento emergiram três empresas que servirão de referência nesta análise: uma já amplamente consolidada no mercado nacional — embora atue além do universo \emph{coworking} — e outras duas que, apesar de conhecidas, ainda estão em expansão, mas com foco mais relacionado ao da nossa proposta, o que as torna concorrentes que merecem maior atenção estratégica.

\subsection{Trinks}

%inicio de figura
\begin{figure}[htb]
	\centering
	\caption{Logo plataforma Trinks}
	\includegraphics[width=0.3\textwidth]{cap01-Introducao/Images/1.4.1_Trinks}
	\label{fig:Trinks}
	\fonte{\cite{Trinks}}
\end{figure}

 \FloatBarrier

Trinks é uma plataforma já bem consolidada no mercado de gestão de negócios de beleza, com soluções personalizadas para barbearias, salões de beleza e clínicas de estética. Criada em 2012, é hoje a plataforma de gestão para beleza com a maior base instalada do país, englobando aproximadamente 2,8\,milhões de usuários e mais de 40\,mil estabelecimentos, sediada no Rio de Janeiro. A plataforma começou como um empreendimento de consultoria em software personalizado, mas logo identificou uma oportunidade no mercado da beleza e mudou de nicho. Em 2024, foi adquirida pelo grupo Stone, o que alavancou ainda mais funcionalidades do aplicativo, como o autoatendimento.

Atualmente, a Trinks oferece software de \emph{back-office} (conjunto de módulos internos que controlam o funcionamento do negócio como finanças, estoque, comissões e relatórios), \emph{marketplace \gls{b2c}} e meios de pagamento próprios (Trinks \emph{Pay}), funcionando praticamente como um “\gls{erp} + \emph{iFood}” para salões e barbearias. Existe um plano grátis que engloba apenas 150 agendamentos por mês, e os planos pagos variam de R\$ 59 a R\$ 249/mês \cite{Trinks}.

Além dos serviços comuns, seus principais diferenciais são:

\begin{itemize}
	\item \textbf{\gls{pdv} completo:} integração com \gls{tef}, \gls{pix} e split de comissão, atendendo desde \gls{mei}s até redes com exigência de \gls{nfce} e \gls{sat}/\gls{ecf};
	
	\item \textbf{Estrutura em nuvem madura:} Conta com \gls{sla} de 99,9\,\% e aplicativos nativos para \gls{ios}/Android;
	
	\item \textbf{\emph{Marketplace}:} O domínio \textit{Trinks.com} gera maior fluxo de clientes, expõe o salão ao	público final e permite pagamento antecipado.
\end{itemize}

Apesar dos grandes benefícios, identificamos algumas brechas do ponto de vista do negócio da nossa parceira de extensão, Bruna:

\begin{itemize}
	\item \textbf{Complexidade visual da interface:} Devido ao grande número de funcionalidades voltadas para diversos públicos, a plataforma apresenta uma curva de aprendizado elevada para gestores focados exclusivamente em \emph{coworking}. Para superar essa barreira, o \emph{BS Beauty} propõe uma interface minimalista e focada nas operações essenciais do modelo de negócio da parceira, garantindo maior agilidade e uma experiência de uso intuitiva;
	
	\item \textbf{Baixo foco no aluguel de estações:} A gestão de locação de espaços não é uma funcionalidade nativa, exigindo configurações manuais de comissão que são pouco eficientes e suscetíveis a erros. Em contrapartida, o \emph{BS Beauty} é projetado com o aluguel de estações como funcionalidade central. O sistema permitirá a configuração de diferentes modelos de locação (por hora, dia ou percentual), automatizando o cálculo e o débito dos valores devidos, o que elimina a necessidade de controles manuais;
	
	\item \textbf{Segregação de funcionalidades em planos superiores:} Recursos importantes para uma gestão completa ficam restritos aos planos mais caros. Já a proposta do \emph{BS Beauty} visa incluir todas as funcionalidades essenciais para a gestão de coworking em um único plano, oferecendo uma solução completa desde o início, sem custos inesperados para acessar ferramentas estratégicas.
	
\end{itemize}

\subsection{Gendo}

%inicio de figura
\begin{figure}[htb]
	\centering
	\caption{Logo plataforma Gendo}
	\includegraphics[width=0.3\textwidth]{cap01-Introducao/Images/1.4.2_Gendo}
	\label{fig:Gendo}
	\fonte{\cite{Gendo}}
\end{figure}

 \FloatBarrier

Lançado em 2017 e sediado em Curitiba-PR, o Gendo se posiciona como um hub\footnote{Hub: plataforma centralizada que integra agenda, \gls{pdv}, finanças e pagamentos em um único ambiente, funcionando como “nó” que organiza os fluxos de dados do negócio.} de gestão 100\,\% em nuvem para negócios além do setor da beleza, como estética, saúde, bem-estar, \emph{pet-shop} e mais recentemente, espaços em formato \emph{coworking}. 
Atualmente mantém mais de 10 mil assinantes, com maior penetração nas regiões Sul e Sudeste do Brasil. Foi criado no modelo \gls{saas} com o intuito de oferecer prontamente agenda \emph{on-line}, automação de lembretes (\emph{e-mail/WhatsApp}), módulo financeiro completo e integrações com \emph{gateways} de pagamento (Stone, Cielo e Mercado Pago). Atualmente, os planos são somente pagos e variam de R\$ 32 a R\$ 293/mês, após 14 dias de teste gratuito \cite{Gendo}.

Seus principais diferenciais são:

\begin{itemize}
	\item \textbf{Caixa do profissional:} Módulo pensado para \emph{coworking}, possibilitando débito automático de aluguel de estação e visualização dos ganhos de cada profissional;
	
	\item \textbf{Aplicativo Gendo Pro (\gls{ios} e Android):} É permitido ao profissional ver a agenda, acompanhar comissões, pedir saques e registrar fotos de antes e depois dos serviços;
	
	\item \textbf{Relatórios instantâneos:} São exibidos \emph{ticket} médio, previsão de faturamento e dados de cancelamentos, com opção de exportar para \emph{Excel}.
\end{itemize}


Já os maiores pontos de melhoria identificados são:

\begin{itemize}
	\item \textbf{Dependência de gateways externos:} A integração com gateways de pagamento externos para realizar o \emph{split}\footnote{\emph{Split} é a divisão automática do pagamento entre salão e profissional que, se feita por um \emph{gateway} externo, gera uma taxa extra.} adiciona custos transacionais e complexidade à operação. Para endereçar essa questão, o \emph{BS Beauty} irá abstrair a camada de transação financeira. O sistema será responsável apenas pelo cálculo detalhado dos valores devidos e pelo registro do método de pagamento, enquanto a liquidação financeira ocorrerá de forma presencial. Essa abordagem estratégica elimina os custos com taxas de gateway e simplifica o fluxo de caixa, focando o sistema na gestão e conciliação dos recebimentos;
	
	
	\item \textbf{Acesso restrito a relatórios fiscais:} Funcionalidades de análise fiscal mais aprofundadas estão disponíveis apenas no plano \emph{Premium}. O \emph{BS Beauty}, por sua vez, disponibilizará um painel de relatórios financeiros e fiscais detalhados como parte de sua oferta principal, permitindo que a gestão tenha acesso a insights estratégicos sem a necessidade de um upgrade.
	
\end{itemize}

\subsection{Avec}

\begin{figure}[htb]
	\centering
	\caption{Logo plataforma Avec}
	\includegraphics[width=0.3\textwidth]{cap01-Introducao/Images/1.4.3_Avec}
	\label{fig:Avec}
	\fonte{\cite{Avec}}
\end{figure}

 \FloatBarrier

Atualmente, a Avec é a principal concorrente do nosso projeto, pois a entidade parceira que motivou este trabalho utiliza essa plataforma para gerenciar seu salão de beleza em modelo \emph{coworking}. Por esse motivo, ela foi adotada como referência: buscamos manter as funcionalidades que já funcionam bem na Avec e, ao mesmo tempo, acrescentar ou aprimorar recursos que ainda fazem falta para a nossa parceira.

Lançada em 2014 e sediada em São Paulo-SP, a Avec se apresenta como solução``360º'' para salões, barbearias, esmaltarias, spas e estúdios de tatuagem. A plataforma integra software de gestão, um sistema próprio de pagamentos (\emph{Avec Pay}) e um \emph{marketplace \gls{b2c}} que encaminha novos clientes aos estabelecimentos. Segundo a empresa, mais de 40 mil negócios utilizam o serviço no
Brasil. Também desenvolvida no modelo \gls{saas}, a ferramenta oferece agenda \emph{on-line} multiprofissional com confirmações via \emph{WhatsApp} ou \gls{sms}, \gls{pdv} completo com \gls{tef}, \gls{pix} e \emph{split} interno de comissões, além de módulo financeiro integrado. Dispõe ainda de uma carteira digital empregada em pacotes pré-pagos, \emph{gift-cards}, \emph{cashback}, e possui dois aplicativos: o \emph{Avec}, voltado ao cliente final, e o \emph{Avec Pro}, destinado aos profissionais. Há um plano gratuito ``\emph{Avec Go}'' que inclui funções básicas e cobra apenas a taxa transacional, enquanto os planos pagos variam de R\$ 77 a
R\$249 por mês \cite{Avec}.


Com base no \emph{feedback} da nossa entidade parceira, destacam-se três funcionalidades que a plataforma \emph{Avec} executa bem:

\begin{itemize}
	\item \textbf{\emph{Split} instantâneo de comissões:} A estratégia elimina gastos ao dispensar \emph{gateways} externos;
	
	\item \textbf{\emph{Marketplace} \gls{b2c} e aplicativo do cliente:} Plataformas que ampliam a visibilidade do salão e aumentam os agendamentos \emph{on-line};
	
	\item \textbf{Aplicativo \emph{Avec Pro} (\gls{ios}/Android):} O profissional pode acompanhar facilmente agenda, comissões, saques, e registrar fotos de “antes e depois” dos serviços.
\end{itemize}

As principais brechas identificadas são:

\begin{itemize}
	   \item \textbf{Acesso restrito a módulos fiscais:} Recursos como emissão de \gls{nfe} e integração \gls{sat} estão disponíveis apenas nos planos superiores. A solução proposta pelo \emph{BS Beauty} busca democratizar o acesso a essas ferramentas, integrando relatórios fiscais essenciais na versão padrão do sistema;
	
	
	\item \textbf{Dependência de hardware e tarifas próprias:} O uso pleno do sistema está atrelado ao ecossistema \emph{Avec Pay}, limitando a escolha de equipamentos e taxas. O \emph{BS Beauty} será desenvolvido com uma arquitetura mais flexível, não se relacionando com a liquidação de pagamentos e não atrelando o uso do sistema a um hardware específico, o que confere maior liberdade ao gestor;
	
	
	\item \textbf{Custos adicionais para comunicação:} Campanhas de \emph{marketing} via \gls{sms} ou \emph{WhatsApp} em massa geram cobranças extras. Em contrapartida, o \emph{BS Beauty} inclui um módulo de comunicação que utiliza APIs de baixo custo, integrando o envio de notificações ao plano principal para que o salão possa se comunicar com seus clientes de forma mais acessível;
	
	\item \textbf{Instabilidade recorrente da plataforma:} Conforme relatado pela parceira, o sistema apresenta indisponibilidade periódica. Um dos pilares do desenvolvimento do \emph{BS Beauty} é a garantia de alta disponibilidade, utilizando uma infraestrutura em nuvem moderna e robusta para assegurar a estabilidade e a confiabilidade da plataforma.
\end{itemize}

%seção 1.4 quadro comparativo

\subsection{Quadro comparativo}

\begin{quadro}[htb]
	\caption{\label{frame:comparativo_concorrência}Comparação entre as plataformas concorrentes e a aplicação proposta}
	\footnotesize
	\setlength{\tabcolsep}{4pt}
	\renewcommand{\arraystretch}{1.2}
	\begin{tabular}{|p{6.8cm}|c|c|c|c|}
		\hline
		\rowcolor{myblue}\textbf{Recurso} & \textbf{Trinks} & \textbf{Gendo} & \textbf{Avec} & \textbf{BS Beauty}
		\\ \hline
		
		\rowcolor{gray!30}\multicolumn{5}{|l|}{\textbf{Gestão e Operações Essenciais}} \\ \hline
		Aplicação \textit{web} & \checkmark & \checkmark & \checkmark & \checkmark \\ \hline
		Agendamento 100\% \emph{on-line} & \checkmark & \checkmark & \checkmark & \checkmark \\ \hline
		Plataformas do Cliente e do Profissional & \checkmark & \checkmark & \checkmark & \checkmark \\ \hline
		Controle de Conflitos de Agenda & \checkmark & \checkmark & \checkmark & \checkmark 
		\\ \hline
		
		\rowcolor{gray!30}\multicolumn{5}{|l|}{\textbf{Diferenciais Estratégicos para Coworking}} \\ \hline
		Modelo de Negócio Focado em Coworking & — & \checkmark & \checkmark & \checkmark \\ \hline
		Gestão Avançada de Aluguel (hora/dia/\%) & — & \checkmark & — & \checkmark \\ \hline
		Interface Intuitiva e Focada & — & — & — & \checkmark \\ \hline
		Alta Disponibilidade e Estabilidade & \checkmark & \checkmark & — & \checkmark 
		\\ \hline
		
		\rowcolor{gray!30}\multicolumn{5}{|l|}{\textbf{Financeiro e Administrativo}} \\ \hline
		Cálculo de \emph{Split} de Comissão & \checkmark & \checkmark & \checkmark & \checkmark \\ \hline
		Processamento de Pagamento On-line & \checkmark & \checkmark & \checkmark & — \\ \hline
		Gestão de Pagamentos Presenciais & — & — & — & \checkmark \\ \hline
		Zero Taxas de Gateway & — & — & — & \checkmark \\ \hline
		Independência de Hardware de Pagamento & \checkmark & \checkmark & — & \checkmark \\ \hline
		Relatórios Financeiros em Tempo Real & \checkmark & \checkmark & \checkmark & \checkmark \\ \hline
		Relatórios Fiscais Essenciais Inclusos & — & — & — & \checkmark 
		\\ \hline
		
		\rowcolor{gray!30}\multicolumn{5}{|l|}{\textbf{Marketing e Modelo de Negócio}} \\ \hline
		\emph{Marketplace} \gls{b2c} & \checkmark & — & \checkmark & — \\ \hline
		Marketing Integrado (Notificações) & \checkmark & \checkmark & — & \checkmark \\ \hline
		\gls{sso} Google & \checkmark & \checkmark & — & \checkmark \\ \hline
		Plano Gratuito Disponível & \checkmark & — & \checkmark & — \\ \hline
		Plano Único com Todas as Funcionalidades & — & — & — & \checkmark \\ \hline
	\end{tabular}
	\fonte{Produzido pelos autores}
\end{quadro}
