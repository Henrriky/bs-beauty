\section{Problema e Solução Proposta}

A gestão de um salão por pequenos empreendedores é frequentemente desafiadora. Ademais, demandas surgem e muitas vezes são realizadas manualmente. Portanto, quando alguma etapa falha, evidencia‐se a necessidade de uma solução digital capaz de reduzir erros e diminuir o esforço administrativo.

Por isso, o objetivo geral do projeto é suprir as necessidades de um salão de beleza em modelo \emph{coworking} de forma ágil. Como explicado anteriormente, esse modelo de trabalho é recente (popularizado após a pandemia de \gls{covid} em 2020) e atende diferentes profissionais autônomos (relacionados à gerente por locação ou comissão), não uma equipe com objetivo comum. Desta forma, o problema central é a gerência da ocupação de cada profissional no espaço de trabalho, além do controle das finanças e da agenda dos clientes.

Nossa parceira Bruna já utilizava um sistema digital para gerenciamento do salão. Contudo, apesar dos benefícios trazidos pela solução, o sistema apresentava pontos insatisfatórios, sendo o principal deles a instabilidade da plataforma, que gerava insatisfação e perda de clientes.

Nossa solução consiste em criar uma aplicação \emph{web} que mantenha todas as funcionalidades que já atendem bem a Bruna como o agendamento \emph{on-line} e pesquisa de satisfação. Além disso, a plataforma incluirá funções ainda ausentes e ajustará requisitos funcionais e não funcionais cuja concepção é adequada, mas apresenta falhas, como o \emph{login} instável, senhas excessivamente complexas e erros recorrentes na troca de senha. De forma específica, nossa solução facilita o agendamento de serviços para as três entidades existentes no \emph{coworking} de beleza: 

\noindent\textbf{Para os Clientes Finais:} A plataforma possibilita o agendamento de serviços de forma intuitiva e flexível. Os clientes poderão escolher profissionais específicos ou optar pelo melhor horário disponível, visualizando facilmente a lista de prestadores, seus serviços, preços, tempo de execução e agendas atualizadas.

\noindent\textbf{Para os Profissionais Autônomos:} O sistema BS Beauty tem como propósito reforçar a autonomia dos profissionais sobre sua agenda e finanças. A aplicação permite bloquear horários, editar preços e a duração dos serviços, além de acompanhar os agendamentos realizados (sejam eles do dia, futuros ou passados) e visualizar relatórios detalhados com a receita gerada pelos serviços prestados.

\noindent\textbf{Para a Gestora:} Nossa parceira, Bruna, terá acesso a funcionalidades exclusivas que incluem análise de métrica de desempenho (a partir de \emph{dashboards)}, gerenciamento do aluguel ou comissão de cada profissional, visualização do fluxo de agendamentos em períodos específicos, envio de mensagens de \emph{marketing} e promoções aos clientes, e acesso a relatórios financeiros detalhados. Ademais, a gestora poderá incluir ou remover profissionais da plataforma conforme a necessidade.

Em síntese, a solução proposta é uma plataforma com \emph{login} simplificado (integrado ao \gls{sso} \footnote{Single Sign-On é um sistema que permite usar um único nome de usuário e senha para acessar vários serviços diferentes, sem precisar criar contas ou lembrar várias senhas.} do Google) e agendamento fácil e transparente para os clientes (incluindo todos os serviços e atributos necessários para uma melhor decisão). Também contará com agenda totalmente controlada pelos profissionais, notificações de agendamento e cancelamento para clientes e profissionais, lista de aniversariantes, desconto por frequência e retenção de dados em conformidade com a \gls{lgpd}. Além disso, a gerente terá acesso à relatórios financeiros e \emph{dashboards} com métricas de produtividade e frequência de clientes.