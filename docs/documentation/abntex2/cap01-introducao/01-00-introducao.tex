%este arquivo contém todas as seções do capítulo de introdução

Segundo levantamento do \gls{sebrae} 
, em 2024 o setor de beleza no Brasil movimentou aproximadamente US\$\,27\,bilhões, colocando o país entre os cinco maiores mercados do mundo nesse ramo. Esse volume financeiro trouxe uma série de novidades e gerou, consequentemente, novas demandas. Diante de tantas mudanças e inovações, tornou-se indispensável que os empreendedores se adaptem rapidamente às tendências \cite{Sebrae_2024}.

Conforme publicação do hub \emph{Beauty Fair} (Maior evento no setor da beleza no Brasil), até pouco tempo atrás, os profissionais autônomos precisavam deslocar-se até a residência de seus clientes para atendê-los ou firmar parcerias prestando serviços dentro de estabelecimentos de terceiros. Com o surgimento dos \emph{coworkings} de beleza, esse cenário vem se transformando. A própria \emph{Beauty Fair} esclarece que um \emph{coworking} de beleza é um espaço compartilhado que oferece infraestrutura para que profissionais da área possam trabalhar e colaborar. Trata-se de um local no qual cabeleireiros, maquiadores, esteticistas, manicures, massoterapeutas e demais especialistas podem alugar um posto de trabalho, dividir recursos e alcançar potenciais clientes \cite{BeautyFair}.

Reportagem \emph{on-line} na Gazeta do Povo destaca que o ambiente \emph{coworking} vem se consolidando como um dos modelos de negócio que mais crescem no Brasil, oferecendo ao profissional autônomo flexibilidade, troca de experiências e uma infraestrutura completa sem burocracia nem custos inesperados \cite{gazeta-coworking}. Nesse ambiente, o prestador de serviços tem o benefício de não precisar arcar com despesas de instalação ou manutenção de um salão próprio; basta utilizar o espaço, atender seus clientes e agendar a próxima sessão, preservando o controle sobre seus horários e ganhos.

À medida que esse formato de trabalho se expande, aumenta também a necessidade de maximizar a autonomia e a rentabilidade de cada profissional. Portanto, surge o desafio de gerir agendas, espaços e custos de forma ágil e intuitiva, evitando conflitos de reserva ou falhas de cobrança. Este projeto propõe-se a desenvolver uma aplicação web que atenda exatamente a essa demanda.

\section{Objetivos}

A aplicação \emph{web} BS Beauty foi desenvolvida especialmente para gerenciar um salão de beleza que opera em modelo \emph{coworking}, sob a gestão da nossa parceira de extensão Bruna. Seu objetivo principal é otimizar os processos internos e centralizar o agendamento de serviços, atendendo tanto às demandas administrativas da gestora quanto às necessidades logísticas dos profissionais autônomos, e sugestões dos clientes finais.
\section{Problema e Solução Proposta}

A gestão de um salão por pequenos empreendedores é frequentemente desafiadora. Ademais, demandas surgem e muitas vezes são realizadas manualmente. Portanto, quando alguma etapa falha, evidencia‐se a necessidade de uma solução digital capaz de reduzir erros e diminuir o esforço administrativo.

Por isso, o objetivo geral do projeto é suprir as necessidades de um salão de beleza em modelo \emph{coworking} de forma ágil. Como explicado anteriormente, esse modelo de trabalho é recente (popularizado após a pandemia de \gls{covid} em 2020) e atende diferentes profissionais autônomos (relacionados à gerente por locação ou comissão), não uma equipe com objetivo comum. Desta forma, o problema central é a gerência da ocupação de cada profissional no espaço de trabalho, além do controle das finanças e da agenda dos clientes.

Nossa parceira Bruna já utilizava um sistema digital para gerenciamento do salão. Contudo, apesar dos benefícios trazidos pela solução, o sistema apresentava pontos insatisfatórios, sendo o principal deles a instabilidade da plataforma, que gerava insatisfação e perda de clientes.

Nossa solução consiste em criar uma aplicação \emph{web} que mantenha todas as funcionalidades que já atendem bem a Bruna como o agendamento \emph{on-line} e pesquisa de satisfação. Além disso, a plataforma incluirá funções ainda ausentes e ajustará requisitos funcionais e não funcionais cuja concepção é adequada, mas apresenta falhas, como o \emph{login} instável, senhas excessivamente complexas e erros recorrentes na troca de senha. De forma específica, nossa solução facilita o agendamento de serviços para as três entidades existentes no \emph{coworking} de beleza: 

\noindent\textbf{Para os Clientes Finais:} A plataforma possibilita o agendamento de serviços de forma intuitiva e flexível. Os clientes poderão escolher profissionais específicos ou optar pelo melhor horário disponível, visualizando facilmente a lista de prestadores, seus serviços, preços, tempo de execução e agendas atualizadas.

\noindent\textbf{Para os Profissionais Autônomos:} O sistema BS Beauty tem como propósito reforçar a autonomia dos profissionais sobre sua agenda e finanças. A aplicação permite bloquear horários, editar preços e a duração dos serviços, além de acompanhar os agendamentos realizados (sejam eles do dia, futuros ou passados) e visualizar relatórios detalhados com a receita gerada pelos serviços prestados.

\noindent\textbf{Para a Gestora:} Nossa parceira, Bruna, terá acesso a funcionalidades exclusivas que incluem análise de métrica de desempenho (a partir de \emph{dashboards)}, gerenciamento do aluguel ou comissão de cada profissional, visualização do fluxo de agendamentos em períodos específicos, envio de mensagens de \emph{marketing} e promoções aos clientes, e acesso a relatórios financeiros detalhados. Ademais, a gestora poderá incluir ou remover profissionais da plataforma conforme a necessidade.

Em síntese, a solução proposta é uma plataforma com \emph{login} simplificado (integrado ao \gls{sso} \footnote{Single Sign-On é um sistema que permite usar um único nome de usuário e senha para acessar vários serviços diferentes, sem precisar criar contas ou lembrar várias senhas.} do Google) e agendamento fácil e transparente para os clientes (incluindo todos os serviços e atributos necessários para uma melhor decisão). Também contará com agenda totalmente controlada pelos profissionais, notificações de agendamento e cancelamento para clientes e profissionais, lista de aniversariantes, desconto por frequência e retenção de dados em conformidade com a \gls{lgpd}. Além disso, a gerente terá acesso à relatórios financeiros e \emph{dashboards} com métricas de produtividade e frequência de clientes.
\section{Justificativa}

Uma pesquisa de 2023 do SEBRAE indica mais de 1,3 milhão de atividades econômicas ligadas a negócios de beleza no Brasil, abrangendo serviços, indústria e comércio, e gerando aproximadamente R\$ 75 bilhões em faturamento anual \cite{sebrae2023forca}. Neste cenário robusto, que movimentou cerca de 27 bilhões de dólares em 2024 \cite{ecommercenapratica2025}, os desafios operacionais crescem cada vez mais: 

\begin{itemize}
	\item Até 30\% do tempo de um pequeno empreendedor é consumido por tarefas administrativas \cite{senac2022};
	\item Taxa média de não comparecimento de clientes atinge 25\% \cite{booksy2022};
	\item Perda de 20\% da receita por não comparecimento \cite{abihpec2021};
	\item Média de 15 horas semanais dedicadas ao controle manual de agenda e finanças \cite{fgv2020};
	\item Insatisfação de 40\% dos clientes devido a falhas de comunicação e alterações de última hora \cite{mindminers2022}.
\end{itemize}

Paralelamente ao crescimento do setor de beleza, o modelo de coworking, originado em ambientes de escritório, expandiu-se para salões, permitindo o compartilhamento de espaços e recursos e a redução de custos \cite{sebrae_coworking,sebraesc2025}. 

Nesse contexto promissor, justifica-se o projeto de extensão \emph{BS Beauty}, destinado a desenvolver uma aplicação web customizada para o gerenciamento de salões em modelo coworking, sob a coordenação de nossa parceira de extensão Bruna. Ao digitalizar e centralizar processos principais, a BS Beauty empodera pequenos empreendedores reduzindo custos operacionais e minimizando erros humanos, melhora a experiência do cliente, eleva a receita dos profissionais por meio do controle preciso de comissões e frequências, e oferece a oportunidade de \emph{insights} estratégicos através de dashboards e relatórios financeiros detalhados. 

Dessa forma, a solução não só supera os problemas de instabilidade e excesso de esforço administrativo, mas também gera valor para todos os envolvidos no salão de beleza. Além disso, como iniciativa de extensão, o projeto permite que os alunos‐desenvolvedores coloquem em prática e melhorem os conhecimentos técnicos e de gestão,  aprendendo com desafios reais de requisitos, usabilidade e performance. Assim, é possível aproximar a graduação das demandas do mercado.



\section{Análise da Concorrência}
\label{sec:analise-concorrencia}
Foi conduzida uma pesquisa de mercado centrada em plataformas brasileiras que combinam agendamento \emph{on-line} e gestão financeira para espaços de beleza no modelo \emph{coworking}. Deste levantamento emergiram três empresas que servirão de referência nesta análise: uma já amplamente consolidada no mercado nacional — embora atue além do universo \emph{coworking} — e outras duas que, apesar de conhecidas, ainda estão em expansão, mas com foco mais relacionado ao da nossa proposta, o que as torna concorrentes que merecem maior atenção estratégica.

\subsection{Trinks}

%inicio de figura
\begin{figure}[htb]
	\centering
	\caption{Logo plataforma Trinks}
	\includegraphics[width=0.5\textwidth]{cap01-Introducao/Images/1.4.1_Trinks}
	\label{fig:Trinks}
	\fonte{\cite{Trinks}}
\end{figure}

 \FloatBarrier

Trinks é uma plataforma já bem consolidada no mercado de gestão de negócios de beleza, com soluções
personalizadas para barbearias, salões de beleza e clínicas de estética. Criada em 2012, é hoje a
plataforma de gestão para beleza com a maior base instalada do país, englobando aproximadamente
2,8\,milhões de usuários e mais de 40\,mil estabelecimentos, sediada no Rio de Janeiro. A plataforma começou como um empreendimento de consultoria em software personalizado, mas logo identificou uma oportunidade no mercado da beleza e mudou de nicho. Em 2024, foi adquirida pelo grupo Stone, o que alavancou ainda mais funcionalidades do aplicativo, como o autoatendimento.
Atualmente, a Trinks oferece software de \emph{back-office} (conjunto de módulos internos que controlam o funcionamento do negócio como finanças, estoque, comissões e relatórios), \emph{marketplace \gls{b2c}} e meios de pagamento próprios (Trinks \emph{Pay}), funcionando praticamente como um “\gls{erp} + \emph{iFood}” para salões e barbearias. Existe um
plano grátis que engloba apenas 150 agendamentos por mês, e os planos pagos variam de R\$ 59 a R\$ 249/mês \cite{Trinks}.

Além dos serviços comuns, seus principais diferenciais são:

\begin{itemize}
	\item \gls{pdv} completo: integração com \gls{tef}, \gls{pix} e split de comissão, atendendo desde \gls{mei}s até redes com exigência de \gls{nfce} e \gls{sat}/\gls{ecf};
	\item Estrutura em nuvem madura, com \gls{sla} de 99,9\,\% e aplicativos nativos para \gls{ios}/Android.
	\item \emph{Marketplace} \textit{Trinks.com}, que gera maior fluxo de clientes, expõe o salão ao
	público final e permite pagamento antecipado;
\end{itemize}

Apesar dos grandes benefícios, identificamos algumas brechas do ponto de vista do negócio da nossa
parceira de extensão, Bruna:

\begin{itemize}
	\item A interface pode ser considerada “poluída” para clientes iniciantes, devido ao grande
	número de funcionalidades;
	\item Há pouco foco no aluguel de estações típico do \emph{coworking}, exigindo ajustes manuais de
	comissão;
	\item Maior parte das funcionalidades estão presentes apenas nos planos superiores.
\end{itemize}

\subsection{Gendo}

%inicio de figura
\begin{figure}[htb]
	\centering
	\caption{Logo plataforma Gendo}
	\includegraphics[width=0.4\textwidth]{cap01-Introducao/Images/1.4.2_Gendo}
	\label{fig:Gendo}
	\fonte{\cite{Gendo}}
\end{figure}

 \FloatBarrier

Lançado em 2017 e sediado em Curitiba-PR, o Gendo se posiciona como um hub\footnote{Hub: plataforma centralizada que integra agenda, \gls{pdv}, finanças e pagamentos em um único ambiente, funcionando como “nó” que organiza os fluxos de dados do negócio.} de gestão 100\,\% em nuvem para negócios além do setor da beleza, como estética, saúde, bem-estar, \emph{pet-shop} e mais recentemente, espaços em formato \emph{coworking}. 
Atualmente mantém mais de 10 mil assinantes, com maior penetração nas regiões Sul e Sudeste do Brasil. Foi criado no modelo \gls{saas} com o intuito de oferecer prontamente agenda \emph{on-line}, automação de lembretes (\emph{e-mail/WhatsApp}), módulo financeiro completo e integrações com \emph{gateways} de pagamento (Stone, Cielo e Mercado Pago). Atualmente, os planos são somente pagos e variam de R\$ 32 a R\$ 293/mês, após 14 dias de teste gratuito \cite{Gendo}.

Seus principais diferenciais são:
\begin{itemize}
	\item Caixa do profissional: Módulo pensado para \emph{coworking}, possibilitando débito automático de aluguel de estação e visualização dos ganhos de cada profissional;
	\item Aplicativo Gendo Pro (\gls{ios} e Android): permite ao profissional ver a agenda, acompanhar comissões, pedir saques e registrar fotos de antes e depois dos serviços;
	\item Relatórios instantâneos: exibem \emph{ticket} médio, previsão de faturamento e dados de cancelamentos, com opção de exportar para \emph{Excel}.
\end{itemize}


Já os maiores pontos de melhoria identificados são:
\begin{itemize}
	\item Dependência de \emph{gateways} externos, o que adiciona custo extra ao \emph{split} \footnote{\emph{Split} é a divisão automática do pagamento entre salão e profissional que, se feita por um \emph{gateway} externo, gera uma taxa extra.};
	\item Relatórios fiscais avançados disponíveis apenas no plano \emph{Premium}.
\end{itemize}

\subsection{Avec}

\begin{figure}[htb]
	\centering
	\caption{Logo plataforma Avec}
	\includegraphics[width=0.4\textwidth]{cap01-Introducao/Images/1.4.3_Avec}
	\label{fig:Avec}
	\fonte{\cite{Avec}}
\end{figure}

 \FloatBarrier

Atualmente, a Avec é a principal concorrente do nosso projeto, pois a entidade parceira que motivou este trabalho utiliza essa plataforma para gerenciar seu salão de beleza em modelo \emph{coworking}. Por esse motivo, ela foi adotada como referência: buscamos manter as funcionalidades que já funcionam bem na Avec e, ao mesmo tempo, acrescentar ou aprimorar recursos que ainda fazem falta para a nossa parceira.

Lançada em 2014 e sediada em São Paulo-SP, a Avec se apresenta como solução``360º'' para salões, barbearias, esmaltarias, spas e estúdios de tatuagem. A plataforma integra software de gestão, um sistema próprio de pagamentos (\emph{Avec Pay}) e um \emph{marketplace \gls{b2c}} que encaminha novos clientes aos estabelecimentos. Segundo a empresa, mais de 40 mil negócios utilizam o serviço no
Brasil. Também desenvolvida no modelo \gls{saas}, a ferramenta oferece agenda \emph{on-line} multiprofissional com confirmações via \emph{WhatsApp} ou \gls{sms}, \gls{pdv} completo com \gls{tef}, \gls{pix} e \emph{split} interno de comissões, além de módulo financeiro integrado. Dispõe ainda de uma carteira digital empregada em pacotes pré-pagos, \emph{gift-cards}, \emph{cashback}, e possui dois aplicativos: o \emph{Avec}, voltado ao cliente final, e o \emph{Avec Pro}, destinado aos profissionais. Há um plano gratuito ``\emph{Avec Go}'' que inclui funções básicas e cobra apenas a taxa transacional, enquanto os planos pagos variam de R\$ 77 a
R\$249 por mês \cite{Avec}.


Com base no \emph{feedback} da nossa entidade parceira, destacam-se três funcionalidades que
a plataforma \emph{Avec} executa bem:
\begin{itemize}
	\item \emph{Split} instantâneo de comissões, dispensando \emph{gateways} externos;
	\item \emph{Marketplace} \gls{b2c} e aplicativo do cliente, que ampliam a visibilidade do salão e aumentam os agendamentos \emph{on-line};
	\item Aplicativo \emph{Avec Pro} (\gls{ios}/Android), no qual o profissional acompanha agenda,
	comissões, saques e registra fotos de “antes e depois” dos serviços.
\end{itemize}

As principais brechas identificadas são:
\begin{itemize}
	\item módulos fiscais avançados (\gls{nfe} e \gls{sat}) disponíveis apenas nos planos superiores;
	\item dependência do hardware e das tarifas do próprio \emph{Avec Pay} para uso pleno do sistema;
	\item custos adicionais para envios em massa de \gls{sms}/\emph{WhatsApp} em campanhas de \emph{marketing};
	\item instabilidade recorrente: o domínio eventualmente fica fora do ar.
\end{itemize}

%seção 1.4 quadro comparativo

\subsection{Quadro comparativo}

\begin{quadro}[htb]
	\caption{\label{frame:comparativo_concorrência}Comparação entre as plataformas concorrentes e a aplicação proposta}
	\footnotesize
	\setlength{\tabcolsep}{4pt}
	\begin{tabular}{|p{6.8cm}|c|c|c|c|}
		\hline
		\textbf{Recurso}                                   & \textbf{Trinks} & \textbf{Gendo} & \textbf{Avec} & \textbf{BS Beauty}\\ \hline 
		Aplicação \textit{web}  & — & \checkmark & \checkmark & \checkmark \\ \hline
		Flexível para \emph{coworking}  & — & \checkmark & \checkmark & \checkmark \\ \hline
		Controle de acesso para gestão  & — & \checkmark  & \checkmark & \checkmark \\ \hline
		Agendamento de serviços 100\% \emph{on-line} & \checkmark & \checkmark & \checkmark & \checkmark \\ \hline
		Controle de conflitos de agenda                    & \checkmark & \checkmark & \checkmark & \checkmark \\ \hline
		Avaliação pós-serviço  & \checkmark & \checkmark  & \checkmark  & \checkmark \\ \hline
		Plataforma do cliente                               & \checkmark & \checkmark & \checkmark & \checkmark \\ \hline
		Plataforma do profissional                                      & \checkmark & \checkmark & \checkmark & \checkmark \\ \hline
		Confirmação automática (\emph{WhatsApp} / \gls{sms} / \emph{e-mail})                    & \checkmark & \checkmark & \checkmark & \checkmark \\ \hline
		Cálculo de \emph{Split} de comissão      & \checkmark & \checkmark & \checkmark & \checkmark  \\ \hline
		Pagamento \emph{on-line}                 & \checkmark & — & \checkmark & \ — \\ \hline
		\emph{\emph{Marketplace} \gls{b2c}}                         & \checkmark & — & \checkmark & \ — \\ \hline
		\emph{Marketing} integrado (envio em massa de \gls{sms}/\emph{WhatsApp/e-mail})  & \checkmark & — & \checkmark & \checkmark \\ \hline
		Programa de indicação  & \checkmark & — & \checkmark & \checkmark \\ \hline
		Relatório financeiro em tempo real    & \checkmark & \checkmark & \checkmark & \checkmark \\ \hline
		\emph{Login} simplificado com integração Google            & \checkmark & \checkmark &  — & \checkmark \\ \hline
		Plano gratuito disponível                                           & \checkmark & — & \checkmark & — \\ \hline
		Lista de aniversariantes para promoções  & \checkmark & \checkmark & — & \checkmark \\ \hline
		
	\end{tabular}
	\fonte{Produzido pelos autores}
\end{quadro}

