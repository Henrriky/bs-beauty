% !TeX root = main.tex

\documentclass[
% -- opções da classe memoir --
12pt,				% tamanho da fonte
openany,			% capítulos iniciam na próxima pág disponível (sem inserir páginas em branco)
oneside,			% para impressão em recto e verso. Oposto a oneside
a4paper,			% tamanho do papel. 
% -- opções da classe abntex2 --
%chapter=TITLE,		% títulos de capítulos convertidos em letras maiúsculas
%section=TITLE,		% títulos de seções convertidos em letras maiúsculas
%subsection=TITLE,	% títulos de subseções convertidos em letras maiúsculas
%subsubsection=TITLE,% títulos de subsubseções convertidos em letras maiúsculas
% -- opções do pacote babel --
english,			% idioma adicional para hifenização
french,				% idioma adicional para hifenização
spanish,			% idioma adicional para hifenização
brazil				% o último idioma é o principal do documento
chapter=TITLE % deixa capitulos em maiusculo
]{abntex2}

% ---
% Pacotes básicos 
% ---
\usepackage{lmodern}			% Usa a fonte Latin Modern			
\usepackage[T1]{fontenc}		% Selecao de codigos de fonte.
\usepackage[utf8]{inputenc}		% Codificacao do documento (conversão automática dos acentos)
\usepackage{indentfirst}		% Indenta o primeiro parágrafo de cada seção.
\usepackage{color}				% Controle das cores
\usepackage{graphicx}			% Inclusão de gráficos
\usepackage{microtype} 			% para melhorias de justificação
% ---

% ---
% Pacotes adicionais, usados apenas no âmbito do Modelo Canônico do abnteX2
% ---
\usepackage{lipsum}				% para geração de dummy text
% ---
\usepackage{amssymb}   %para checkmark
% ---
% Pacotes de citações
% ---
\usepackage[brazilian,hyperpageref]{backref}	 % Paginas com as citações na bibl
\usepackage[alf]{abntex2cite}	% Citações padrão ABNT
% ---
% Pacotes adicionados pelos autores
% ---
\usepackage{enumitem}							% para personalizar listas numeradas
\usepackage{qrcode}								% para gerar QR code
\usepackage{longtable}         					%para tabelas longas
\usepackage[section]{placeins}					% para barreiras

% Pacotes para tabelas
\usepackage{array}								% para formatação de tabelas
\usepackage{xcolor,colortbl}	
\usepackage{tabularx}
\usepackage{multirow}							% para mesclar células
\usepackage{siunitx}							% para alinhamento de números
\usepackage{booktabs} 							% para linhas profissionais em tabelas

% --- 
% CONFIGURAÇÕES DE PACOTES
% ---

\definecolor{cinza-escuro}{RGB}{110, 110, 110}
\definecolor{mygold}{RGB}{255,215,0} % Define um dourado padrão 
\definecolor{myblue}{RGB}{220, 230, 240} % Define um azul pastel acinzentado
\newcommand{\bloco}{\cellcolor{cinza-escuro}\rule{0pt}{2.5ex}}

% Configurações do pacote backref
% Usado sem a opção hyperpageref de backref
\renewcommand{\backrefpagesname}{Citado na(s) página(s):~}
% Texto padrão antes do número das páginas
\renewcommand{\backref}{}
% Define os textos da citação
\renewcommand*{\backrefalt}[4]{
	\ifcase #1 %
	Nenhuma citação no texto.%
	\or
	Citado na página #2.%
	\else
	Citado #1 vezes nas páginas #2.%
	\fi}%

% Configuração do pacote enumitem
% Aplicando o recuo em listas numeradas globalmente
\setlist[enumerate]{leftmargin=2cm}
% ---

% ---
% Informações de dados para CAPA e FOLHA DE ROSTO
% ---
\instituicao{IFSP – Instituto Federal de Educação,\\
	Ciência e Tecnologia Câmpus São Paulo}

\titulo{\emph{BS Beauty Academy}}
\local{São Paulo – SP – Brasil}
\data{2025}
\orientador{Marcelo Tavares de Santana}
\tipotrabalho{Projeto Integrado de Extensão}

% ------------ AUTORES (só aparecem na capa) ---
\newcommand{\autorescapa}{%
	\begin{tabular}[t]{@{}l@{\hspace{3em}}l@{}}
		ALYSON CÉSAR FUMAGALLI & SP3121071\\
		BRUNO DE ALMEIDA FISCHER & 
		SP3120139\\
		ELIEL DA SILVA & 
		SP3121054\\
		GIOVANNA CAMILLE SILVA CARVALHO & SP3123162\\
		HENRIQUE SANTIAGO PIRES & SP312262X\\
		HENRRIKY JHONNY DE OLIVEIRA BASTOS & SP3123103\\
\end{tabular}}

% O preambulo deve conter o tipo do trabalho, o objetivo, 
% o nome da instituição e a área de concentração 
\preambulo{Este projeto integrado de extensão, desenvolvido como parte do curso de Tecnologia em Análise e Desenvolvimento de Sistemas do Instituto Federal de Educação, Ciência e Tecnologia de São Paulo, tem como objetivo desenvolver uma aplicação web para otimizar a gestão e o agendamento de serviços de estética em um ambiente coworking. A área de concentração do projeto é a inovação tecnológica aplicada ao setor de serviços de beleza.}
% ---

% Configurações de aparência do PDF final

% alterando o aspecto da cor azul
\definecolor{blue}{RGB}{41,5,195}

% informações do PDF
\makeatletter
\hypersetup{
	%pagebackref=true,
	pdftitle={\@title}, 
	pdfauthor={\@author},
	pdfsubject={\imprimirpreambulo},
	pdfcreator={LaTeX with abnTeX2},
	pdfkeywords={abnt}{latex}{abntex}{abntex2}{trabalho acadêmico}, 
	colorlinks=true,       		% false: boxed links; true: colored links
	linkcolor=blue,          	% color of internal links
	citecolor=blue,        		% color of links to bibliography
	filecolor=magenta,      		% color of file links
	urlcolor=blue,
	bookmarksdepth=4
}
\makeatother
% --- 

% ---
% Posiciona figuras e tabelas no topo da página quando adicionadas sozinhas
% em um página em branco. Ver https://github.com/abntex/abntex2/issues/170
\makeatletter
\setlength{\@fptop}{5pt} % Set distance from top of page to first float
\makeatother
% ---

% ---
% Possibilita criação de Quadros e Lista de quadros.
% Ver https://github.com/abntex/abntex2/issues/176
%
\newcommand{\quadroname}{Quadro}
\newcommand{\listofquadrosname}{LISTA DE QUADROS}

\newfloat[chapter]{quadro}{loq}{\quadroname}
\newlistof{listofquadros}{loq}{\listofquadrosname}
\newlistentry{quadro}{loq}{0}

% configurações para atender às regras da ABNT
\setfloatadjustment{quadro}{\centering}
\counterwithout{quadro}{chapter}
\renewcommand{\cftquadroname}{\quadroname\space} 
\renewcommand*{\cftquadroaftersnum}{\hfill--\hfill}

\setfloatlocations{quadro}{hbtp} % Ver https://github.com/abntex/abntex2/issues/176
% ---

% --- 
% Espaçamentos entre linhas e parágrafos 
% --- 

% O tamanho do parágrafo é dado por:
\setlength{\parindent}{1.3cm}

% Controle do espaçamento entre um parágrafo e outro:
\setlength{\parskip}{0.2cm}  % tente também \onelineskip

% ---
% compila o indice
% ---
\makeindex
% ---

% — antes de tudo, passe a opção para hyperref evitar destinos duplicados:
\PassOptionsToPackage{hypertexnames=false}{hyperref}
\usepackage[acronym,nomain]{glossaries}
 
% Definição dos acrônimos em ordem alfabética
% -----------------------------------
\newacronym{api}   {API}     {\emph{Application Programming Interface} – Interface de Programação de Aplicações}
\newacronym{aws}   {AWS}     {\emph{Amazon Web Services}}
\newacronym{b2c}   {B2C}     {\emph{Business to Consumer} – Negócio para Consumidor}

\newacronym{cidr}  {CIDR}    {\emph{Classless Inter-Domain Routing} – Roteamento Interdomínio sem Classes}
\newacronym{cli}   {CLI}     {\emph{Command Line Interface} – Interface de Linha de Comando}
\newacronym{covid} {COVID-19}{\emph{Coronavirus Disease 2019} – Doença por Coronavírus 2019}
\newacronym{css}   {CSS}     {\emph{Cascading Style Sheets} }
\newacronym{csv}   {CSV}     {\emph{Comma-Separated Values} – Valores Separados por Vírgula}

\newacronym{der}   {DER}     {Diagrama Entidade-Relacionamento}
\newacronym{dns}   {DNS}     {\emph{Domain Name System} – Sistema de Nomes de Domínio}
\newacronym{dvd}   {DVD}     {\emph{Digital Versatile Disc} – Disco Digital Versátil}

\newacronym{ec2}   {EC2}     {\emph{Elastic Compute Cloud} – Nuvem de Computação Elástica}
\newacronym{ecf}   {ECF}     {Emissor de Cupom Fiscal}
\newacronym{erp}   {ERP}     {\emph{Enterprise Resource Planning} – Sistema Integrado de Gestão Empresarial}

\newacronym{fk}    {FK}      {\emph{Foreign Key}}

\newacronym{html}  {HTML}    {\emph{HyperText Markup Language} - Linguagem de Marcação de Hipertexto}
\newacronym{http}  {HTTP}    {\emph{HyperText Transfer Protocol} – Protocolo de Transferência de Hipertexto}
\newacronym{https} {HTTPS}   {\emph{HyperText Transfer Protocol Secure} – Protocolo de Transferência de Hipertexto Seguro}

\newacronym{ifsp}  {IFSP-SPO}{Instituto Federal de Educação, Ciência e Tecnologia de São Paulo – Campus São Paulo}
\newacronym{ip}    {IP}      {\emph{Internet Protocol} – Protocolo de Internet}
\newacronym{ios}   {iOS}     {\emph{iPhone Operating System} – Sistema Operacional do iPhone}
\newacronym{js}    {JS}      {\emph{JavaScript}}

\newacronym{lgpd}  {LGPD}    {Lei Geral de Proteção de Dados – Lei nº 13.709/2018}

\newacronym{mei}   {MEI}     {Microempreendedor Individual}
\newacronym{mer}   {MER}     {Modelo Entidade-Relacionamento}
\newacronym{mvc}   {MVC}     {\emph{Model-View-Controller} – Modelo-Visão-Controlador}
\newacronym{mvp}   {MVP}	 {\emph{Minimum Viable Product}}

\newacronym{nat}   {NAT}     {\emph{Network Address Translation} – Tradução de Endereços de Rede}
\newacronym{nfe}   {NF-e}    {Nota Fiscal Eletrônica}
\newacronym{nfce}  {NFC-e}   {Nota Fiscal de Consumidor Eletrônica}

\newacronym{oauth} {OAuth}   {\emph{Open Authorization} – Autorização Aberta}

\newacronym{pdf}   {PDF}     {\emph{Portable Document Format} – Formato de Documento Portátil}
\newacronym{pdv}   {PDV}     {Ponto de Venda}
\newacronym{pie}   {PIE}     {Projeto Integrado de Extensão}
\newacronym{pix}   {Pix}     {Pagamento Instantâneo}
\newacronym{pk}    {PK}      {\emph{Primary Key}}
\newacronym{po}    {PO}      {Product Owner}

\newacronym{rest}  {REST}    {\emph{Representational State Transfer} – Transferência de Estado Representacional}
\newacronym{rf}    {RF}      {Requisito Funcional}
\newacronym{rh}    {RH}      {Recursos Humanos}
\newacronym{rn}    {RN}      {Regra de Negócio}
\newacronym{rnf}   {RNF}     {Requisito Não Funcional}

\newacronym{saas}  {SaaS}    {\emph{Software as a Service} – Software como Serviço}
\newacronym{sat}   {SAT}     {Sistema Autenticador e Transmissor de Cupons Fiscais}
\newacronym{sebrae}{SEBRAE}  {Serviço Brasileiro de Apoio às Micro e Pequenas Empresas}
\newacronym{sgbd}  {SGBD}    {Sistema Gerenciador de Banco de Dados}
\newacronym{sla}   {SLA}     {\emph{Service Level Agreement} – Acordo de Nível de Serviço}
\newacronym{sms}   {SMS}     {\emph{Short Message Service} – Serviço de Mensagens Curtas}
\newacronym{spopie}{SPOPIE1} {Projeto Integrado de Extensão I}
\newacronym{ssh}   {SSH}     {\emph{Secure Shell}}
\newacronym{sso}   {SSO}     {\emph{Single Sign-On} – Autenticação Única}

\newacronym{tef}   {TEF}     {Transferência Eletrônica de Fundos}
\newacronym{tls}   {TLS}     {\emph{Transport Layer Security} – Segurança da Camada de Transporte}

\newacronym{vpc}   {VPC}     {\emph{Virtual Private Cloud} – Nuvem Privada Virtual}
\newacronym{vpn}   {VPN}     {\emph{Virtual Private Network} – Rede Privada Virtual}



\makeglossaries
% ----
%  REDEFINIÇÃO DA CAPA (instituição + local/ano centralizados)
% =================================================================
\makeatletter
\renewcommand{\imprimircapa}{%
	\begin{capa}
		\begin{center}
			% --------- Instituição -------------
			{\ABNTEXchapterfont\large\imprimirinstituicao\par}
			\vspace*{2cm}
			% --------- Autores -----------------
			\imprimirautor
			\vfill
			% --------- Título ------------------
			{\ABNTEXchapterfont\bfseries\Large\imprimirtitulo\par}
			\vfill
			% --------- Local e ano -------------
			{\large\imprimirlocal\par}
			{\large\imprimirdata}
		\end{center}
\end{capa}}
\makeatother



% Início do documento
% ----
\begin{document}
	
	% Seleciona o idioma do documento (conforme pacotes do babel)
	%\selectlanguage{english}
	\selectlanguage{brazil}
	
	% Retira espaço extra obsoleto entre as frases.
	\frenchspacing 
	
	% ----------------------------------------------------------
	% ELEMENTOS PRÉ-TEXTUAIS
	% ----------------------------------------------------------
	% \pretextual
	
	% ---
	% Capa
	% ---
	\autor{\autorescapa}
	\imprimircapa
	\autor{} 
	% ---
	
	% ---
	% Folha de rosto
	% (o * indica que haverá a ficha bibliográfica)
	% ---
	
	\imprimirfolhaderosto*
	% ---
	
	% ---
	% Inserir a ficha bibliografica
	% ---
	
	% Isto é um exemplo de Ficha Catalográfica, ou ``Dados internacionais de
	% catalogação-na-publicação''. Você pode utilizar este modelo como referência. 
	% Porém, provavelmente a biblioteca da sua universidade lhe fornecerá um PDF
	% com a ficha catalográfica definitiva após a defesa do trabalho. Quando estiver
	% com o documento, salve-o como PDF no diretório do seu projeto e substitua todo
	% o conteúdo de implementação deste arquivo pelo comando abaixo:
	%
	% \begin{fichacatalografica}
		%     \includepdf{fig_ficha_catalografica.pdf}
		% \end{fichacatalografica}
	
%	\begin{fichacatalografica}
%			\begin{center}
%			\imprimirinstituicao
%		\end{center}
%		\sffamily
%		\vspace*{\fill}					% Posição vertical
%		\begin{center}					% Minipage Centralizado
%			\fbox{\begin{minipage}[c][8cm]{13.5cm}		% Largura
%					\small
%					\imprimirautor
%					%Sobrenome, Nome do autor
%					
%					\hspace{0.5cm} \imprimirtitulo  / \imprimirautor. --
%					\imprimirlocal, \imprimirdata
%					
%					\hspace{0.5cm} \thelastpage p. : il. color; 30cm 
%					
%					\hspace{0.5cm} \imprimirorientadorRotulo~\imprimirorientador\\
%					
%					\hspace{0.5cm}
%					\parbox[t]{\textwidth}{\imprimirtipotrabalho~--~\imprimirinstituicao,
%						\imprimirdata.}\\
%					
%					\hspace{0.5cm}
%					1. Graduação
%					2. Extensão
%					3. Integrado \\
%					I. Marcelo Tavares de Santana.
%					II. IFSP.
%					III. Análise e Desenvolvimento de Sistemas. 
%					IV. SPOPIE1 			
%			\end{minipage}}
%		\end{center}
%	\end{fichacatalografica}
	% ---
	
	% ---
	% Inserir folha de aprovação
	% ---
	
	% Isto é um exemplo de Folha de aprovação, elemento obrigatório da NBR
	% 14724/2011 (seção 4.2.1.3). Você pode utilizar este modelo até a aprovação
	% do trabalho. Após isso, substitua todo o conteúdo deste arquivo por uma
	% imagem da página assinada pela banca com o comando abaixo:
	%
	% \begin{folhadeaprovacao}
		% \includepdf{folhadeaprovacao_final.pdf}
		% \end{folhadeaprovacao}
	%
	\begin{folhadeaprovacao}
			\begin{center}
			\imprimirinstituicao
		\end{center}
		
		\begin{center}
			{\ABNTEXchapterfont\large\imprimirautor}
			
			\vspace*{\fill}\vspace*{\fill}
			\begin{center}
				\ABNTEXchapterfont\bfseries\Large\imprimirtitulo
			\end{center}
			\vspace*{\fill}
			
			\hspace{.45\textwidth}
			\begin{minipage}{.5\textwidth}
				\imprimirpreambulo
			\end{minipage}%
			\vspace*{\fill}
		\end{center}
		
Trabalho aprovado. \imprimirlocal, 
\makebox[2cm]{\hrulefill}~de~\makebox[3cm]{\hrulefill}~de~2025:
		
		\assinatura{\textbf{\imprimirorientador} \\ Orientador 1} 
		\assinatura{\textbf{Johnata Souza Santicioli} \\ Orientador 2}
		\assinatura{\textbf{Daniela dos Santos Santana} \\ Convidado 1}
		\assinatura{\textbf{Leonardo Andrade Motta de Lima} \\ Convidado 2}
		%\assinatura{\textbf{Professor} \\ Convidado 3}
		
		\begin{center}
			\vspace*{0.5cm}
			{\large\imprimirlocal}
			\par
			{\large\imprimirdata}
			\vspace*{1cm}
		\end{center}
		
	\end{folhadeaprovacao}
	% ---
	
	% ---
	% RESUMOS
	% ---
	
	% resumo em português
	\setlength{\absparsep}{18pt} % ajusta o espaçamento dos parágrafos do resumo
	\renewcommand{\resumoname}{RESUMO}
	
	\begin{resumo}
	Este projeto integrado de extensão apresenta o desenvolvimento da aplicação \emph{web} BS \emph{Beauty}, com o objetivo de otimizar a gestão e o agendamento de serviços de beleza no ambiente \emph{coworking}, sob responsabilidade da gestora, nossa parceira de extensão.
	
	Como parte de uma iniciativa competitiva, a solução digital desenvolvida tem como propósito melhorar o atendimento e fidelizar clientes, reduzindo o tempo gasto em tarefas administrativas e garantindo uma interface de usuário intuitiva. Para isso, o sistema centraliza agendas, previne conflitos de horário e fornece notificações automáticas, relatórios financeiros e \emph{dashboards} de desempenho.
	
	A fim de atender as demandas da nossa parceira, a comunicação constante foi essencial para o levantamento de requisitos, análise de concorrentes e definição de regras de negócio. Para a gestão das etapas do projeto, foi adotado o framework ágil \emph{Scrum}, formalizando o planejamento e controle de tarefas no \emph{ProjectLibre}, um software de gestão de projetos que permite gerenciar cronogramas e alocar recursos para as tarefas definidas.
	
	A arquitetura da aplicação foi idealizada em camadas, sendo detalhada em diagramas de componentes e de implantação. Paralelamente, elaborou-se o plano de testes, padronizou-se a documentação e avaliou-se a viabilidade financeira em cenários realistas, otimistas e pessimistas.
	
	Como resultado, a aplicação desenvolvida fortalece conhecimentos teóricos do curso de graduação, aproxima-os das demandas de mercado e promove inovação tecnológica no setor de beleza apoiando o modelo de \emph{coworking}.

\textbf{Palavras-chave:} aplicação \emph{web}. agendamento online. \emph{coworking} de beleza. gestão de serviços. \emph{Scrum}.

	\end{resumo}

\renewcommand{\abstractname}{\MakeUppercase{ABSTRACT}}	
	% resumo em inglês
	\begin{resumo}[abstract]
	\begin{otherlanguage*}{english}
		
		This integrated extension project presents the development of the BS Beauty web application, aimed at optimizing the management and scheduling of beauty services in a coworking environment under the responsibility of the manager, our extension partner.
		
		As part of a competitive initiative, the developed digital solution aims to improve service and foster customer loyalty by reducing the time spent on administrative tasks and ensuring an intuitive user interface. Therefore, the system centralizes schedules, prevents scheduling conflicts and provides automatic notifications, financial reports and performance dashboards.
		
		In order to meet our partner’s demands, constant communication was essential for requirements gathering, competitor analysis and definition of business rules. For project phase management, the agile framework Scrum was adopted, however the planning and task control was formalized in ProjectLibre, a project management software that allows managing schedules and allocating resources to tasks.
		
		The application architecture was designed in layers, detailed in component and deployment diagrams. In parallel, the test plan was developed, documentation was standardized and financial viability was assessed in realistic, optimistic and pessimistic scenarios.
		
		As a result, the developed application strengthens the theoretical knowledge of the undergraduate course, aligns it with market demands and promotes technological innovation in the beauty sector by supporting the coworking model.
		
		\textbf{Keywords:} web application. online scheduling. beauty coworking. service management. Scrum.
		
	\end{otherlanguage*}
\end{resumo}

\pretextual	
	% ---
	% inserir lista de ilustrações
	% ---
	\pdfbookmark[0]{\listfigurename}{lof}
	\renewcommand{\listfigurename}{\MakeUppercase{LISTA DE ILUSTRAÇÕES}}
	\listoffigures*
	\cleardoublepage
	% ---
	
	% ---
	% inserir lista de quadros
	% ---
	\pdfbookmark[0]{\listofquadrosname}{loq}
	\listofquadros*
	\cleardoublepage
	% ---
	
	% ---
	% inserir lista de tabelas
	% ---
	\pdfbookmark[0]{\listtablename}{lot}
\renewcommand{\listtablename}{\MakeUppercase{LISTA DE TABELAS}}
	\listoftables*
	\cleardoublepage
	% ---
	
	% ---
	% inserir lista de abreviaturas e siglas
	
	% Cria os arquivos auxiliares de glossários
\cleardoublepage
\printglossary[
type=\acronymtype,
title={LISTA DE SIGLAS }
]
\cleardoublepage


	
	% ---
	
	% ---
	% inserir lista de símbolos
	% ---
%	\providecommand{\listsimbolosname}{}
%	\renewcommand{\listsimbolosname}{\MakeUppercase{LISTA DE SÍMBOLOS}}

\chapter*{LISTA DE SÍMBOLOS}           % título da página

\begin{description}
	\item[R\$]   Real (moeda brasileira)
	\item[US\$]  Dólar (moeda estadunidense)
	\item[\%]    Porcentagem
	\item[W]     Watt
\end{description}
\cleardoublepage                       % garante começar página nova
	% ---

	% inserir o sumario
	% ---
	\pdfbookmark[0]{\contentsname}{toc}
	\renewcommand{\contentsname}{\MakeUppercase{SUMÁRIO}}
	\tableofcontents*
	\cleardoublepage

	% ----------------------------------------------------------
	\textual
	
% ----------------------------------------------------------
	\cleardoublepage
	\chaptermark{INTRODUÇÃO}
	\chapter{INTRODUÇÃO}		
	%este arquivo contém todas as seções do capítulo de introdução

A ascensão de novos modelos de negócio vem redefinindo o setor da beleza no Brasil e impulsionando a autonomia de seus profissionais. Entre as inovações mais significativas, destaca-se o \emph{coworking} de beleza, que transforma a dinâmica de trabalho ao oferecer infraestrutura compartilhada e flexível. 

A ideia do \emph{coworking} de beleza surgiu após a popularização das salas de escritório compartilhadas, denominadas como espaço \emph{coworking} (do inglês "trabalhando em conjunto"), durante a pandemia da covid-19 em 2020. Assim como no formato original, o maior benefício do \emph{coworking} de beleza é a possibilidade dos profissionais autônomos de dividir os altos custos de um salão próprio. Além disso, esse modelo, em específico, libera cabeleireiros, maquiadores, esteticistas e outros profissionais da beleza de depender de parcerias em estabelecimentos de terceiros ou de atender a domicílio, práticas muito comuns nessa área \cite{BeautyFair, gazeta-coworking}. 

Essa modernização ocorre em um mercado robusto, que movimentou aproximadamente US\$\,27\,bilhões em 2024 e posicionou o país entre os cinco maiores do mundo no ramo, evidenciando a necessidade de adaptação contínua dos empreendedores às novas tendências \cite{Sebrae_2024}. 

Contudo, à medida que esse formato de trabalho se expande, a gestão eficiente de agendas, espaços e custos torna-se um desafio central para maximizar a autonomia e a rentabilidade. A necessidade de evitar conflitos de reserva e falhas de cobrança, de forma ágil e intuitiva, mostra-se cada vez mais evidente.

Este projeto propõe-se, portanto, a desenvolver uma aplicação web para otimizar reservas, uso de espaços e gestão financeira em ambientes de \emph{coworking} de beleza.

\section{Objetivos}

A aplicação \emph{web} BS Beauty foi desenvolvida especialmente para gerenciar um salão de beleza que opera em modelo \emph{coworking}, sob a gestão da nossa parceira de extensão Bruna. Seu objetivo principal é otimizar os processos internos e centralizar o agendamento de serviços, atendendo tanto às demandas administrativas da gestora quanto às necessidades logísticas dos profissionais autônomos, e sugestões dos clientes finais.
\section{Problema e Solução Proposta}

A gestão de um salão por pequenos empreendedores é frequentemente desafiadora. Ademais, demandas surgem e muitas vezes são realizadas manualmente. Portanto, quando alguma etapa falha, evidencia‐se a necessidade de uma solução digital capaz de reduzir erros e diminuir o esforço administrativo.

Por isso, o objetivo geral do projeto é suprir as necessidades de um salão de beleza em modelo \emph{coworking} de forma ágil. Como explicado anteriormente, esse modelo de trabalho é recente (popularizado após a pandemia de \gls{covid} em 2020) e atende diferentes profissionais autônomos (relacionados à gerente por locação ou comissão), não uma equipe com objetivo comum. Desta forma, o problema central é a gerência da ocupação de cada profissional no espaço de trabalho, além do controle das finanças e da agenda dos clientes.

Nossa parceira Bruna já utilizava um sistema digital para gerenciamento do salão. Contudo, apesar dos benefícios trazidos pela solução, o sistema apresentava pontos insatisfatórios, sendo o principal deles a instabilidade da plataforma, que gerava insatisfação e perda de clientes.

Nossa solução consiste em criar uma aplicação \emph{web} que mantenha todas as funcionalidades que já atendem bem a Bruna como o agendamento \emph{on-line} e pesquisa de satisfação. Além disso, a plataforma incluirá funções ainda ausentes e ajustará requisitos funcionais e não funcionais cuja concepção é adequada, mas apresenta falhas, como o \emph{login} instável, senhas excessivamente complexas e erros recorrentes na troca de senha. De forma específica, nossa solução facilita o agendamento de serviços para as três entidades existentes no \emph{coworking} de beleza: 

\noindent\textbf{Para os Clientes Finais:} A plataforma possibilita o agendamento de serviços de forma intuitiva e flexível. Os clientes poderão escolher profissionais específicos ou optar pelo melhor horário disponível, visualizando facilmente a lista de prestadores, seus serviços, preços, tempo de execução e agendas atualizadas.

\noindent\textbf{Para os Profissionais Autônomos:} O sistema BS Beauty tem como propósito reforçar a autonomia dos profissionais sobre sua agenda e finanças. A aplicação permite bloquear horários, editar preços e a duração dos serviços, além de acompanhar os agendamentos realizados (sejam eles do dia, futuros ou passados) e visualizar relatórios detalhados com a receita gerada pelos serviços prestados.

\noindent\textbf{Para a Gestora:} Nossa parceira, Bruna, terá acesso a funcionalidades exclusivas que incluem análise de métrica de desempenho (a partir de \emph{dashboards)}, gerenciamento do aluguel ou comissão de cada profissional, visualização do fluxo de agendamentos em períodos específicos, envio de mensagens de \emph{marketing} e promoções aos clientes, e acesso a relatórios financeiros detalhados. Ademais, a gestora poderá incluir ou remover profissionais da plataforma conforme a necessidade.

Em síntese, a solução proposta é uma plataforma com \emph{login} simplificado (integrado ao \gls{sso} \footnote{Single Sign-On é um sistema que permite usar um único nome de usuário e senha para acessar vários serviços diferentes, sem precisar criar contas ou lembrar várias senhas.} do Google) e agendamento fácil e transparente para os clientes (incluindo todos os serviços e atributos necessários para uma melhor decisão). Também contará com agenda totalmente controlada pelos profissionais, notificações de agendamento e cancelamento para clientes e profissionais, lista de aniversariantes, desconto por frequência e retenção de dados em conformidade com a \gls{lgpd}. Além disso, a gerente terá acesso à relatórios financeiros e \emph{dashboards} com métricas de produtividade e frequência de clientes.
\section{Justificativa}

Uma pesquisa de 2023 do SEBRAE indica mais de 1,3 milhão de atividades econômicas ligadas a negócios de beleza no Brasil, abrangendo serviços, indústria e comércio, e gerando aproximadamente R\$ 75 bilhões em faturamento anual \cite{sebrae2023forca}. Neste cenário robusto, que movimentou cerca de 27 bilhões de dólares em 2024 \cite{ecommercenapratica2025}, os desafios operacionais crescem cada vez mais: 

\begin{itemize}
	\item Até 30\% do tempo de um pequeno empreendedor é consumido por tarefas administrativas \cite{senac2022};
	\item Taxa média de não comparecimento de clientes atinge 25\% \cite{booksy2022};
	\item Perda de 20\% da receita por não comparecimento \cite{abihpec2021};
	\item Média de 15 horas semanais dedicadas ao controle manual de agenda e finanças \cite{fgv2020};
	\item Insatisfação de 40\% dos clientes devido a falhas de comunicação e alterações de última hora \cite{mindminers2022}.
\end{itemize}

Paralelamente ao crescimento do setor de beleza, o modelo de coworking, originado em ambientes de escritório, expandiu-se para salões, permitindo o compartilhamento de espaços e recursos e a redução de custos \cite{sebrae_coworking,sebraesc2025}. 

Nesse contexto promissor, justifica-se o projeto de extensão \emph{BS Beauty}, destinado a desenvolver uma aplicação web customizada para o gerenciamento de salões em modelo coworking, sob a coordenação de nossa parceira de extensão Bruna. Ao digitalizar e centralizar processos principais, a BS Beauty empodera pequenos empreendedores reduzindo custos operacionais e minimizando erros humanos, melhora a experiência do cliente, eleva a receita dos profissionais por meio do controle preciso de comissões e frequências, e oferece a oportunidade de \emph{insights} estratégicos através de dashboards e relatórios financeiros detalhados. 

Dessa forma, a solução não só supera os problemas de instabilidade e excesso de esforço administrativo, mas também gera valor para todos os envolvidos no salão de beleza. Além disso, como iniciativa de extensão, o projeto permite que os alunos‐desenvolvedores coloquem em prática e melhorem os conhecimentos técnicos e de gestão,  aprendendo com desafios reais de requisitos, usabilidade e performance. Assim, é possível aproximar a graduação das demandas do mercado.



\section{Análise da Concorrência}
\label{sec:analise-concorrencia}
Foi conduzida uma pesquisa de mercado centrada em plataformas brasileiras que combinam agendamento \emph{on-line} e gestão financeira para espaços de beleza no modelo \emph{coworking}. Deste levantamento emergiram três empresas que servirão de referência nesta análise: uma já amplamente consolidada no mercado nacional — embora atue além do universo \emph{coworking} — e outras duas que, apesar de conhecidas, ainda estão em expansão, mas com foco mais relacionado ao da nossa proposta, o que as torna concorrentes que merecem maior atenção estratégica.

\subsection{Trinks}

%inicio de figura
\begin{figure}[htb]
	\centering
	\caption{Logo plataforma Trinks}
	\includegraphics[width=0.5\textwidth]{cap01-Introducao/Images/1.4.1_Trinks}
	\label{fig:Trinks}
	\fonte{\cite{Trinks}}
\end{figure}

 \FloatBarrier

Trinks é uma plataforma já bem consolidada no mercado de gestão de negócios de beleza, com soluções
personalizadas para barbearias, salões de beleza e clínicas de estética. Criada em 2012, é hoje a
plataforma de gestão para beleza com a maior base instalada do país, englobando aproximadamente
2,8\,milhões de usuários e mais de 40\,mil estabelecimentos, sediada no Rio de Janeiro. A plataforma começou como um empreendimento de consultoria em software personalizado, mas logo identificou uma oportunidade no mercado da beleza e mudou de nicho. Em 2024, foi adquirida pelo grupo Stone, o que alavancou ainda mais funcionalidades do aplicativo, como o autoatendimento.
Atualmente, a Trinks oferece software de \emph{back-office} (conjunto de módulos internos que controlam o funcionamento do negócio como finanças, estoque, comissões e relatórios), \emph{marketplace \gls{b2c}} e meios de pagamento próprios (Trinks \emph{Pay}), funcionando praticamente como um “\gls{erp} + \emph{iFood}” para salões e barbearias. Existe um
plano grátis que engloba apenas 150 agendamentos por mês, e os planos pagos variam de R\$ 59 a R\$ 249/mês \cite{Trinks}.

Além dos serviços comuns, seus principais diferenciais são:

\begin{itemize}
	\item \gls{pdv} completo: integração com \gls{tef}, \gls{pix} e split de comissão, atendendo desde \gls{mei}s até redes com exigência de \gls{nfce} e \gls{sat}/\gls{ecf};
	\item Estrutura em nuvem madura, com \gls{sla} de 99,9\,\% e aplicativos nativos para \gls{ios}/Android.
	\item \emph{Marketplace} \textit{Trinks.com}, que gera maior fluxo de clientes, expõe o salão ao
	público final e permite pagamento antecipado;
\end{itemize}

Apesar dos grandes benefícios, identificamos algumas brechas do ponto de vista do negócio da nossa
parceira de extensão, Bruna:

\begin{itemize}
	\item A interface pode ser considerada “poluída” para clientes iniciantes, devido ao grande
	número de funcionalidades;
	\item Há pouco foco no aluguel de estações típico do \emph{coworking}, exigindo ajustes manuais de
	comissão;
	\item Maior parte das funcionalidades estão presentes apenas nos planos superiores.
\end{itemize}

\subsection{Gendo}

%inicio de figura
\begin{figure}[htb]
	\centering
	\caption{Logo plataforma Gendo}
	\includegraphics[width=0.4\textwidth]{cap01-Introducao/Images/1.4.2_Gendo}
	\label{fig:Gendo}
	\fonte{\cite{Gendo}}
\end{figure}

 \FloatBarrier

Lançado em 2017 e sediado em Curitiba-PR, o Gendo se posiciona como um hub\footnote{Hub: plataforma centralizada que integra agenda, \gls{pdv}, finanças e pagamentos em um único ambiente, funcionando como “nó” que organiza os fluxos de dados do negócio.} de gestão 100\,\% em nuvem para negócios além do setor da beleza, como estética, saúde, bem-estar, \emph{pet-shop} e mais recentemente, espaços em formato \emph{coworking}. 
Atualmente mantém mais de 10 mil assinantes, com maior penetração nas regiões Sul e Sudeste do Brasil. Foi criado no modelo \gls{saas} com o intuito de oferecer prontamente agenda \emph{on-line}, automação de lembretes (\emph{e-mail/WhatsApp}), módulo financeiro completo e integrações com \emph{gateways} de pagamento (Stone, Cielo e Mercado Pago). Atualmente, os planos são somente pagos e variam de R\$ 32 a R\$ 293/mês, após 14 dias de teste gratuito \cite{Gendo}.

Seus principais diferenciais são:
\begin{itemize}
	\item Caixa do profissional: Módulo pensado para \emph{coworking}, possibilitando débito automático de aluguel de estação e visualização dos ganhos de cada profissional;
	\item Aplicativo Gendo Pro (\gls{ios} e Android): permite ao profissional ver a agenda, acompanhar comissões, pedir saques e registrar fotos de antes e depois dos serviços;
	\item Relatórios instantâneos: exibem \emph{ticket} médio, previsão de faturamento e dados de cancelamentos, com opção de exportar para \emph{Excel}.
\end{itemize}


Já os maiores pontos de melhoria identificados são:
\begin{itemize}
	\item Dependência de \emph{gateways} externos, o que adiciona custo extra ao \emph{split} \footnote{\emph{Split} é a divisão automática do pagamento entre salão e profissional que, se feita por um \emph{gateway} externo, gera uma taxa extra.};
	\item Relatórios fiscais avançados disponíveis apenas no plano \emph{Premium}.
\end{itemize}

\subsection{Avec}

\begin{figure}[htb]
	\centering
	\caption{Logo plataforma Avec}
	\includegraphics[width=0.4\textwidth]{cap01-Introducao/Images/1.4.3_Avec}
	\label{fig:Avec}
	\fonte{\cite{Avec}}
\end{figure}

 \FloatBarrier

Atualmente, a Avec é a principal concorrente do nosso projeto, pois a entidade parceira que motivou este trabalho utiliza essa plataforma para gerenciar seu salão de beleza em modelo \emph{coworking}. Por esse motivo, ela foi adotada como referência: buscamos manter as funcionalidades que já funcionam bem na Avec e, ao mesmo tempo, acrescentar ou aprimorar recursos que ainda fazem falta para a nossa parceira.

Lançada em 2014 e sediada em São Paulo-SP, a Avec se apresenta como solução``360º'' para salões, barbearias, esmaltarias, spas e estúdios de tatuagem. A plataforma integra software de gestão, um sistema próprio de pagamentos (\emph{Avec Pay}) e um \emph{marketplace \gls{b2c}} que encaminha novos clientes aos estabelecimentos. Segundo a empresa, mais de 40 mil negócios utilizam o serviço no
Brasil. Também desenvolvida no modelo \gls{saas}, a ferramenta oferece agenda \emph{on-line} multiprofissional com confirmações via \emph{WhatsApp} ou \gls{sms}, \gls{pdv} completo com \gls{tef}, \gls{pix} e \emph{split} interno de comissões, além de módulo financeiro integrado. Dispõe ainda de uma carteira digital empregada em pacotes pré-pagos, \emph{gift-cards}, \emph{cashback}, e possui dois aplicativos: o \emph{Avec}, voltado ao cliente final, e o \emph{Avec Pro}, destinado aos profissionais. Há um plano gratuito ``\emph{Avec Go}'' que inclui funções básicas e cobra apenas a taxa transacional, enquanto os planos pagos variam de R\$ 77 a
R\$249 por mês \cite{Avec}.


Com base no \emph{feedback} da nossa entidade parceira, destacam-se três funcionalidades que
a plataforma \emph{Avec} executa bem:
\begin{itemize}
	\item \emph{Split} instantâneo de comissões, dispensando \emph{gateways} externos;
	\item \emph{Marketplace} \gls{b2c} e aplicativo do cliente, que ampliam a visibilidade do salão e aumentam os agendamentos \emph{on-line};
	\item Aplicativo \emph{Avec Pro} (\gls{ios}/Android), no qual o profissional acompanha agenda,
	comissões, saques e registra fotos de “antes e depois” dos serviços.
\end{itemize}

As principais brechas identificadas são:
\begin{itemize}
	\item módulos fiscais avançados (\gls{nfe} e \gls{sat}) disponíveis apenas nos planos superiores;
	\item dependência do hardware e das tarifas do próprio \emph{Avec Pay} para uso pleno do sistema;
	\item custos adicionais para envios em massa de \gls{sms}/\emph{WhatsApp} em campanhas de \emph{marketing};
	\item instabilidade recorrente: o domínio eventualmente fica fora do ar.
\end{itemize}

%seção 1.4 quadro comparativo

\subsection{Quadro comparativo}

\begin{quadro}[htb]
	\caption{\label{frame:comparativo_concorrência}Comparação entre as plataformas concorrentes e a aplicação proposta}
	\footnotesize
	\setlength{\tabcolsep}{4pt}
	\begin{tabular}{|p{6.8cm}|c|c|c|c|}
		\hline
		\textbf{Recurso}                                   & \textbf{Trinks} & \textbf{Gendo} & \textbf{Avec} & \textbf{BS Beauty}\\ \hline 
		Aplicação \textit{web}  & — & \checkmark & \checkmark & \checkmark \\ \hline
		Flexível para \emph{coworking}  & — & \checkmark & \checkmark & \checkmark \\ \hline
		Controle de acesso para gestão  & — & \checkmark  & \checkmark & \checkmark \\ \hline
		Agendamento de serviços 100\% \emph{on-line} & \checkmark & \checkmark & \checkmark & \checkmark \\ \hline
		Controle de conflitos de agenda                    & \checkmark & \checkmark & \checkmark & \checkmark \\ \hline
		Avaliação pós-serviço  & \checkmark & \checkmark  & \checkmark  & \checkmark \\ \hline
		Plataforma do cliente                               & \checkmark & \checkmark & \checkmark & \checkmark \\ \hline
		Plataforma do profissional                                      & \checkmark & \checkmark & \checkmark & \checkmark \\ \hline
		Confirmação automática (\emph{WhatsApp} / \gls{sms} / \emph{e-mail})                    & \checkmark & \checkmark & \checkmark & \checkmark \\ \hline
		Cálculo de \emph{Split} de comissão      & \checkmark & \checkmark & \checkmark & \checkmark  \\ \hline
		Pagamento \emph{on-line}                 & \checkmark & — & \checkmark & \ — \\ \hline
		\emph{\emph{Marketplace} \gls{b2c}}                         & \checkmark & — & \checkmark & \ — \\ \hline
		\emph{Marketing} integrado (envio em massa de \gls{sms}/\emph{WhatsApp/e-mail})  & \checkmark & — & \checkmark & \checkmark \\ \hline
		Programa de indicação  & \checkmark & — & \checkmark & \checkmark \\ \hline
		Relatório financeiro em tempo real    & \checkmark & \checkmark & \checkmark & \checkmark \\ \hline
		\emph{Login} simplificado com integração Google            & \checkmark & \checkmark &  — & \checkmark \\ \hline
		Plano gratuito disponível                                           & \checkmark & — & \checkmark & — \\ \hline
		Lista de aniversariantes para promoções  & \checkmark & \checkmark & — & \checkmark \\ \hline
		
	\end{tabular}
	\fonte{Produzido pelos autores}
\end{quadro}



	% ------------------------------------------------------
	\cleardoublepage
	\chaptermark{REVISÃO DA LITERATURA}
	\chapter{REVISÃO DA LITERATURA}
	\input{cap02-revisao/02-00-revisao-literatura.tex}	
	
	%  --------------------------------------------------
	\cleardoublepage
	\chaptermark{GESTÃO DO PROJETO}
	\chapter{GESTÃO DO PROJETO}
	\chapter{GESTÃO DO PROJETO}

Neste capítulo de Gestão do Projeto são abordados tópicos como a organização da equipe (definindo as responsabilidades, papéis e atividades de cada integrante do grupo), a metodologia adotada para a gestão e desenvolvimento do projeto, bem como o repositório da aplicação.

\section{Organização da Equipe}

A equipe do presente projeto é composta por seis docentes do curso de graduação Superior de Tecnologia em Análise e Desenvolvimento de Sistemas do \gls{ifsp}, a saber:

\begin{itemize}
	\item Alyson César Fumagalli dos Santos Júnior
	\item Bruno de Almeida Fischer
	\item Eliel da Silva
	\item Giovanna Camille Silva Carvalho
	\item Henrique Santiago Pires
	\item Henrriky Jhonny de Oliveira Bastos
\end{itemize}

O grupo de trabalho foi formado logo no início da primeira disciplina de \gls{pie} por estudantes que já haviam realizado diversos outros trabalhos em conjunto e, portanto, estavam acostumados a trabalhar em equipe.

Visando uma transmissão de informações clara e centralizada, utilizou-se o \emph{Discord} \cite{discord-2025} como ferramenta de comunicação para realizar reuniões assíncronas; e o \emph{Taiga} \cite{taiga-2025} para organizar documentos, atribuir tarefas e monitorar o andamento do projeto (juntamente com o \emph{ProjectLibre} \cite{projectlibre-2025}).

\subsection{Responsabilidades / Papéis / Atividades}
\label{subsec:papeis-equipe}

Para cada integrante da equipe foi definido um papel contendo responsabilidades e atividades definidas com base em suas respectivas proficiências em diferentes áreas, a fim de distribuir as tarefas do projeto de maneira eficiente.

O Quadro \ref{frame:distribuicao-papeis} descreve a distribuição dos membros do grupo com base em seus respectivos papéis de uma forma mais geral.

\begin{quadro}[ht]
	\setlength{\tabcolsep}{3pt}
	\begin{center}
		\caption{\label{frame:distribuicao-papeis}Membros e seus respectivos papéis}
		\begin{tabular}{|l|c|c|m{3cm}|c|}
			\hline
			\textbf{Membro} & \textbf{Gestor} & \textbf{\textit{Tech Lead}} & \centering\textbf{Desenvolvedor \textit{Full Stack}} & \textbf{Analista de Documentação} \\
			\hline
			Alyson & \checkmark &  & \centering\checkmark &  \\
			\hline
			Bruno &  &  & \centering\checkmark &  \\
			\hline
			Eliel &  &  & \centering\checkmark &  \\
			\hline
			Giovanna &  &  &  & \checkmark \\
			\hline
			Henrique &  &  & \centering\checkmark &  \\
			\hline
			Henrriky &  & \checkmark & \centering\checkmark &  \\
			\hline
		\end{tabular}
		\fonte{Produzido pelos autores}
	\end{center}
\end{quadro}

\indent Assim, constata-se que a equipe conta com 1 (um) gestor, responsável pelo gerenciamento de todo o projeto; 1 (um) \textit{tech lead}, encarregado de guiar a equipe de desenvolvimento; 5 (cinco) desenvolvedores \textit{full stack} (com 2 deles desempenhando outros papéis paralelos) incumbidos por desenvolver o sistema em todas as suas etapas e 1 (uma) analista de documentação para supervisionar as documentações do projeto. 

O Quadro \ref{quad:membros-atividades} apresenta as atividades desempenhadas pelos membros da equipe nas diversas áreas que contemplaram o desenvolvimento do projeto.

\begin{quadro}[h]
	\setlength{\tabcolsep}{2pt}
	\begin{center}
		\caption{\label{quad:membros-atividades}Membros e suas atividades}
		\begin{tabular}{|l|c|m{2.5cm}|m{2cm}|c|m{2cm}|c|}
			\hline
			\textbf{Membro} & \textbf{\emph{Taiga}} & \centering\textbf{\textit{Front-End Back-End}} & \centering\textbf{Banco de Dados} & \textbf{Documentação} & \centering\textbf{Diário de Bordo} & \textbf{\emph{ProjectLibre}}\\
			\hline
			Alyson & \checkmark & \centering\checkmark & \centering\checkmark & \checkmark & \centering\checkmark & \checkmark \\
			\hline
			Bruno & \checkmark & \centering\checkmark & \centering\checkmark & \checkmark & & \checkmark\\
			\hline
			Eliel & \checkmark & \centering\checkmark & \centering\checkmark & \checkmark & & \checkmark\\
			\hline
			Giovanna & \checkmark &  & \centering\checkmark & \checkmark & \centering\checkmark & \checkmark\\
			\hline
			Henrique & \checkmark & \centering\checkmark & \centering\checkmark & \checkmark & & \\
			\hline
			Henrriky & \checkmark & \centering\checkmark & \centering\checkmark & \checkmark & & \\
			\hline
		\end{tabular}
		\fonte{Produzido pelos autores}
	\end{center}
\end{quadro}
\section{Metodologias de Gestão e Desenvolvimento}

Para o projeto, foi adotada a metodologia Scrum \cite{Scrum} tanto voltada para a gestão quanto para o desenvolvimento do sistema.

\subsection{Scrum}

O \textit{framework} Scrum foi escolhido por ter sido bastante estudado em disciplinas anteriores e também devido à equipe já ter uma certa familiaridade em trabalhar com ele. A metodologia foi adaptada de forma a incorporar alguns elementos do Kanban \cite{Kanban} a fim de ter um sistema visual para monitorar as atividades em andamento.

Tendo em mente as responsabilidades provenientes do Scrum, o Quadro \ref{frame:papeis-scrum} descreve como os membros do grupo foram distribuídos seguindo os papéis da metodologia.

\begin{quadro}[ht]
	\setlength{\tabcolsep}{3pt}
	\begin{center}
		\caption{\label{frame:papeis-scrum}Papéis dos integrantes com base no Scrum}
		\begin{tabular}{|l|c|c|c|}
			\hline
			\textbf{Membro} & \textbf{\textit{Product Owner}} & \textbf{\textit{Scrum Master}} & \textbf{Equipe de Desenvolvimento} \\
			\hline
			Alyson &  & \checkmark & \checkmark \\
			\hline
			Bruno &  &  & \checkmark \\
			\hline
			Eliel &  &  & \checkmark \\
			\hline
			Giovanna &  &  & \checkmark \\
			\hline
			Henrique &  &  & \checkmark \\
			\hline
			Henrriky & \checkmark &  & \checkmark \\
			\hline
		\end{tabular}
		\fonte{Produzido pelos autores}
	\end{center}
\end{quadro}

A plataforma Notion \cite{Notion} teve um papel significativo na implementação do Scrum: por meio dela, criou-se um espaço de trabalho para registrar o \textit{product backlog}, os \textit{sprints} e suas respectivas tarefas.

Com essa ferramenta, ainda foi possível atribuir informações --- como prioridade, \textit{status}, prazos e responsáveis --- para cada atividade, além de criar visualizações para simular as colunas e cartões do Kanban, criando uma estrutura organizada e informativa que facilitou o gerenciamento do projeto.

\subsubsection{Sprints}

Texto
\section{Repositório da Aplicação}

Nessa seção do capítulo, apresenta-se o repositório da aplicação, que contém todos os arquivos relevantes ao projeto (como código-fonte, documentos e diagramas).

\subsection{Definição do repositório da aplicação}

Baseado na familiaridade dos integrantes em utilizar o sistema de controle de versão Git \cite{git-2025}, escolheu-se o GitHub \cite{github-2025} para hospedar o repositório remoto da aplicação e tornar o desenvolvimento mais colaborativo.

Visando garantir um repositório organizado e eficiente, estabeleceu-se uma estrutura de diretórios separando código-fonte da documentação do projeto e adotou-se o modelo de fluxo de trabalho \textit{Git Flow} \cite{gitflow-2023} para organizar o versionamento de ramificações. Além disso, foi utilizado \textit{Conventional Commits} \cite{convcommits-2025} para padronizar as mensagens de \textit{commit}, conferindo ainda mais organização.

\subsubsection{Link do repositório e especificações para acesso}

O \emph{QR Code} abaixo (Figura~\ref{fig:qrcode-repositorio}) contém o link que leva diretamente ao repositório remoto do projeto hospedado no \emph{GitHub}. É possível tanto escanear quanto clicar no código abaixo para acessar o repositório.

\begin{figure}[h]
	\centering
		\caption{\emph{QR Code} do repositório da aplicação}
		\label{fig:qrcode-repositorio}
	\scalebox{1.5}{\qrcode{https://github.com/Henrriky/bs-beauty}}
	\fonte{Produzido pelos autores}
\end{figure}

Acessado o repositório, será aberta uma guia no navegador contendo uma página semelhante à Figura \ref{fig:inicio-repositorio}. Todos os arquivos do projeto podem ser visualizados diretamente pelo navegador.

\begin{figure}[h]
	\centering
	\caption{Página inicial do repositório}
	\includegraphics[width=0.8\textwidth]{cap03-gestao/imagens/bsbeauty-repositorio.png}
	\label{fig:inicio-repositorio}
	\fonte{Produzido pelos autores}
\end{figure}

Como o repositório é público, qualquer pessoa pode acessá-lo e cloná-lo localmente usando HTTPS. Para tanto, é necessário seguir as seguintes etapas:
 
\begin{enumerate}
	\item Instalar o Git na máquina (caso não esteja instalado).
	\item Escolher o diretório no qual o repositório será clonado.
	\item Abrir o terminal e alterar o diretório atual para o diretório escolhido.
	\item Usar o comando ``\texttt{git clone https://github.com/Henrriky/bs-beauty}''.
	\item Usar o comando ``\texttt{cd bs-beauty}'' para acessar o repositório clonado.
\end{enumerate}

Assim, o repositório estará devidamente clonado na máquina. Para execução local, utilize a \textit{branch} principal \texttt{main} e consulte o arquivo \texttt{README.MD} do repositório, que contém instruções de instalação de dependências e execução do projeto.

Para utilizar outros métodos de clonagem (como SSH, \emph{GitHub} CLI ou baixar o arquivo \texttt{.zip} do projeto), consulte a documentação oficial do \emph{GitHub} \cite{clone-2025} referente às formas de clonar um repositório.
	
	
	%----------------------------------------------------------
	\stepcounter{chapter}
	%\cleardoublepage
	\chaptermark{DESENVOLVIMENTO DO PROJETO}
	\addtocounter{chapter}{-1}
	\chapter{DESENVOLVIMENTO DO PROJETO}
		\chapter{Desenvolvimento do projeto}
	% ---
	\section{Escopo do Projeto}
	\subsection{Regras de Negócio}
	\subsection{Requisitos Funcionais}
	\subsection{Requisitos Não Funcionais}
	% ---
	\section{Histórias de usuário}
	se aplicável para o scrum
	\subsection{Descrição das Histórias de Usuário}
	% ---

	\section{Arquitetura}

Nessa seção do capítulo, apresenta-se a arquitetura da aplicação, que define como os componentes do sistema interagem entre si e com o usuário.

\subsection{Definições da arquitetura}

O sistema foi estruturado com base no modelo cliente-servidor, no qual o front-end e o back-end operam como aplicações independentes que se comunicam por meio de requisições HTTP, seguindo a arquitetura RESTful.

O back-end é responsável pelo processamento da lógica de negócios e pela persistência dos dados, enquanto o front-end realiza a apresentação e interação com o usuário. Essa separação garante maior modularidade e facilita a manutenção da aplicação, cujo domínio é um ambiente de \textit{coworking} para salões de beleza.

A arquitetura do sistema adota um padrão baseado no Model-View-Controller (MVC) de forma adaptada, aproximando-se de uma arquitetura em camadas. Cada camada possui uma responsabilidade bem definida, conforme descrito a seguir:

\begin{itemize}
  \item \textbf{Controller:} Responsável por receber e tratar as requisições provenientes do cliente (front-end), encaminhando-as para a camada de serviço correspondente. Esta camada lida diretamente com aspectos de infraestrutura externa, como servidores de borda e APIs públicas.
  \item \textbf{Service:} Centraliza a lógica de negócio da aplicação. É responsável por processar as regras do domínio e pode tanto consumir outras funções de serviço quanto interagir com a camada de persistência.
  \item \textbf{Repository:} Trata das operações relacionadas à persistência de dados. Atua como uma interface entre os serviços e os mecanismos de armazenamento, como bancos de dados relacionais ou caches, promovendo o desacoplamento da lógica de negócio em relação à camada de dados.
\end{itemize}

\subsection{Diagrama da arquitetura}

Esta subseção apresenta dois diagramas que representam a arquitetura do sistema desenvolvido: o diagrama de componentes e o diagrama de implantação. Esses diagramas auxiliam na visualização do relacionamento entre as partes da aplicação, bem como a sua distribuição nos ambientes computacionais.

\subsubsection{Diagrama de componentes}


O diagrama de componentes representa a organização lógica dos principais módulos da aplicação, evidenciando as dependências e as formas de comunicação entre eles. Ele demonstra como os componentes interagem por meio de interfaces — que definem contratos de uso — e implementações — que oferecem as funcionalidades esperadas.

Esse tipo de diagrama é útil para visualizar a estrutura modular da aplicação, facilitando o entendimento da separação de responsabilidades e da reutilização de código, além de apoiar decisões relacionadas à manutenção e evolução do sistema.

\begin{figure}[htb]
  \centering
  \includegraphics[width=\textwidth]{cap04-desenvolvimento/images/4-3-2-1-diagrama-componentes}
  \caption{Diagrama de componente da aplicação}
  \label{fig:diagrama-componente}
\end{figure}

\subsubsection{Diagrama de Implantação}

O diagrama de implantação mostra como os componentes do sistema estão distribuídos em termos de infraestrutura, seja em servidores físicos ou ambientes virtuais. Ele ajuda a entender onde cada parte da aplicação está rodando, como os serviços se conectam entre si e quais recursos são necessários para que tudo funcione bem em produção.

Esse tipo de representação é especialmente útil para quem for implantar ou manter o sistema, pois facilita a visualização de elementos como servidores, banco de dados, gateways de rede, e outras dependências da aplicação. Além disso, o diagrama contribui para o planejamento de permissões, acessos e políticas de segurança que precisam ser configuradas na infraestrutura.

\begin{figure}[htb]
  \centering
  \includegraphics[width=\textwidth]{cap04-desenvolvimento/images/4-3-2-2-diagrama-implantacao}
  \caption{Diagrama de componente da aplicação}
  \label{fig:diagrama-implantacao}
\end{figure}

O diagrama\ref{fig:diagrama-implantacao} acima é composto pelos seguintes componentes:
\begin{itemize}
  \item \textbf{\textit{User Device:}} No digrama proposto, por se tratar de uma aplicação web, o dispositivo do usuário será responsável por executar a aplicação \textit{client-side}, que interpreta através do navegador os arquivos CSS, Javascript e HTML gerados no empacotamento ou \textit{build} do projeto em React. 
  \item \textbf{\textit{Amazon Public Instance:}}
  \item \textbf{\textit{Amazon Private Instance:}}
  \item \textbf{\textit{Google Cloud:}}
\end{itemize}


\subsubsection{Diagrama Geral}

Com o objetivo de fornecer uma visão mais aprofundada da infraestrutura da aplicação na nuvem, o Diagrama Geral apresenta a disposição dos principais componentes implantados na arquitetura da Amazon Web Services (AWS). 

Este diagrama ilustra elementos de infraestrutura fundamentais como sub-redes públicas e privadas, resolução de DNS, Virtual Private Cloud (VPC), Bastion Server, NAT Gateway, Internet Gateway, banco de dados, entre outros recursos. A representação facilita a compreensão técnica da topologia de rede e da distribuição dos serviços, evidenciando como a aplicação foi projetada para atender requisitos de segurança, escalabilidade e disponibilidade no ambiente da AWS.

\begin{figure}[htb]
  \centering
  \includegraphics[width=\textwidth]{cap04-desenvolvimento/images/4-3-2-3-diagrama-geral}
  \caption{Diagrama Geral da Arquitetura}
  \label{fig:diagrama-geral}
\end{figure}


	% ---
	\section{Tecnologias}
	\subsection{Front-End}
	\subsection{Back-End}
	\subsection{Banco de dados}
	\subsection{Infraestrutura}
	\subsubsection{Amazon Web Services (AWS)}
	exemplo, colocar todos, abrindo demais itens
	%---
	\section{Testes e Manutenibilidade}
Nessa seção do capítulo, apresenta-se os mecanismos e ferramentas adotados para garantir a qualidade de software do projeto ao longo do desenvolvimento.
Será abordado o plano de testes, assim como cada teste que está incluso. Além disso, assuntos como infraestrutura de testes e convenções de código (coding convention)
serão detalhados, evidenciando práticas que promovem a manutenibilidade, padronização e confiabilidade do sistema ao longo de sua evolução.

\subsection{Plano de Testes}
O plano de testes define a estratégia adotada para garantir a qualidade e confiabilidade da aplicação. 
Ele inclui os tipos de testes aplicados, as ferramentas utilizadas, o escopo das validações, e os critérios de sucesso e falha.

Tendo em vista que a arquitetura do back-end é constituída por \textit{controllers}, \textit{services} e \textit{repositories} usando o framework \textbf{Express}, 
é necessário garantir um ótimo funcionamento da comunicação entre essas camadas. Portanto, conforme os recursos da RESTful API
são desenvolvidos (agendamentos, serviços de beleza, clientes, funcionários, etc) com as respectivas camadas que foram comentadas,
urge-se a demanda de serem testadas em paralelo, cobrindo os cenários possíveis cenários de sucesso/falha. 

Pretende-se atingir, no mínimo, 80\% de cobertura nos testes unitários e integrados aplicados 
sobre as camadas de \textit{controllers} e \textit{services} da API. 
Essa métrica será obtida por meio de ferramentas integradas ao ambiente de testes, como o \textbf{Vitest} com suporte a geração de relatórios de cobertura. Embora a cobertura de testes não garanta por si só a ausência de falhas, ela serve como um forte indicador de que a maior parte da lógica de negócio está sendo exercitada e validada durante a execução dos testes. Essa prática contribui diretamente para a robustez do sistema 
e facilita a detecção precoce de regressões ao longo do desenvolvimento.

O mesmo propósito de cobertura de testes é válido para o front-end desenvolvido em \textbf{React}. Como essa tecnologia adota um princípio de componentização de elementos da interface,
é interessante que as páginas com cenários lógicos mais críticos (como as telas de agendamento) sejam validadas de forma precisa.
Garantindo que os elementos da interface estejam atendendo o comportamento esperado.

Quanto à cobertura de testes do front-end, é tido como propósito, realizar uma cobertura de testes automatizados que envolva todos os processos que foram mapeados no escopo do projeto.

Os arquivos contendo as classes/funções de testes devem estar localizados em diretórios específicos 
de testes, adotando uma nomenclatura compreensível como 
\texttt{analytics.\allowbreak controller.\allowbreak spec.\allowbreak unit.ts} (referenciando um teste unitário) e 
\texttt{analytics.\allowbreak controller.\allowbreak spec.\allowbreak integration.ts} (referenciando um teste integrado).

\subsection{Infraestrutura de Testes}

A infraestrutura de testes do projeto está fortemente integrada ao processo de integração contínua (CI) e entrega contínua (CD),
utilizando a ferramenta \textbf{GitHub Actions}. Essa integração visa garantir que o código entregue atenda a padrões mínimos de qualidade e estabilidade 
em todas as etapas do desenvolvimento.  A cada push ou pull request, fluxos automatizados são executados para validar o código por meio de testes automatizados, 
análise estática, e verificação de cobertura. Esse processo assegura que apenas alterações estáveis e em conformidade com os 
padrões de qualidade sejam incorporadas à base de código principal, promovendo entregas seguras e contínuas ao longo do ciclo de desenvolvimento.

\subsection{Análise Estáticas}
A análise estática de código é realizada utilizando a ferramenta \textbf{SonarQube}, 
que permite detectar problemas de qualidade, como vulnerabilidades, bugs e código duplicado, sem a necessidade de executar a aplicação. 
Essa etapa ajuda a manter o código limpo, seguro e sustentável ao longo do tempo.

\subsection{Testes Automatizados}
A automação de testes tem como objetivo aumentar a confiabilidade do software e permitir validações rápidas e constantes. O projeto conta com:

\begin{itemize}
  \item \textbf{Testes unitários:} Validam o comportamento de funções e componentes isolados. \textbf{Vitest}.
  \item \textbf{Testes integrados:} Verificam a interação entre módulos e componentes da aplicação, também utilizando o \textbf{Vitest}.
%  \item \textbf{Testes de interface:} Serão aplicados, se necessário, nas páginas de maior relevância funcional da aplicação. Ferramentas como \textbf{Cypress} poderão ser utilizadas.
\end{itemize}

\subsection{Logs}
\subsection{Code Convention}
Para garantir a legibilidade e padronização do código, são adotadas convenções definidas com base em boas práticas da comunidade JavaScript/TypeScript.
Essas diretrizes ajudam a manter o código uniforme entre os diferentes desenvolvedores do time, reduzindo ambiguidades e facilitando o entendimento do sistema como um todo.


As principais práticas adotadas incluem:

\begin{itemize}
  \item \textbf{Ferramentas de Linting e Formatação:}
  \begin{itemize}
    \item Utilização do \textbf{ESLint} para garantir padrões de estilo e detectar possíveis erros ou práticas inadequadas de codificação.
    \item Uso do \textbf{Prettier} para formatação automática do código, assegurando que todos os arquivos mantenham a mesma estrutura visual (espaçamento, quebras de linha, indentação, etc).
  \end{itemize}

  \item \textbf{Padrões de Nomenclatura:}
  \begin{itemize}
    \item Uso de \texttt{camelCase} para variáveis e funções.
    \item Uso de \texttt{PascalCase} para componentes e classes.
    \item Uso de \texttt{UPPER\_SNAKE\_CASE} para constantes globais.
  \end{itemize}

  \item \textbf{Organização do Código:}
  \begin{itemize}
    \item Estrutura modular com separação clara entre camadas (controllers, services, repositories).
    \item Agrupamento de arquivos por domínio funcional.
  \end{itemize}

  \item \textbf{Boas Práticas:}
  \begin{itemize}
    \item Escrita de código limpo e legível, evitando duplicações.
    \item Utilização de comentários apenas quando necessário, priorizando nomes autoexplicativos.
    \item Aplicação do princípio DRY (Don't Repeat Yourself).
  \end{itemize}

  \item \textbf{Revisões e Padronização em Equipe:}
  \begin{itemize}
    \item Adoção de pull requests com as devidas descrições das funcionalidades desenvolvidas.
    \item Documentação e comunicação clara de decisões técnicas relevantes.
  \end{itemize}
\end{itemize}
% \subsection{Testes de Performance}
% \subsection{Testes de Componente}
% \subsection{Testes Funcionais}
% \subsection{Testes não Funcionais}
% \subsection{Testes de Carga}
% \subsection{Testes de Configuração}

	%---
	\section{Segurança, Privacidade e Legislação}
	\subsection{Critérios de segurança e privacidade}
	\subsection{Observância à Legislação}
	%---
	\section{Modelo de Banco de Dados}
	\subsection{Modelo Entidade Relacionamento - MER}
	\subsection{DIagrama Entidade Relacionamento - DER}
	tabelas
	\subsection{Dicionário de Dados}
	%---   
	\section{Cronograma}
	pensar no projeto todo, não só MVP
	\subsection{Análise da Duração do Projeto}
	considerar o gerenciamento ágil
	
	% ----------------------------------------------------------
	\cleardoublepage
	\chaptermark{VIABILIDADE FINANCEIRA}
	\chapter{VIABILIDADE FINANCEIRA}
	\chapter{Viabilidade Financeira}

mesmo usando uma hospedagem gratis (AWS), precisamos pesquisar uma paga para colocar na tabela de custos

\section{Custos}

\label{sec:custos}

A Tabela \ref{tab:custo-mensal-projeto} apresenta os custos com mão de obra, infraestrutura e ferramentas aplicadas no desenvolvimento do projeto.

\begin{table}[htbp]
	\centering
	\caption{Custos mensais estimados do projeto}
	\label{tab:custo-mensal-projeto}
	\begin{tabular}{lrr}
		\toprule
		\textbf{Item de Custo} & \textbf{Valor Unitário (R\$)} & \textbf{Valor Total (R\$)} \\
		\midrule
		\multicolumn{3}{l}{\textbf{Mão de Obra}} \\
		\quad Gestor & 1.250,00 & 1.250,00 \\
		\quad Tech Lead & 2.000,00 & 2.000,00 \\
		\quad Desenvolvedor Fullstack (3x) & 750,00 & 2.250,00 \\
		\quad Analista de Documentação & 500,00 & 500,00 \\
		\cmidrule{3-3}
		\multicolumn{2}{l}{\textbf{Subtotal Mão de Obra}} & \textbf{6.000,00} \\
		\midrule
		\multicolumn{3}{l}{\textbf{Infraestrutura e Ferramentas}} \\
		\quad AWS EC2 (múltiplas instâncias) & --- & 250,00 \\
		\quad Internet (banda larga) & 120,00 & 720,00 \\
		\quad Consumo elétrico (computador) & 6,65 & 40,00 \\
		\quad SonarQube (licença comercial) & --- & 350,00 \\
		\quad Notion Plus (equipe) & 55,00 & 330,00 \\
		\quad Figma Professional (acesso full + dev) & --- & 160,00 \\
		\cmidrule{3-3}
		\multicolumn{2}{l}{\textbf{Subtotal Infraestrutura}} & \textbf{1.850,00} \\
		\midrule
		\multicolumn{2}{l}{\textbf{TOTAL MENSAL}} & \textbf{7.850,00} \\
		\bottomrule
	\end{tabular}
	\fonte{Produzido pelos autores}
\end{table}

Para determinar os custos da mão de obra, utilizou-se a plataforma \emph{Glassdoor} \cite{glassdoor-2025}. Nela, pesquisou-se o salário médio de cada um dos cargos definidos na subseção \ref{subsec:papeis-equipe} considerando a cidade de São Paulo.

Conforme aconselhado pelo orientador do projeto, considerou-se um dia de trabalho inteiro equivalente a apenas uma hora para se aproximar da disponibilidade real que os membros da equipe podiam dedicar ao desenvolvimento do projeto. Assim, o custo mensal de cada cargo foi obtido considerando o valor da hora trabalhada.

O preço da infraestrutura das instâncias \gls{ec2}, operando 24 horas por dia, foi calculado utilizando a própria calculadora de preços da \gls{aws} \cite{aws-calculadora-2025}. Já o custo de internet foi baseado numa média de preços online \cite{internet-precos-2025}, enquanto o de consumo elétrico considerou a tarifa residencial da cidade de São Paulo no ano de 2024 \cite{enel-tarifa-2024} para 6 computadores com um consumo médio de 300 W por hora.

Quanto às ferramentas utilizadas no projeto, levou-se em conta os preços do \emph{SonarQube} \cite{sonarqube-preco-2025}, as mensalidades do \emph{Notion} \cite{notion-preco-2025} e os planos do \emph{Figma} \cite{figma-preco-2025}. Os valores em dólar foram convertidos considerando uma cotação média de R\$ 5,50 em junho de 2025.

A Tabela \ref{tab:custo-total-projeto} apresenta o custo total do projeto considerando os 9 meses de desenvolvimento previstos na Seção \ref{sec:duracao}.

\begin{table}[htbp]
	\centering
	\caption{Custos totais do projeto}
	\label{tab:custo-total-projeto}
	\begin{tabular}{lrr}
		\toprule
		\textbf{Item de Custo} & \textbf{Valor Mensal (R\$)} & \textbf{Valor Total (R\$)} \\
		\midrule
		\multicolumn{3}{l}{\textbf{Mão de Obra}} \\
		\quad Gestor & 1.250,00 & 11.250,00 \\
		\quad Tech Lead & 2.000,00 & 18.000,00 \\
		\quad Desenvolvedor Fullstack (3x) & 2.250,00 & 20.250,00 \\
		\quad Analista de Documentação & 500,00 & 4.500,00 \\
		\cmidrule{3-3}
		\multicolumn{2}{l}{\textbf{Subtotal Mão de Obra}} & \textbf{54.000,00} \\
		\midrule
		\multicolumn{3}{l}{\textbf{Infraestrutura e Ferramentas}} \\
		\quad AWS EC2 (múltiplas instâncias) & 250,00 & 2.250,00 \\
		\quad Internet (banda larga) & 720,00 & 6.480,00 \\
		\quad Consumo elétrico (computador) & 40,00 & 360,00 \\
		\quad SonarQube (licença comercial) & 350,00 & 3.150,00 \\
		\quad Notion Plus (equipe) & 330,00 & 2.970,00 \\
		\quad Figma Professional (acesso full + dev) & 160,00 & 1.440,00 \\
		\cmidrule{3-3}
		\multicolumn{2}{l}{\textbf{Subtotal Infraestrutura}} & \textbf{16.650,00} \\
		\midrule
		\multicolumn{2}{l}{\textbf{TOTAL}} & \textbf{70.650,00} \\
		\bottomrule
	\end{tabular}
	\fonte{Produzido pelos autores}
\end{table}
\section{Receitas}

Com base nos custos mensais e totais da Seção \ref{sec:custos}, constata-se que o projeto não é financeiramente viável para a entidade parceira, pois --- mesmo parcelando --- ela teria que arcar com todas as despesas do projeto descritas.

Assim sendo, será considerada uma eventual adaptação do sistema para um modelo \gls{saas} a fim de analisar as receitas que a aplicação geraria. A Tabela \ref{tab:receitas-saas} apresenta os valores das mensalidades dos planos definidos para a aplicação.

\begin{table}[htbp]
	\centering
	\caption{Projeção de receitas mensais - SaaS para salões de beleza}
	\label{tab:receitas-saas}
	\begin{tabular}{lrr}
		\toprule
		\textbf{Plano} & \multicolumn{2}{r}{\textbf{Valor Mensal (R\$)} }\\
		\midrule
		\multicolumn{1}{l}{\textbf{Planos de Assinatura}} & & \\
		\quad Básico & & 59,90 \\
		\quad Profissional & & 99,90 \\
		\bottomrule
	\end{tabular}
	\fonte{Produzido pelos autores}
\end{table}

A precificação dos planos foi feita baseando-se nos preços praticados pelos concorrentes identificados na Seção \ref{sec:analise-concorrencia}, além de considerar os custos operacionais e tentar oferecer um preço atrativo para estimular possíveis clientes a usarem o sistema.
\section{Cenário realista}

Texto
\section{Cenário Otimista}

Texto
\section{Cenário Pessimista}

A Figura \ref{fig:cenario-pessimista} apresenta as receitas e custos acumulados no cenário pessimista considerando o projeto como um \gls{saas}. Nesse contexto, para o cálculo do acúmulo de receitas, estabeleceu-se um aumento mensal baixo de 5 clientes para o plano básico e 3 para o profissional.

Os custos consideram somente os gastos com mão de obra, infraestrutura e mensalidades ou licenças das ferramentas utilizadas no desenvolvimento e manutenção da aplicação.

No cenário pessismista, o ponto de equilíbrio não é atingido no doze meses de análise e, ao fim desse intervalo de tempo, há um prejuízo de R\$ 343 mil.
\begin{figure}[h]
	\centering
	\caption{Cenário pessimista}
	\fbox{\includegraphics[width=0.9\textwidth]{cap05-viabilidade/imagens/cenario-pessimista.png}}
	\label{fig:cenario-pessimista}
	\fonte{Produzido pelos autores}
\end{figure}
	
	% ---------------------------------------------------------------
	\cleardoublepage
	\chaptermark{CONSIDERAÇÕES FINAIS}
	\chapter{CONSIDERAÇÕES FINAIS}
	

A aplicação web \emph{BS Beauty} representa uma contribuição relevante para o mercado de serviços de beleza, sobretudo no modelo de \emph{coworking}. Ao centralizar e digitalizar o agendamento e a gestão do fluxo de trabalho de diferentes profissionais autônomos, a aplicação reduz erros administrativos e melhora a experiência do cliente. Considerando que, segundo estudo do \cite{senac2022}, até 30\% do tempo dos pequenos empreendedores é consumido em tarefas manuais, o \emph{BS Beauty} oferece um diferencial competitivo, permitindo que gestores e profissionais dediquem mais tempo ao atendimento do que às operações administrativas.

Durante o desenvolvimento da aplicação, a escolha da metodologia Scrum foi fundamental para o planejamento e a entrega do projeto. Os ciclos de \emph{sprints} permitiram validar rapidamente cada funcionalidade junto à nossa parceira de extensão, garantindo flexibilidade na evolução de requisitos. Durante discussões sobre os requisitos, o módulo de pagamento \emph{on-line} foi estrategicamente descartado do sistema, uma vez que a gestora optou por manter os pagamentos apenas de forma presencial. Além disso, o fluxo de agendamento foi inicialmente pensado como horário$\to$profissional, porém, após sugestões do orientador, foi dividido em três caminhos distintos: agendamento apenas por horário, apenas por profissional ou de forma combinada. Essas mudanças só foram possíveis graças à flexibilidade do Scrum.

A comunicação com a gestora Bruna e a coordenação interna da equipe, apesar de bem-sucedidas, representaram desafios significativos. A necessidade de validações constantes das regras de negócio, aliada a conflitos de agenda, impossibilitou reuniões presenciais com todos os \emph{stakeholders}. Por isso, grande parte das interações foi conduzida por mensagens de texto ou ligações. Ferramentas como \emph{Discord}, \emph{GitHub} e \emph{Taiga} foram essenciais para alinhar demandas, formalizar decisões e manter a coesão no código e na documentação.

Espera-se que, após a implantação, o sistema elimine conflitos de agenda, reduza a taxa de não-comparecimento por meio de lembretes automáticos e aumente a receita mensal em virtude da otimização da ocupação das estações, da clareza nos relatórios e da fidelização de clientes. Além disso, ao disponibilizar \emph{dashboards} de performance e indicadores de satisfação, o \emph{BS Beauty} criará uma base de dados estratégica para decisões futuras de \emph{marketing} e expansão.

Olhando para o futuro, o encerramento deste ciclo inicial abre caminho para novas fases de evolução do projeto. Os próximos passos para o \emph{BS Beauty} podem incluir o desenvolvimento de um aplicativo móvel dedicado aos profissionais, para consulta ágil de agenda e comissões, e a implementação de um módulo de análise de dados avançado. Este módulo poderá, futuramente, utilizar o histórico de agendamentos para gerar insights preditivos sobre horários de pico e preferências de clientes, consolidando a plataforma como uma ferramenta não apenas operacional, mas também estratégica.

Em síntese, este projeto de extensão cumpriu seu objetivo ao entregar uma solução digital que se destaca da concorrência por meio de diferenciais claros: uma interface minimalista focada no \emph{coworking}, um sistema robusto para a gestão de aluguel de estações e um modelo de negócio transparente, com todas as funcionalidades operacionais necessárias inclusas no plano básico e sem taxas de pagamento. Por fim, o \emph{BS Beauty} não é apenas um sistema de agendamento, mas uma ferramenta de empoderamento para pequenos empreendedores, projetada para apoiar o crescimento sustentável dos \emph{coworkings} de beleza no Brasil.

	% ----------------------------------------------------------

	
	%Página de referencias
	\renewcommand{\bibname}{REFERÊNCIAS} %renomeia o título da página de referencias
	\bibliography{referencias-bibliograficas}
	\bibliographystyle{abntex2-alf}
	\bibliographystyle{abntex2-num}
	
	
	% ----------------------------------------------------------
	% Glossário
	% ----------------------------------------------------------
	%
	% Consulte o manual da classe abntex2 para orientações sobre o glossário.
	%
	%\glossary
	
	% ----------------------------------------------------------
	% Apêndices
	% ----------------------------------------------------------
	
	% ---
	% Inicia os apêndices
	% ---
	\begin{apendicesenv}
		
		% Imprime uma página indicando o início dos apêndices
		\partapendices
		
		\chapter{Cobertura de Testes Unitários}
\label{ap:teste-unitario}

Este apêndice apresenta um recorte dos relatórios de cobertura de testes unitários
obtidos durante o desenvolvimento do sistema, bem como capturas de tela geradas
automaticamente pelo \textit{framework} \textbf{Vitest}. Os relatórios aqui
apresentados referem-se à versão estável utilizada para a elaboração deste documento.

\section*{Resumo da cobertura}

De forma geral, o \textit{back-end} alcançou os seguintes índices de cobertura:

\begin{itemize}
	\item \textbf{Cobertura global do diretório \texttt{src/}:}
	cerca de \textbf{96{,}71\%} de instruções/linhas,
	\textbf{98{,}55\%} de funções e
	\textbf{92{,}32\%} de ramos.
	\item \textbf{Módulo \texttt{src/controllers}:}
	aproximadamente \textbf{91{,}52\%} de instruções/linhas,
	\textbf{95{,}6\%} de funções e
	\textbf{94{,}54\%} de ramos.
	\item \textbf{Módulo \texttt{src/controllers/auth}:}
	aproximadamente \textbf{97{,}36\%} de instruções/linhas,
	\textbf{100\%} de funções e
	\textbf{96{,}15\%} de ramos.
	\item \textbf{Módulo \texttt{src/services}:}
	aproximadamente \textbf{98{,}21\%} de instruções/linhas,
	\textbf{99{,}25\%} de funções e
	\textbf{92{,}68\%} de ramos.
	\item \textbf{Módulo \texttt{src/services/use-cases/auth}:}
	\textbf{100\%} de instruções/linhas e funções, com
	cerca de \textbf{97{,}65\%} de ramos.
\end{itemize}

Esses valores indicam que os principais fluxos de negócio estão amplamente exercitados pelos testes unitários.

\section*{Capturas dos relatórios de cobertura}

A seguir, são apresentadas capturas da interface de relatório de cobertura, com foco
na visão geral do projeto e nos módulos mais relevantes.

\begin{figure}[htbp]
	\centering
	\includegraphics[width=\textwidth]{apendices/images/cobertura-geral.png}
	\caption{Visão geral da cobertura de testes unitários do diretório \texttt{src/}.}
	\label{fig:coverage-src-geral}
	\fonte{Produzido pelos autores a partir do relatório de cobertura do Vitest.}
\end{figure}

\begin{figure}[htbp]
	\centering
	\includegraphics[width=\textwidth]{apendices/images/cobertura-controllers.png}
	\caption{Cobertura de testes unitários das \textit{controllers} do back-end.}
	\label{fig:coverage-controllers}
	\fonte{Produzido pelos autores a partir do relatório de cobertura do Vitest.}
\end{figure}

\begin{figure}[htbp]
	\centering
	\includegraphics[width=\textwidth]{apendices/images/cobertura-auth-controllers.png}
	\caption{Cobertura de testes unitários das \textit{controllers} de autenticação.}
	\label{fig:coverage-controllers-auth}
	\fonte{Produzido pelos autores a partir do relatório de cobertura do Vitest.}
\end{figure}

\begin{figure}[htbp]
	\centering
	\includegraphics[width=\textwidth]{apendices/images/cobertura-services.png}
	\caption{Cobertura de testes unitários dos \textit{services} (casos de uso) do back-end.}
	\label{fig:coverage-services}
	\fonte{Produzido pelos autores a partir do relatório de cobertura do Vitest.}
\end{figure}

\begin{figure}[htbp]
	\centering
	\includegraphics[width=\textwidth]{apendices/images/cobertura-auth-services.png}
	\caption{Cobertura de testes unitários dos casos de uso de autenticação.}
	\label{fig:coverage-services-auth}
	\fonte{Produzido pelos autores a partir do relatório de cobertura do Vitest.}
\end{figure}

	
		
	\end{apendicesenv}
	
	% ----------------------------------------------------------
	% Anexos
	% Inicia os anexos
	
%	\begin{anexosenv}
		
		% Imprime uma página indicando o início dos anexos
	%	\partanexos
		
		% ---
%		\chapter{exemplo 1}
	%	matérias de outras fontes que não o próprio autor do trabalho
		
%	\end{anexosenv}
	
	%---------------------------------------------------------------------
	% INDICE REMISSIVO
	%---------------------------------------------------------------------
	\phantompart
	\printindex
	%---------------------------------------------------------------------
	
\end{document}
