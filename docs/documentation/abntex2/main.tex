\documentclass[
% -- opções da classe memoir --
12pt,				% tamanho da fonte
openany,			% capítulos iniciam na próxima pág disponível (sem inserir páginas em branco)
twoside,			% para impressão em recto e verso. Oposto a oneside
a4paper,			% tamanho do papel. 
% -- opções da classe abntex2 --
%chapter=TITLE,		% títulos de capítulos convertidos em letras maiúsculas
%section=TITLE,		% títulos de seções convertidos em letras maiúsculas
%subsection=TITLE,	% títulos de subseções convertidos em letras maiúsculas
%subsubsection=TITLE,% títulos de subsubseções convertidos em letras maiúsculas
% -- opções do pacote babel --
english,			% idioma adicional para hifenização
french,				% idioma adicional para hifenização
spanish,			% idioma adicional para hifenização
brazil				% o último idioma é o principal do documento
]{abntex2}

% ---
% Pacotes básicos 
% ---
\usepackage{lmodern}			% Usa a fonte Latin Modern			
\usepackage[T1]{fontenc}		% Selecao de codigos de fonte.
\usepackage[utf8]{inputenc}		% Codificacao do documento (conversão automática dos acentos)
\usepackage{indentfirst}		% Indenta o primeiro parágrafo de cada seção.
\usepackage{color}				% Controle das cores
\usepackage{graphicx}			% Inclusão de gráficos
\usepackage{microtype} 			% para melhorias de justificação
% ---

% ---
% Pacotes adicionais, usados apenas no âmbito do Modelo Canônico do abnteX2
% ---
\usepackage{lipsum}				% para geração de dummy text
% ---
\usepackage{amssymb}   %para checkmark
% ---
% Pacotes de citações
% ---
\usepackage[brazilian,hyperpageref]{backref}	 % Paginas com as citações na bibl
\usepackage[alf]{abntex2cite}	% Citações padrão ABNT

% --- 
% CONFIGURAÇÕES DE PACOTES
% ---

% Configurações do pacote backref
% Usado sem a opção hyperpageref de backref
\renewcommand{\backrefpagesname}{Citado na(s) página(s):~}
% Texto padrão antes do número das páginas
\renewcommand{\backref}{}
% Define os textos da citação
\renewcommand*{\backrefalt}[4]{
	\ifcase #1 %
	Nenhuma citação no texto.%
	\or
	Citado na página #2.%
	\else
	Citado #1 vezes nas páginas #2.%
	\fi}%
% ---

% ---
% Informações de dados para CAPA e FOLHA DE ROSTO
% ---
\titulo{Projeto Integrado de Extensão I}
\autor{BS Beauty Academy}
\local{São Paulo, SP - Brasil}
\data{2025}
\orientador{Marcelo Tavares}
\instituicao{%
	Instituto Federal de Educação, Ciência e\\ Tecnologia de São Paulo -- IFSP
	\par
	Tecnologia em Análise e Desenvolvimento de Sistemas
	\par
	Programa de graduação}
\tipotrabalho{Projeto Integrado de Extensão}
% O preambulo deve conter o tipo do trabalho, o objetivo, 
% o nome da instituição e a área de concentração 
\preambulo{Este projeto integrado de extensão, desenvolvido como parte do curso de Tecnologia em Análise e Desenvolvimento de Sistemas do Instituto Federal de Educação, Ciência e Tecnologia de São Paulo, tem como objetivo desenvolver uma aplicação web para otimizar a gestão e o agendamento de serviços de estética em um ambiente coworking. A área de concentração do projeto é a inovação tecnológica aplicada ao setor de serviços de beleza.}
% ---

% Configurações de aparência do PDF final

% alterando o aspecto da cor azul
\definecolor{blue}{RGB}{41,5,195}

% informações do PDF
\makeatletter
\hypersetup{
	%pagebackref=true,
	pdftitle={\@title}, 
	pdfauthor={\@author},
	pdfsubject={\imprimirpreambulo},
	pdfcreator={LaTeX with abnTeX2},
	pdfkeywords={abnt}{latex}{abntex}{abntex2}{trabalho acadêmico}, 
	colorlinks=true,       		% false: boxed links; true: colored links
	linkcolor=blue,          	% color of internal links
	citecolor=blue,        		% color of links to bibliography
	filecolor=magenta,      		% color of file links
	urlcolor=blue,
	bookmarksdepth=4
}
\makeatother
% --- 

% ---
% Posiciona figuras e tabelas no topo da página quando adicionadas sozinhas
% em um página em branco. Ver https://github.com/abntex/abntex2/issues/170
\makeatletter
\setlength{\@fptop}{5pt} % Set distance from top of page to first float
\makeatother
% ---

% ---
% Possibilita criação de Quadros e Lista de quadros.
% Ver https://github.com/abntex/abntex2/issues/176
%
\newcommand{\quadroname}{Quadro}
\newcommand{\listofquadrosname}{Lista de quadros}

\newfloat[chapter]{quadro}{loq}{\quadroname}
\newlistof{listofquadros}{loq}{\listofquadrosname}
\newlistentry{quadro}{loq}{0}

% configurações para atender às regras da ABNT
\setfloatadjustment{quadro}{\centering}
\counterwithout{quadro}{chapter}
\renewcommand{\cftquadroname}{\quadroname\space} 
\renewcommand*{\cftquadroaftersnum}{\hfill--\hfill}

\setfloatlocations{quadro}{hbtp} % Ver https://github.com/abntex/abntex2/issues/176
% ---

% --- 
% Espaçamentos entre linhas e parágrafos 
% --- 

% O tamanho do parágrafo é dado por:
\setlength{\parindent}{1.3cm}

% Controle do espaçamento entre um parágrafo e outro:
\setlength{\parskip}{0.2cm}  % tente também \onelineskip

% ---
% compila o indice
% ---
\makeindex
% ---

% ----
% Início do documento
% ----
\begin{document}
	
	% Seleciona o idioma do documento (conforme pacotes do babel)
	%\selectlanguage{english}
	\selectlanguage{brazil}
	
	% Retira espaço extra obsoleto entre as frases.
	\frenchspacing 
	
	% ----------------------------------------------------------
	% ELEMENTOS PRÉ-TEXTUAIS
	% ----------------------------------------------------------
	% \pretextual
	
	% ---
	% Capa
	% ---
	\imprimircapa
	% ---
	
	% ---
	% Folha de rosto
	% (o * indica que haverá a ficha bibliográfica)
	% ---
	\imprimirfolhaderosto*
	% ---
	
	% ---
	% Inserir a ficha bibliografica
	% ---
	
	% Isto é um exemplo de Ficha Catalográfica, ou ``Dados internacionais de
	% catalogação-na-publicação''. Você pode utilizar este modelo como referência. 
	% Porém, provavelmente a biblioteca da sua universidade lhe fornecerá um PDF
	% com a ficha catalográfica definitiva após a defesa do trabalho. Quando estiver
	% com o documento, salve-o como PDF no diretório do seu projeto e substitua todo
	% o conteúdo de implementação deste arquivo pelo comando abaixo:
	%
	% \begin{fichacatalografica}
		%     \includepdf{fig_ficha_catalografica.pdf}
		% \end{fichacatalografica}
	
	\begin{fichacatalografica}
		\sffamily
		\vspace*{\fill}					% Posição vertical
		\begin{center}					% Minipage Centralizado
			\fbox{\begin{minipage}[c][8cm]{13.5cm}		% Largura
					\small
					\imprimirautor
					%Sobrenome, Nome do autor
					
					\hspace{0.5cm} \imprimirtitulo  / \imprimirautor. --
					\imprimirlocal, \imprimirdata
					
					\hspace{0.5cm} \thelastpage p. : il. color; 30cm 
					
					\hspace{0.5cm} \imprimirorientadorRotulo~\imprimirorientador\\
					
					\hspace{0.5cm}
					\parbox[t]{\textwidth}{\imprimirtipotrabalho~--~\imprimirinstituicao,
						\imprimirdata.}\\
					
					\hspace{0.5cm}
					1. Graduação
					2. Extensão
					3. Integrado \\
					I. Marcelo Tavares.
					II. IFSP.
					III. Análise e Desenvolvimento de Sistemas. 
					IV. PIE1 			
			\end{minipage}}
		\end{center}
	\end{fichacatalografica}
	% ---
	
	% ---
	% Inserir errata
	% ---
	\begin{errata}
		pagina opcional para descrever algum correção que deve ser feita no documento.
		
		\begin{table}[htb]
			\center
			\footnotesize
			\begin{tabular}{|p{1.4cm}|p{1cm}|p{3cm}|p{3cm}|}
				\hline
				\textbf{Folha} & \textbf{Linha}  & \textbf{Onde se lê}  & \textbf{Leia-se}  \\
				\hline
				1 & 10 & auto-conclavo & autoconclavo\\
				\hline
			\end{tabular}
		\end{table}
		
	\end{errata}
	% ---
	
	% ---
	% Inserir folha de aprovação
	% ---
	
	% Isto é um exemplo de Folha de aprovação, elemento obrigatório da NBR
	% 14724/2011 (seção 4.2.1.3). Você pode utilizar este modelo até a aprovação
	% do trabalho. Após isso, substitua todo o conteúdo deste arquivo por uma
	% imagem da página assinada pela banca com o comando abaixo:
	%
	% \begin{folhadeaprovacao}
		% \includepdf{folhadeaprovacao_final.pdf}
		% \end{folhadeaprovacao}
	%
	\begin{folhadeaprovacao}
		
		\begin{center}
			{\ABNTEXchapterfont\large\imprimirautor}
			
			\vspace*{\fill}\vspace*{\fill}
			\begin{center}
				\ABNTEXchapterfont\bfseries\Large\imprimirtitulo
			\end{center}
			\vspace*{\fill}
			
			\hspace{.45\textwidth}
			\begin{minipage}{.5\textwidth}
				\imprimirpreambulo
			\end{minipage}%
			\vspace*{\fill}
		\end{center}
		
		Trabalho aprovado. \imprimirlocal, xx de xxx de 2025:
		
		\assinatura{\textbf{\imprimirorientador} \\ Orientador} 
		\assinatura{\textbf{Professor} \\ Convidado 1}
		\assinatura{\textbf{Professor} \\ Convidado 2}
		%\assinatura{\textbf{Professor} \\ Convidado 3}
		%\assinatura{\textbf{Professor} \\ Convidado 4}
		
		\begin{center}
			\vspace*{0.5cm}
			{\large\imprimirlocal}
			\par
			{\large\imprimirdata}
			\vspace*{1cm}
		\end{center}
		
	\end{folhadeaprovacao}
	% ---
	
	% ---
	% Dedicatória
	% ---
	\begin{dedicatoria}
		\vspace*{\fill}
		\centering
		\noindent
		\textit{ Este trabalho é dedicado às crianças adultas que,\\
			quando pequenas, sonharam em se tornar cientistas.} \vspace*{\fill}
	\end{dedicatoria}
	% ---
	
	% ---
	% Agradecimentos
	% ---
	\begin{agradecimentos}
	escrever agradecimentos (um paragrafo para cada?)
		
	\end{agradecimentos}
	% ---
	
	% ---
	% Epígrafe
	% ---
	\begin{epigrafe}
		\vspace*{\fill}
		\begin{flushright}
			\textit{``Não vos amoldeis às estruturas deste mundo, \\
				mas transformai-vos pela renovação da mente, \\
				a fim de distinguir qual é a vontade de Deus: \\
				o que é bom, o que Lhe é agradável, o que é perfeito.\\
				(Bíblia Sagrada, Romanos 12, 2)}
		\end{flushright}
	\end{epigrafe}
	% ---
	
	% ---
	% RESUMOS
	% ---
	
	% resumo em português
	\setlength{\absparsep}{18pt} % ajusta o espaçamento dos parágrafos do resumo
	\begin{resumo}
		o resumo deve ressaltar o
		objetivo, o método, os resultados e as conclusões do documento. A ordem e a extensão
		destes itens dependem do tipo de resumo (informativo ou indicativo) e do
		tratamento que cada item recebe no documento original. O resumo deve ser
		precedido da referência do documento, com exceção do resumo inserido no
		próprio documento. As palavras-chave devem figurar logo abaixo do
		resumo, antecedidas da expressão Palavras-chave:, separadas entre si por
		ponto e finalizadas também por ponto.
		
		\textbf{Palavras-chave}: latex. abntex. editoração de texto.
	\end{resumo}
	
	% resumo em inglês
	\begin{resumo}[abstract]
		\begin{otherlanguage*}{english}
			This is the english abstract.
			
			\vspace{\onelineskip}
			
			\noindent 
			\textbf{Keywords}: latex. abntex. text editoration.
		\end{otherlanguage*}
	\end{resumo}
	
	% ---
	% inserir lista de ilustrações
	% ---
	\pdfbookmark[0]{\listfigurename}{lof}
	\listoffigures*
	\cleardoublepage
	% ---
	
	% ---
	% inserir lista de quadros
	% ---
	\pdfbookmark[0]{\listofquadrosname}{loq}
	\listofquadros*
	\cleardoublepage
	% ---
	
	% ---
	% inserir lista de tabelas
	% ---
	\pdfbookmark[0]{\listtablename}{lot}
	\listoftables*
	\cleardoublepage
	% ---
	
	% ---
	% inserir lista de abreviaturas e siglas
	% ---
	\begin{siglas}
		\item [PIE1] Projeto Integrado de Extensão I
		\item [B2C] Business to Consumer
		\item [ECF] Emissor de Cupom Fiscal			\item [ERP] Enterprise Resource Planning (Sistema Integrado de Gestão Empresarial)
		\item [iOS] iPhone Operating System
		\item [MEI] Microempreendedor Individual
		\item [NFC- e] Nota Fiscal de Consumidor Eletrônica
		\item [PDV] Ponto de Venda
		\item [Pix] Pagamento Instantâneo
		\item [SAT] Sistema Autenticador e Transmissor de Cupons Fiscais
		\item [SLA] Service Level Agreement (Acordo de Nível de Serviço)
		\item [SMS] Short Message Service
		\item [TEF] Transferência Eletrônica de Fundos
	\end{siglas}
	% ---
	
	% ---
	% inserir lista de símbolos
	% ---
	\begin{simbolos}
			\item [R\$] Real (moeda brasileira)
			\item [\%]  Porcentagem
		
	\end{simbolos}
	% ---

	% inserir o sumario
	% ---
	\pdfbookmark[0]{\contentsname}{toc}
	\tableofcontents*
	\cleardoublepage

	% ----------------------------------------------------------
	\textual
	
% ----------------------------------------------------------
	\chapter{Introdução}
	
% o input do capitulo 1 já contem todas as seções dele	
%este arquivo contém todas as seções do capítulo de introdução

A ascensão de novos modelos de negócio vem redefinindo o setor da beleza no Brasil e impulsionando a autonomia de seus profissionais. Entre as inovações mais significativas, destaca-se o \emph{coworking} de beleza, que transforma a dinâmica de trabalho ao oferecer infraestrutura compartilhada e flexível. 

A ideia do \emph{coworking} de beleza surgiu após a popularização das salas de escritório compartilhadas, denominadas como espaço \emph{coworking} (do inglês "trabalhando em conjunto"), durante a pandemia da covid-19 em 2020. Assim como no formato original, o maior benefício do \emph{coworking} de beleza é a possibilidade dos profissionais autônomos de dividir os altos custos de um salão próprio. Além disso, esse modelo, em específico, libera cabeleireiros, maquiadores, esteticistas e outros profissionais da beleza de depender de parcerias em estabelecimentos de terceiros ou de atender a domicílio, práticas muito comuns nessa área \cite{BeautyFair, gazeta-coworking}. 

Essa modernização ocorre em um mercado robusto, que movimentou aproximadamente US\$\,27\,bilhões em 2024 e posicionou o país entre os cinco maiores do mundo no ramo, evidenciando a necessidade de adaptação contínua dos empreendedores às novas tendências \cite{Sebrae_2024}. 

Contudo, à medida que esse formato de trabalho se expande, a gestão eficiente de agendas, espaços e custos torna-se um desafio central para maximizar a autonomia e a rentabilidade. A necessidade de evitar conflitos de reserva e falhas de cobrança, de forma ágil e intuitiva, mostra-se cada vez mais evidente.

Este projeto propõe-se, portanto, a desenvolver uma aplicação web para otimizar reservas, uso de espaços e gestão financeira em ambientes de \emph{coworking} de beleza.

\section{Objetivos}

A aplicação \emph{web} BS Beauty foi desenvolvida especialmente para gerenciar um salão de beleza que opera em modelo \emph{coworking}, sob a gestão da nossa parceira de extensão Bruna. Seu objetivo principal é otimizar os processos internos e centralizar o agendamento de serviços, atendendo tanto às demandas administrativas da gestora quanto às necessidades logísticas dos profissionais autônomos, e sugestões dos clientes finais.
\section{Problema e Solução Proposta}

A gestão de um salão por pequenos empreendedores é frequentemente desafiadora. Ademais, demandas surgem e muitas vezes são realizadas manualmente. Portanto, quando alguma etapa falha, evidencia‐se a necessidade de uma solução digital capaz de reduzir erros e diminuir o esforço administrativo.

Por isso, o objetivo geral do projeto é suprir as necessidades de um salão de beleza em modelo \emph{coworking} de forma ágil. Como explicado anteriormente, esse modelo de trabalho é recente (popularizado após a pandemia de \gls{covid} em 2020) e atende diferentes profissionais autônomos (relacionados à gerente por locação ou comissão), não uma equipe com objetivo comum. Desta forma, o problema central é a gerência da ocupação de cada profissional no espaço de trabalho, além do controle das finanças e da agenda dos clientes.

Nossa parceira Bruna já utilizava um sistema digital para gerenciamento do salão. Contudo, apesar dos benefícios trazidos pela solução, o sistema apresentava pontos insatisfatórios, sendo o principal deles a instabilidade da plataforma, que gerava insatisfação e perda de clientes.

Nossa solução consiste em criar uma aplicação \emph{web} que mantenha todas as funcionalidades que já atendem bem a Bruna como o agendamento \emph{on-line} e pesquisa de satisfação. Além disso, a plataforma incluirá funções ainda ausentes e ajustará requisitos funcionais e não funcionais cuja concepção é adequada, mas apresenta falhas, como o \emph{login} instável, senhas excessivamente complexas e erros recorrentes na troca de senha. De forma específica, nossa solução facilita o agendamento de serviços para as três entidades existentes no \emph{coworking} de beleza: 

\noindent\textbf{Para os Clientes Finais:} A plataforma possibilita o agendamento de serviços de forma intuitiva e flexível. Os clientes poderão escolher profissionais específicos ou optar pelo melhor horário disponível, visualizando facilmente a lista de prestadores, seus serviços, preços, tempo de execução e agendas atualizadas.

\noindent\textbf{Para os Profissionais Autônomos:} O sistema BS Beauty tem como propósito reforçar a autonomia dos profissionais sobre sua agenda e finanças. A aplicação permite bloquear horários, editar preços e a duração dos serviços, além de acompanhar os agendamentos realizados (sejam eles do dia, futuros ou passados) e visualizar relatórios detalhados com a receita gerada pelos serviços prestados.

\noindent\textbf{Para a Gestora:} Nossa parceira, Bruna, terá acesso a funcionalidades exclusivas que incluem análise de métrica de desempenho (a partir de \emph{dashboards)}, gerenciamento do aluguel ou comissão de cada profissional, visualização do fluxo de agendamentos em períodos específicos, envio de mensagens de \emph{marketing} e promoções aos clientes, e acesso a relatórios financeiros detalhados. Ademais, a gestora poderá incluir ou remover profissionais da plataforma conforme a necessidade.

Em síntese, a solução proposta é uma plataforma com \emph{login} simplificado (integrado ao \gls{sso} \footnote{Single Sign-On é um sistema que permite usar um único nome de usuário e senha para acessar vários serviços diferentes, sem precisar criar contas ou lembrar várias senhas.} do Google) e agendamento fácil e transparente para os clientes (incluindo todos os serviços e atributos necessários para uma melhor decisão). Também contará com agenda totalmente controlada pelos profissionais, notificações de agendamento e cancelamento para clientes e profissionais, lista de aniversariantes, desconto por frequência e retenção de dados em conformidade com a \gls{lgpd}. Além disso, a gerente terá acesso à relatórios financeiros e \emph{dashboards} com métricas de produtividade e frequência de clientes.
\section{Justificativa}

Uma pesquisa de 2023 do SEBRAE indica mais de 1,3 milhão de atividades econômicas ligadas a negócios de beleza no Brasil, abrangendo serviços, indústria e comércio, e gerando aproximadamente R\$ 75 bilhões em faturamento anual \cite{sebrae2023forca}. Neste cenário robusto, que movimentou cerca de 27 bilhões de dólares em 2024 \cite{ecommercenapratica2025}, os desafios operacionais crescem cada vez mais: 

\begin{itemize}
	\item Até 30\% do tempo de um pequeno empreendedor é consumido por tarefas administrativas \cite{senac2022};
	\item Taxa média de não comparecimento de clientes atinge 25\% \cite{booksy2022};
	\item Perda de 20\% da receita por não comparecimento \cite{abihpec2021};
	\item Média de 15 horas semanais dedicadas ao controle manual de agenda e finanças \cite{fgv2020};
	\item Insatisfação de 40\% dos clientes devido a falhas de comunicação e alterações de última hora \cite{mindminers2022}.
\end{itemize}

Paralelamente ao crescimento do setor de beleza, o modelo de coworking, originado em ambientes de escritório, expandiu-se para salões, permitindo o compartilhamento de espaços e recursos e a redução de custos \cite{sebrae_coworking,sebraesc2025}. 

Nesse contexto promissor, justifica-se o projeto de extensão \emph{BS Beauty}, destinado a desenvolver uma aplicação web customizada para o gerenciamento de salões em modelo coworking, sob a coordenação de nossa parceira de extensão Bruna. Ao digitalizar e centralizar processos principais, a BS Beauty empodera pequenos empreendedores reduzindo custos operacionais e minimizando erros humanos, melhora a experiência do cliente, eleva a receita dos profissionais por meio do controle preciso de comissões e frequências, e oferece a oportunidade de \emph{insights} estratégicos através de dashboards e relatórios financeiros detalhados. 

Dessa forma, a solução não só supera os problemas de instabilidade e excesso de esforço administrativo, mas também gera valor para todos os envolvidos no salão de beleza. Além disso, como iniciativa de extensão, o projeto permite que os alunos‐desenvolvedores coloquem em prática e melhorem os conhecimentos técnicos e de gestão,  aprendendo com desafios reais de requisitos, usabilidade e performance. Assim, é possível aproximar a graduação das demandas do mercado.



\section{Análise da Concorrência}
\label{sec:analise-concorrencia}
Foi conduzida uma pesquisa de mercado centrada em plataformas brasileiras que combinam agendamento \emph{on-line} e gestão financeira para espaços de beleza no modelo \emph{coworking}. Deste levantamento emergiram três empresas que servirão de referência nesta análise: uma já amplamente consolidada no mercado nacional — embora atue além do universo \emph{coworking} — e outras duas que, apesar de conhecidas, ainda estão em expansão, mas com foco mais relacionado ao da nossa proposta, o que as torna concorrentes que merecem maior atenção estratégica.

\subsection{Trinks}

%inicio de figura
\begin{figure}[htb]
	\centering
	\caption{Logo plataforma Trinks}
	\includegraphics[width=0.5\textwidth]{cap01-Introducao/Images/1.4.1_Trinks}
	\label{fig:Trinks}
	\fonte{\cite{Trinks}}
\end{figure}

 \FloatBarrier

Trinks é uma plataforma já bem consolidada no mercado de gestão de negócios de beleza, com soluções
personalizadas para barbearias, salões de beleza e clínicas de estética. Criada em 2012, é hoje a
plataforma de gestão para beleza com a maior base instalada do país, englobando aproximadamente
2,8\,milhões de usuários e mais de 40\,mil estabelecimentos, sediada no Rio de Janeiro. A plataforma começou como um empreendimento de consultoria em software personalizado, mas logo identificou uma oportunidade no mercado da beleza e mudou de nicho. Em 2024, foi adquirida pelo grupo Stone, o que alavancou ainda mais funcionalidades do aplicativo, como o autoatendimento.
Atualmente, a Trinks oferece software de \emph{back-office} (conjunto de módulos internos que controlam o funcionamento do negócio como finanças, estoque, comissões e relatórios), \emph{marketplace \gls{b2c}} e meios de pagamento próprios (Trinks \emph{Pay}), funcionando praticamente como um “\gls{erp} + \emph{iFood}” para salões e barbearias. Existe um
plano grátis que engloba apenas 150 agendamentos por mês, e os planos pagos variam de R\$ 59 a R\$ 249/mês \cite{Trinks}.

Além dos serviços comuns, seus principais diferenciais são:

\begin{itemize}
	\item \gls{pdv} completo: integração com \gls{tef}, \gls{pix} e split de comissão, atendendo desde \gls{mei}s até redes com exigência de \gls{nfce} e \gls{sat}/\gls{ecf};
	\item Estrutura em nuvem madura, com \gls{sla} de 99,9\,\% e aplicativos nativos para \gls{ios}/Android.
	\item \emph{Marketplace} \textit{Trinks.com}, que gera maior fluxo de clientes, expõe o salão ao
	público final e permite pagamento antecipado;
\end{itemize}

Apesar dos grandes benefícios, identificamos algumas brechas do ponto de vista do negócio da nossa
parceira de extensão, Bruna:

\begin{itemize}
	\item A interface pode ser considerada “poluída” para clientes iniciantes, devido ao grande
	número de funcionalidades;
	\item Há pouco foco no aluguel de estações típico do \emph{coworking}, exigindo ajustes manuais de
	comissão;
	\item Maior parte das funcionalidades estão presentes apenas nos planos superiores.
\end{itemize}

\subsection{Gendo}

%inicio de figura
\begin{figure}[htb]
	\centering
	\caption{Logo plataforma Gendo}
	\includegraphics[width=0.4\textwidth]{cap01-Introducao/Images/1.4.2_Gendo}
	\label{fig:Gendo}
	\fonte{\cite{Gendo}}
\end{figure}

 \FloatBarrier

Lançado em 2017 e sediado em Curitiba-PR, o Gendo se posiciona como um hub\footnote{Hub: plataforma centralizada que integra agenda, \gls{pdv}, finanças e pagamentos em um único ambiente, funcionando como “nó” que organiza os fluxos de dados do negócio.} de gestão 100\,\% em nuvem para negócios além do setor da beleza, como estética, saúde, bem-estar, \emph{pet-shop} e mais recentemente, espaços em formato \emph{coworking}. 
Atualmente mantém mais de 10 mil assinantes, com maior penetração nas regiões Sul e Sudeste do Brasil. Foi criado no modelo \gls{saas} com o intuito de oferecer prontamente agenda \emph{on-line}, automação de lembretes (\emph{e-mail/WhatsApp}), módulo financeiro completo e integrações com \emph{gateways} de pagamento (Stone, Cielo e Mercado Pago). Atualmente, os planos são somente pagos e variam de R\$ 32 a R\$ 293/mês, após 14 dias de teste gratuito \cite{Gendo}.

Seus principais diferenciais são:
\begin{itemize}
	\item Caixa do profissional: Módulo pensado para \emph{coworking}, possibilitando débito automático de aluguel de estação e visualização dos ganhos de cada profissional;
	\item Aplicativo Gendo Pro (\gls{ios} e Android): permite ao profissional ver a agenda, acompanhar comissões, pedir saques e registrar fotos de antes e depois dos serviços;
	\item Relatórios instantâneos: exibem \emph{ticket} médio, previsão de faturamento e dados de cancelamentos, com opção de exportar para \emph{Excel}.
\end{itemize}


Já os maiores pontos de melhoria identificados são:
\begin{itemize}
	\item Dependência de \emph{gateways} externos, o que adiciona custo extra ao \emph{split} \footnote{\emph{Split} é a divisão automática do pagamento entre salão e profissional que, se feita por um \emph{gateway} externo, gera uma taxa extra.};
	\item Relatórios fiscais avançados disponíveis apenas no plano \emph{Premium}.
\end{itemize}

\subsection{Avec}

\begin{figure}[htb]
	\centering
	\caption{Logo plataforma Avec}
	\includegraphics[width=0.4\textwidth]{cap01-Introducao/Images/1.4.3_Avec}
	\label{fig:Avec}
	\fonte{\cite{Avec}}
\end{figure}

 \FloatBarrier

Atualmente, a Avec é a principal concorrente do nosso projeto, pois a entidade parceira que motivou este trabalho utiliza essa plataforma para gerenciar seu salão de beleza em modelo \emph{coworking}. Por esse motivo, ela foi adotada como referência: buscamos manter as funcionalidades que já funcionam bem na Avec e, ao mesmo tempo, acrescentar ou aprimorar recursos que ainda fazem falta para a nossa parceira.

Lançada em 2014 e sediada em São Paulo-SP, a Avec se apresenta como solução``360º'' para salões, barbearias, esmaltarias, spas e estúdios de tatuagem. A plataforma integra software de gestão, um sistema próprio de pagamentos (\emph{Avec Pay}) e um \emph{marketplace \gls{b2c}} que encaminha novos clientes aos estabelecimentos. Segundo a empresa, mais de 40 mil negócios utilizam o serviço no
Brasil. Também desenvolvida no modelo \gls{saas}, a ferramenta oferece agenda \emph{on-line} multiprofissional com confirmações via \emph{WhatsApp} ou \gls{sms}, \gls{pdv} completo com \gls{tef}, \gls{pix} e \emph{split} interno de comissões, além de módulo financeiro integrado. Dispõe ainda de uma carteira digital empregada em pacotes pré-pagos, \emph{gift-cards}, \emph{cashback}, e possui dois aplicativos: o \emph{Avec}, voltado ao cliente final, e o \emph{Avec Pro}, destinado aos profissionais. Há um plano gratuito ``\emph{Avec Go}'' que inclui funções básicas e cobra apenas a taxa transacional, enquanto os planos pagos variam de R\$ 77 a
R\$249 por mês \cite{Avec}.


Com base no \emph{feedback} da nossa entidade parceira, destacam-se três funcionalidades que
a plataforma \emph{Avec} executa bem:
\begin{itemize}
	\item \emph{Split} instantâneo de comissões, dispensando \emph{gateways} externos;
	\item \emph{Marketplace} \gls{b2c} e aplicativo do cliente, que ampliam a visibilidade do salão e aumentam os agendamentos \emph{on-line};
	\item Aplicativo \emph{Avec Pro} (\gls{ios}/Android), no qual o profissional acompanha agenda,
	comissões, saques e registra fotos de “antes e depois” dos serviços.
\end{itemize}

As principais brechas identificadas são:
\begin{itemize}
	\item módulos fiscais avançados (\gls{nfe} e \gls{sat}) disponíveis apenas nos planos superiores;
	\item dependência do hardware e das tarifas do próprio \emph{Avec Pay} para uso pleno do sistema;
	\item custos adicionais para envios em massa de \gls{sms}/\emph{WhatsApp} em campanhas de \emph{marketing};
	\item instabilidade recorrente: o domínio eventualmente fica fora do ar.
\end{itemize}

%seção 1.4 quadro comparativo

\subsection{Quadro comparativo}

\begin{quadro}[htb]
	\caption{\label{frame:comparativo_concorrência}Comparação entre as plataformas concorrentes e a aplicação proposta}
	\footnotesize
	\setlength{\tabcolsep}{4pt}
	\begin{tabular}{|p{6.8cm}|c|c|c|c|}
		\hline
		\textbf{Recurso}                                   & \textbf{Trinks} & \textbf{Gendo} & \textbf{Avec} & \textbf{BS Beauty}\\ \hline 
		Aplicação \textit{web}  & — & \checkmark & \checkmark & \checkmark \\ \hline
		Flexível para \emph{coworking}  & — & \checkmark & \checkmark & \checkmark \\ \hline
		Controle de acesso para gestão  & — & \checkmark  & \checkmark & \checkmark \\ \hline
		Agendamento de serviços 100\% \emph{on-line} & \checkmark & \checkmark & \checkmark & \checkmark \\ \hline
		Controle de conflitos de agenda                    & \checkmark & \checkmark & \checkmark & \checkmark \\ \hline
		Avaliação pós-serviço  & \checkmark & \checkmark  & \checkmark  & \checkmark \\ \hline
		Plataforma do cliente                               & \checkmark & \checkmark & \checkmark & \checkmark \\ \hline
		Plataforma do profissional                                      & \checkmark & \checkmark & \checkmark & \checkmark \\ \hline
		Confirmação automática (\emph{WhatsApp} / \gls{sms} / \emph{e-mail})                    & \checkmark & \checkmark & \checkmark & \checkmark \\ \hline
		Cálculo de \emph{Split} de comissão      & \checkmark & \checkmark & \checkmark & \checkmark  \\ \hline
		Pagamento \emph{on-line}                 & \checkmark & — & \checkmark & \ — \\ \hline
		\emph{\emph{Marketplace} \gls{b2c}}                         & \checkmark & — & \checkmark & \ — \\ \hline
		\emph{Marketing} integrado (envio em massa de \gls{sms}/\emph{WhatsApp/e-mail})  & \checkmark & — & \checkmark & \checkmark \\ \hline
		Programa de indicação  & \checkmark & — & \checkmark & \checkmark \\ \hline
		Relatório financeiro em tempo real    & \checkmark & \checkmark & \checkmark & \checkmark \\ \hline
		\emph{Login} simplificado com integração Google            & \checkmark & \checkmark &  — & \checkmark \\ \hline
		Plano gratuito disponível                                           & \checkmark & — & \checkmark & — \\ \hline
		Lista de aniversariantes para promoções  & \checkmark & \checkmark & — & \checkmark \\ \hline
		
	\end{tabular}
	\fonte{Produzido pelos autores}
\end{quadro}


	% --------------------------------------
	
	\chapter{Revisão da Literatura}
	explicar os topicos principais do projeto
	
	%o input do capitulo 2 já contem todas as seções dele
	
	%este arquivo contém todas as seções do capitulo 2

\section{Histórico do assunto}
%----------------------------------------------
\section{Atualidade do assunto}
%----------------------------------------------
\section{Outros contextos do assunto (opcional)}	
	%  ----------------------------------------------------------
	\chapter{Gestão do Projeto}
	% ---
	\section{Organização da equipe}
	\subsection{Responsabilidades/papéis/atividades}
	% ---
	\section{Metodologias de Gestão e Desenvolvimento}
	\subsection{Scrum}
	\subsection{Sprints}
	% ---
	\section{Repositórios da aplicação}
	\subsection{Definição do Repositório da Aplicação}
	\subsection{Link do repositório e especificações para acesso}
	
	%-------------------------------------------
	
	\chapter{Desenvolvimento do projeto}
	% ---
	\section{Escopo do Projeto}
	\subsection{Regras de Negócio}
	\subsection{Requisitos Funcionais}
	\subsection{Requisitos Não Funcionais}
	% ---
	\section{Histórias de usuário}
	se aplicável para o scrum
	\subsection{Descrição das Histórias de Usuário}
	% ---
	\section{Arquitetura}
	\subsection{Definições da arquitetura}
	\subsection{Diagrama da arquitetura}
	% ---
	\section{Tecnologias}
	\subsection{Front-End}
	\subsection{Back-End}
	\subsection{Banco de dados}
	\subsection{Infraestrutura}
	\subsubsection{Amazon Web Services (AWS)}
	exemplo, colocar todos, abrindo demais itens
	%---
	\section{Testes e Manutenibilidade}
	\subsection{Plano de Testes}
	\subsection{Análise Estática}
	\subsection{Testes Funcionais}
	\subsection{Testes Unitários}
	\subsection{Testes de Componente}
	\subsection{Testes de Integração}
	\subsection{Testes end-to-end}
	\subsection{Testes não Funcionais}
	\subsection{Testes de Performance}
	\subsection{Testes de Carga}
	\subsection{Testes de Configuração}
	\subsection{Testes Automatizados}
	\subsection{Logs}
	\subsection{Code Convention}
	%---
	\section{Segurança, Privacidade e Legislação}
	\subsection{Critérios de segurança e privacidade}
	\subsection{Observância à Legislação}
	%---
	\section{Modelo de Banco de Dados}
	\subsection{Modelo Entidade Relacionamento - MER}
	\subsection{DIagrama Entidade Relacionamento - DER}
	tabelas
	\subsection{Dicionário de Dados}
	%---   
	\section{Cronograma}
	pensar no projeto todo, não só MVP
	\subsection{Análise da Duração do Projeto}
	considerar o gerenciamento ágil
	% ----------------------------------------------------------
	
	\chapter{Viabilidade Financeira}
	
	mesmo usando uma hospedagem gratis (AWS), precisamos pesquisar uma paga para colocar na tabela de custos
	
	\section{Custos}
	
	\section{Receitas}
	\section{Cenário realista}
	\section{Cenário Otimista}
	\section{Cenário Pessimista}
	% ---------------------------------------------------------------
	
	\chapter{Considerações Finais}
	\section{Dificuldades, escolhas e Descartes}
	\cite{ibge1993}
	% ----------------------------------------------------------
	
	ver documento "abntex2-modelo-include-comandos" para dicas e tutoriais de como fazer tabelas, graficos, quadros, inserir imagens e documentos externos etc aqui. 
	% ---
	
	%Página de referencias
	\renewcommand{\bibname}{Referências Bibliográficas} %renomeia o título da página de referencias
	\bibliography{referencias-bibliograficas}
	\bibliographystyle{abntex2-alf}
	\bibliographystyle{abntex2-num}
	
	
	% ----------------------------------------------------------
	% Glossário
	% ----------------------------------------------------------
	%
	% Consulte o manual da classe abntex2 para orientações sobre o glossário.
	%
	%\glossary
	
	% ----------------------------------------------------------
	% Apêndices
	% ----------------------------------------------------------
	
	% ---
	% Inicia os apêndices
	% ---
	\begin{apendicesenv}
		
		% Imprime uma página indicando o início dos apêndices
		\partapendices
		
		% ----------------------------------------------------------
		\chapter{exemplo 1}
		% ----------------------------------------------------------
		
		materiais desenvolvidos pelo próprio autor do trabalho
		
	\end{apendicesenv}
	
	% ----------------------------------------------------------
	% Anexos
	% Inicia os anexos
	
	\begin{anexosenv}
		
		% Imprime uma página indicando o início dos anexos
		\partanexos
		
		% ---
		\chapter{exemplo 1}
		matérias de outras fontes que não o próprio autor do trabalho
		
	\end{anexosenv}
	
	%---------------------------------------------------------------------
	% INDICE REMISSIVO
	%---------------------------------------------------------------------
	\phantompart
	\printindex
	%---------------------------------------------------------------------
	
\end{document}
