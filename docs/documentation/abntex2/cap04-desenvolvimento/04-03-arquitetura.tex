\section{Arquitetura}

Nessa seção, apresenta-se a arquitetura da aplicação, que define como os componentes do sistema interagem entre si e com o usuário.

\subsection{Definições da arquitetura}

O \emph{BSBeauty} adota uma arquitetura de software em camadas, na qual cada camada da aplicação é responsável por uma função específica dentro do sistema. Essa arquitetura organiza o código de forma modular, promovendo a separação de responsabilidades, a facilidade de manutenção e escalabilidade da aplicação.

No contexto do projeto, o \emph{front-end} é responsável pela camada de \emph{View}, sendo implementada com a biblioteca \emph{React}. Ele centraliza todas as funcionalidades relacionadas à interface do usuário, incluindo a renderização dinâmica das telas e a interação com o usuário final. O \emph{front-end} consome os dados fornecidos pela \gls{api} do \emph{back-end} por meio de requisições \gls{http}, seguindo os princípios RESTFul~\cite{RESTArch}.

O \emph{back-end}, por sua vez, foi estruturado com base nos princípios da arquitetura em camadas descrita por Fowler~\cite{FowlerPatterns}. Nesse modelo, o sistema é dividido em camadas lógicas, cada uma com uma responsabilidade bem definida. Essa divisão permite o desacoplamento entre a lógica de negócio, o controle de fluxo e o acesso de dados, tornando o sistema mais coeso e flexível. 

Essa escolha organiza o código de forma modular, o que proporciona uma definição clara das responsabilidades de cada camada. A seguir, são apresentadas as camadas que compõem a aplicação:

\begin{itemize}
  \item \textbf{\emph{View:}} Implementada no \emph{front-end} com a biblioteca \emph{React}, é responsável por toda a interface do usuário, incluindo a apresentação dos dados e a captura das interações do usuário. Ela consome os dados fornecidos pela \gls{api} do \emph{back-end} para renderizar as telas de forma dinâmica.
  \item \textbf{\emph{Controller:}} Responsável por receber e tratar as requisições provenientes do cliente (\emph{front-end}), encaminhando-as para a camada de serviço correspondente. Esta camada lida diretamente com aspectos de infraestrutura externa,  como servidores de borda ou \emph{gateways} de entrada da aplicação, servindo como ponto inicial de processamento das solicitações.
  \item \textbf{\emph{Service:}} Centraliza a lógica de negócio da aplicação. É responsável por processar as regras do domínio e pode tanto consumir outras funções de serviço quanto interagir com a camada de persistência.
  \item \textbf{\emph{Repository:}} Trata das operações relacionadas à persistência de dados. Atua como uma interface entre os serviços e os mecanismos de armazenamento, como bancos de dados relacionais ou caches, promovendo o desacoplamento da lógica de negócio em relação à camada de dados.
\end{itemize}



Esse modelo de arquitetura é amplamente adotado em frameworks modernos de desenvolvimento \emph{back-end}, como o \emph{Spring Boot}\cite{SpringBootLayers} (Java) e o \emph{NestJS}\cite{NestJSLayers} (Node.js), que também seguem o padrão de separação entre  as camadas de \emph{Controller}, \emph{Service} e \emph{Repository}. A escolha dessa arquitetura dentro da aplicação proporciona uma série de vantagens: primeiro, há um claro desacoplamento entre a interface do usuário e a lógica de negócio, o que permite que diferentes equipes possam realizar o desenvolvimento do \emph{front-end} e do \emph{back-end} de forma independente, utilizando tecnologias e ferramentas específicas para cada camada. Além disso, cada camada pode ser mantida e evoluída de forma independente, o que facilita a manutenção do sistema ao longo do tempo. Essa separação melhora significativamente a testabilidade, permitindo a criação de testes isolados para cada camada, o que contribui para a qualidade do software. Por fim, o modelo favorece a escalabilidade e o reuso de componentes, uma vez que cada camada pode ser dimensionada conforme a demanda e reutilizada em diferentes contextos, garantindo maior modularidade e flexibilidade na evolução do sistema.

\subsection{Diagrama da arquitetura}

Esta subseção apresenta dois diagramas que representam a arquitetura do sistema desenvolvido: o diagrama de componentes e o diagrama de implantação. Esses diagramas auxiliam na visualização do relacionamento entre os componentes do sistema, bem como a sua distribuição nos ambientes computacionais.

\subsubsection{Diagrama de componentes}

O diagrama de componentes representa a estrutura modular dos principais módulos do sistema, evidenciando as dependências e as formas de comunicação entre eles, como comentado em ~\cite{Booch2005}. Ele demonstra como os componentes interagem por meio de interfaces — que definem contratos de uso — e implementações — que oferecem as funcionalidades esperadas.

Esse tipo de diagrama é útil para visualizar a estrutura modular da aplicação, facilitando o entendimento da separação de responsabilidades e da reutilização de código, além de apoiar decisões relacionadas à manutenção e evolução do sistema.

\begin{figure}[htb]
  \centering
  \caption{Diagrama de componentes da aplicação}
  \includegraphics[width=1.1\textwidth]{cap04-desenvolvimento/images/4-3-2-1-diagrama-componentes}
  \label{fig:diagrama-componente}
  \fonte{Produzido pelos autores}
\end{figure}
\FloatBarrier


No diagrama de componentes proposto, a aplicação é composta por diversos módulos que representam funcionalidades distintas e organizadas de forma modular. Esses componentes são responsáveis por encapsular regras de negócio e oferecer interfaces bem definidas para comunicação entre si. A seguir, são descritos os principais módulos do sistema:

\begin{itemize}

  \item \textbf{\emph{Analytics}}: Responsável por gerar relatórios e fornecer estatísticas baseadas nas informações do sistema. Auxilia principalmente no acompanhamento de desempenho da plataforma por parte dos gerentes do salão.
  
  \item \textbf{\emph{Appointments}}: Gerencia os agendamentos realizados pelos clientes. Engloba tanto a criação, listagem e atualização dos agendamentos quanto a associação com os serviços ofertados.
  
  \item \textbf{\emph{Auth}}: Componente responsável pela autenticação de usuários e integração com serviços externos, como o \emph{Google OAuth} \cite{GoogleOAuth}. Garante um processo facilitado de login para os usuários e autenticação no acesso das funcionalidades protegidas da aplicação.
  
  \item \textbf{\emph{Blocked Times}}: Trata do registro de períodos temporários em que um profissional não está disponível (por exemplo, férias, ausências ou compromissos). Diferencia-se dos turnos regulares (\emph{Shift}) por representar exceções pontuais que impedem o agendamento de atendimentos.
  
  \item \textbf{\emph{Customers}}: Controla os dados relacionados aos clientes da plataforma. Permite o cadastro, consulta e edição de informações do perfil dos usuários.
  
  \item \textbf{\emph{Notifications}}: Responsável por enviar notificações aos usuários, como lembretes, confirmações de agendamento e atualizações importantes. Pode incluir o envio por \emph{e-mail} ou outros canais.
  
  \item \textbf{\emph{Offers}}: Define a relação entre profissionais e os serviços que eles oferecem. Cada oferta especifica o tempo estimado e o valor cobrado por um profissional para realizar determinado serviço. Esse componente é fundamental para a composição de um agendamento, pois determina quais combinações de profissional e serviço estão disponíveis.
  
  \item \textbf{\emph{Payment Records}}: Possibilita que as profissionais do salão registrem os pagamentos realizados por clientes em relação aos serviços agendados, sendo utilizado também para geração de relatórios financeiros.
  
  \item \textbf{\emph{Permissions}}: Gerencia as permissões granulares do sistema, definindo ações e privilégios que podem ser atribuídos a papéis (roles) ou diretamente a usuários. Esse módulo é utilizado pelo mecanismo de autorização para controlar o acesso a recursos e operações sensíveis da aplicação.

  \item \textbf{\emph{Professionals}}: Administra os dados dos profissionais que prestam serviços na aplicação, incluindo informações cadastrais, disponibilidade e associação a serviços específicos.
  
  \item \textbf{\emph{Rating}}: Permite que os clientes avaliem os serviços e os profissionais após os atendimentos, promovendo um sistema de \emph{feedback} contínuo.

  \item \textbf{\emph{Roles}}: Responsável pelo gerenciamento de papéis (roles) que agrupam permissões, simplificando a atribuição de conjuntos de privilégios a profissionais ou usuários. Facilita a administração de acesso em grande escala e a composição de políticas de autorização.
  
  \item \textbf{\emph{Salon Info}}: Permite à gerente atualizar informações relevantes referentes ao salão de beleza (como contato, redes sociais, localização, horário de funcionamento e antecedência mínima) que podem ser visualizadas tanto por clientes quanto por profissionais.

  \item \textbf{\emph{Services}}: Representa os serviços oferecidos pela empresa, armazenando informações descritivas como nome e descrição. Este módulo não define valores ou tempos de execução, pois esses dados são especificados nas ofertas individuais de cada profissional (por meio do módulo \textit{Offers}).

  \item \textbf{\emph{Shift}}: Trata do controle de turnos de trabalho dos profissionais, possibilitando a definição de horários disponíveis para realização dos agendamentos.
  
  \item \textbf{\emph{Static Files (Front-end)}}: Componente responsável por servir os arquivos estáticos da interface do usuário, gerados após o processo de \textit{build} do projeto \emph{front-end} (em \emph{React}). Inclui arquivos \gls{html}, \gls{css} e \emph{JavaScript} que são entregues ao navegador do usuário final via servidor \emph{NGINX}.
\end{itemize}

\subsubsection{Diagrama de implantação}

O diagrama de implantação mostra como os componentes do sistema estão distribuídos em termos de infraestrutura~\cite{Booch2005}, seja em servidores físicos ou ambientes virtuais. Ele ajuda a entender onde cada parte da aplicação está rodando, como os serviços se conectam entre si e quais recursos são necessários para que tudo funcione da forma adequada no ambiente de produção.

Esse tipo de representação é especialmente útil para quem for implantar ou manter o sistema, pois facilita a visualização de elementos como servidores, banco de dados, \emph{gateways} de rede, e outras dependências da aplicação. Além disso, o diagrama contribui para o planejamento de permissões, acessos e políticas de segurança que precisam ser configuradas na infraestrutura.

\begin{figure}[htb]
  \centering
  \caption{Diagrama de implantação da aplicação}
  \includegraphics[width=\textwidth]{cap04-desenvolvimento/images/4-3-2-2-diagrama-implantacao}
  \label{fig:diagrama-implantacao}
   \fonte{Produzido pelos autores}
\end{figure}
\FloatBarrier

O diagrama acima é composto pelos seguintes componentes:
\begin{itemize}
  \item \textbf{\textit{User Device:}} No diagrama proposto, por se tratar de uma aplicação \emph{web}, o dispositivo do usuário será responsável por executar a aplicação \textit{client-side}, que interpreta através do navegador os arquivos \gls{css}, \emph{JavaScript} e \gls{html} gerados no empacotamento ou \textit{build} do projeto feito com a biblioteca \emph{React}. Ademais, esse dispositivo é composto por alguns artefatos importantes que são obtidos pelo navegador por meio de uma requisição ao \textit{Proxy Reverso}: \gls{css} (\emph{style)}, \emph{JavaScript} (\emph{script)} e \gls{html}.

  \item \textbf{\textit{Amazon Public Instance (Device):}} A \textit{Amazon Public Instance} é uma instância \gls{ec2} que possui um \gls{ip} público, o que permite que ela seja acessada diretamente pelo usuário ou resolvida por \gls{dns}. Por esse motivo, nela são executados apenas os componentes que devem estar disponíveis publicamente ao cliente final:
    \begin{itemize}
      \item \textit{Componente NGINX}: Serviço que atua como \textit{Proxy Reverso}~\cite{NGINXDocs} para a aplicação que está sendo executada em uma instância privada na arquitetura proposta. Para viabilizar conexões \gls{https}, o \emph{NGINX} utiliza certificados digitais emitidos gratuitamente pelo serviço \emph{Let's Encrypt}, utilizando a ferramenta \emph{Certbot}~\cite{LetsEncryptWithCertbot}, que automatiza todo o processo de emissão e renovação dos certificados. Este serviço é instalado como um \textbf{artefato} adicional no ambiente da instância pública, sendo integrado ao próprio ciclo de configuração e inicialização do \emph{NGINX}.
    \end{itemize}

  \item \textbf{\textit{Amazon Private Instance (Device):}} Por outro lado, a \textit{Amazon Private Instance} é composta por uma instância \gls{ec2} com restrições de rede, o que significa que seu acesso é limitado à rede interna e não possui \gls{ip} público. Nessa instância, componentes da arquitetura que não precisam estar disponíveis de forma pública são o caso de uso perfeito, uma vez que garante maior segurança e isolamento dos aspectos internos da aplicação. Em seu interior, ela é composta pelos seguintes componentes:
    \begin{itemize}
      \item A \textit{\gls{api} (Back-end em NodeJS)}: Principal serviço da aplicação, sendo o responsável por prover os \emph{"endpoints"} que fornecem os dados e arquivos estáticos para o \textit{front-end} através de uma \textit{\gls{api} RestFull}, se comunicando com o banco de dados, um componente que é executado no mesmo dispositivo.
      \item \textit{Banco de Dados (\gls{sgbd} MariaDB)}: Serviço de \gls{sgbd} que provê os dados para o \textit{back-end} da aplicação, o que possibilita a persistência e consulta de forma eficiente. Para garantir a persistência dos dados gerenciados pelo MariaDB, o contêiner utiliza volumes \emph{Docker} montados na instância \gls{ec2} privada. Isso assegura que os dados não sejam perdidos em reinicializações do contêiner.
      \item \textit{Redis}: Banco de dados em memória utilizado para a persistência de dados efêmeros e de acesso rápido, como cache de respostas, sessões e armazenamento temporário de códigos de autenticação/validação. O uso do \emph{Redis} melhora a performance em operações que exigem baixa latência e é empregado para suportar tarefas de autenticação (geração e busca de códigos) e mecanismos de cache da aplicação.
    \end{itemize}
    Além dos serviços em execução, a instância privada também contém os seguintes artefatos essenciais para o empacotamento e execução dos serviços via contêineres \emph{Docker}~\cite{DockerDocs}:
    \begin{itemize}
      \item \textit{Dockerfile}: Esse artefato descreve as instruções necessárias para criar a imagem \emph{Docker} da aplicação \textit{back-end}, especificando o ambiente base (como a imagem do Node.js), os arquivos a serem copiados, dependências a serem instaladas e os comandos de inicialização da aplicação.
      \item \textit{docker-compose.yaml}: Esse arquivo é utilizado como ferramenta de orquestração para os serviços \emph{Docker}~\cite{DockerComposeDocs} da aplicação. Ele define a configuração dos contêineres da aplicação, como o contêiner da \gls{api} e o do banco de dados, bem como as variáveis de ambiente, volumes, redes e dependências entre os serviços. É a partir deste artefato que os contêineres são gerados e executados de forma integrada.
    \end{itemize}

  \item \textbf{\textit{Google Cloud (Device):}} Localizada na nuvem pública da Google (\textit{Google Cloud}), essa \gls{api} é utilizada pelo \emph{back-end} da aplicação para realizar a autenticação de usuários através do protocolo \emph{OAuth} 2.0 \cite{GoogleOAuth}. Esse processo ocorre quando o usuário opta por fazer \emph{login} com sua conta Google. Nesse cenário, a aplicação redireciona o usuário para a tela de autenticação da Google, e após a confirmação, a \gls{api} recebe um \textit{token} de acesso que é utilizado para obter as informações do usuário autenticado. Esse fluxo garante uma autenticação segura, delegando a responsabilidade da validação de identidade à Google. Além disso, o \emph{back-end} integra-se ao servidor SMTP do Gmail para o envio de e-mails aos usuários (por exemplo: notificações de agendamento, códigos de verificação e recuperação de senha), conforme previsto no diagrama de componentes.
\end{itemize}

O fluxo de execução típico da aplicação baseado no diagrama de implantação acima segue os seguintes passos:
\begin{enumerate}
  \item O usuário acessa a aplicação pelo navegador, requisitando os arquivos \gls{html}\slash\gls{css}\slash\gls{js} ao servidor \emph{NGINX}.
  \item O \emph{NGINX}, atuando como proxy reverso, redireciona essas requisições para o serviço de \emph{back-end} na instância privada.
  \item A \gls{api} processa a requisição, acessa o banco de dados quando necessário e retorna os dados.
  \item Em caso de autenticação via Google, a \gls{api} redireciona o usuário para o serviço \emph{Google OAuth} \cite{GoogleOAuth}, que retorna um \emph{token} de acesso após o \emph{login}.
  \item Esse \emph{token} é utilizado pela \gls{api} para obter os dados do usuário autenticado e estabelecer uma sessão.
\end{enumerate}

Além da organização dos componentes do diagrama, a arquitetura também prioriza a segurança da comunicação e do acesso. O tráfego entre o navegador do usuário e a instância pública é realizado por meio do protocolo \gls{https}, o que garante a confidencialidade e a integridade dos dados transmitidos. Para isso, foi utilizado o serviço gratuito de certificação digital \emph{Let's Encrypt} em conjunto com a ferramenta \emph{Certbot}, que automatiza a emissão, renovação e instalação dos certificados \gls{tls} no servidor \emph{NGINX}.

Internamente, a comunicação entre o \emph{NGINX} e os serviços da instância privada ocorre seguindo regras específicas de segurança definidas na \gls{vpc}, utilizando mecanismos como \textit{Security Groups} e \textit{Route Tables}. Isso reduz significativamente a superfície de ataque da aplicação e assegura uma camada adicional de proteção para os aspectos internos.

\subsubsection{Diagrama de referência na \gls{aws}}

Com o objetivo de fornecer uma visão mais aprofundada da infraestrutura da aplicação na nuvem, o diagrama apresenta a disposição dos principais componentes implantados na arquitetura da \gls{aws}.

Este diagrama ilustra elementos de infraestrutura fundamentais como sub-redes públicas e privadas, resolução de \gls{dns}, \gls{vpc}, \emph{Bastion Server}, \gls{nat} \emph{Gateway}, \emph{Internet Gateway}, banco de dados, entre outros recursos. A representação facilita a compreensão técnica da topologia de rede e da distribuição dos serviços, evidenciando como a aplicação foi projetada para atender requisitos de segurança, escalabilidade e disponibilidade no ambiente da \gls{aws}. A seguir, descreve-se brevemente cada elemento presente no diagrama.

\begin{figure}[htb]
  \centering
  \caption{Diagrama Geral da Arquitetura}
  \includegraphics[width=\textwidth]{cap04-desenvolvimento/images/4-3-2-3-diagrama-geral}
  \label{fig:diagrama-geral}
  \fonte{Produzido pelos autores}
\end{figure}
\FloatBarrier

\begin{itemize}
  \item \textbf{\gls{vpc} (Virtual Private Cloud):} A aplicação opera dentro de uma \gls{vpc} personalizada com o bloco \gls{cidr} \texttt{10.0.0.0/16}, que abriga duas sub-redes: uma pública e outra privada, seguindo o princípio de segmentação de rede recomendado pela própria \gls{aws}~\cite{AWSBestPractices}.

  \item \textbf{Sub-rede pública (10.0.0.0/24):} Contém uma instância \gls{ec2} de pequeno porte (\texttt{t2.micro}) que executa o serviço \emph{NGINX}. Esse servidor atua como \emph{proxy} reverso, roteando as requisições provenientes da internet para os serviços internos hospedados em uma sub-rede privada.

  \item \textbf{Sub-rede privada (10.0.0.1/24):} Hospeda uma instância \gls{ec2} de médio porte (\texttt{t2.medium}), na qual são executados os contêineres da aplicação via \emph{Docker Compose}, incluindo o serviço de \emph{back-end} (Node.js) e o banco de dados relacional MariaDB.

  \item \textbf{\gls{nat} \emph{Gateway:}} Permite que os recursos da sub-rede privada (como a instância \gls{ec2} que executa os contêineres) realizem atualizações e acessos à internet de forma segura, sem que sejam diretamente acessíveis externamente.

  \item \textbf{Internet \emph{Gateway:}} Responsável por permitir o tráfego de entrada e saída entre a \gls{vpc} e a internet pública. Está associado à sub-rede pública e \gls{nat} \emph{Gateway}, permitindo que o \emph{NGINX} receba requisições externas e o \gls{nat} receba um \gls{ip}.

  \item \textbf{\emph{DuckDNS:}} Utilizado como serviço de \gls{dns} dinâmico gratuito~\cite{DuckDNSDocs}, permitindo que a aplicação seja acessada por um domínio estável (\texttt{bsbeauty.duckdns.org}), mesmo que o endereço \gls{ip} público da instância \gls{ec2} varie. A resolução de nome é feita de forma transparente para o usuário final, facilitando o acesso à aplicação.

  \item \textbf{Usuários externos (\emph{\gls{aws} General Users):}} Representam os clientes que acessam a aplicação via navegador. O tráfego \gls{http}/\gls{https} chega inicialmente ao serviço \emph{NGINX} na sub-rede pública, que encaminha as requisições para a instância privada onde os serviços da aplicação estão efetivamente em execução.
\end{itemize}

A separação entre sub-rede pública e privada segue boas práticas de segurança e isolamento de ambiente, recomendadas tanto pela documentação oficial da \gls{aws} quanto por autores renomados da área de arquitetura de software em nuvem, como em \cite{AWSBestPractices}. Ao manter os serviços internos em uma sub-rede privada, reduz-se a superfície de ataque da aplicação e melhora a resistência contra acessos não autorizados.

Além disso, a utilização do \emph{DuckDNS} simplifica a exposição da aplicação para o ambiente externo sem a necessidade de configurar manualmente um serviço de \gls{dns} ou pagar por domínios personalizados, o que se alinha aos objetivos de custo deste projeto.