\section{Escopo do Projeto}


Esta seção apresenta o escopo do sistema desenvolvido, com o objetivo de delimitar suas funcionalidades, comportamentos esperados e restrições de operação. As definições aqui descritas foram elaboradas com base em reuniões realizadas com a gestora do salão de beleza, nas quais foram discutidas as necessidades do negócio e as principais demandas da operação cotidiana.

Com base nessas interações, foram estabelecidas as regras de negócio que orientam o funcionamento do sistema, os requisitos funcionais que especificam os serviços a serem oferecidos aos usuários e os requisitos não funcionais que tratam de aspectos como desempenho, segurança e usabilidade.

A definição clara desses elementos é essencial para garantir a coerência do sistema com as necessidades dos usuários e para orientar a equipe de desenvolvimento ao longo das etapas do projeto.

\subsection{Regras de Negócio}

As regras de negócio definem os comportamentos, restrições e condições que regem o funcionamento do sistema, conforme as particularidades da \emph{BS Beauty Academy}, e representam diretrizes que devem ser respeitadas tanto no uso da aplicação quanto no desenvolvimento das funcionalidades.

A seguir, são listadas as principais regras de negócio identificadas:

\begin{longtable}{|p{1.2cm}|p{10cm}|p{4cm}|}
	\caption{Regras de Negócio} \label{tab:regras_de_negócio}\\
	\hline
	\textbf{ID} & \textbf{Descrição da Regra de Negócio} & \textbf{Área Impactada} \\
	\hline
	\endfirsthead
	
	\hline
	\textbf{ID} & \textbf{Descrição da Regra de Negócio} & \textbf{Área Impactada} \\
	\hline
	\endhead
	
	RN01 & Não há políticas ou penalidades referentes a cancelamentos ou atrasos. & Agenda \\
	\hline
	RN02 & Clientes podem agendar múltiplos serviços em um mesmo agendamento. & Agenda, Clientes \\
	\hline
	RN03 & Pagamentos são realizados só presencialmente. & Atendimento, Financeiro, \emph{Marketing} e Vendas \\
	\hline
	RN04 & Clientes devem avaliar os serviços concluídos. & Atendimento, Clientes \\
	\hline
	RN05 & Não deve haver sobreposição de horários em agendamentos. & Agenda, Atendimento \\
	\hline
	RN06 & Agendamentos devem ser feitos com, no mínimo, 3 horas de antecedência. & Agenda \\
	\hline
	RN07 & O cadastro e edição de funcionárias é feito exclusivamente pela gerente. & \gls{rh}/Profissionais \\
	\hline
	RN08 & Profissionais podem definir seus próprios horários de trabalho. & Agenda, \gls{rh}/Profissionais \\
	\hline
	RN09 & Profissionais podem bloquear horários específicos na agenda. & Agenda, \gls{rh}/Profissionais \\
	\hline
	RN10 & Preços e duração de serviços podem ser definidos pela profissional (autônoma) ou pela gerente (modelo comissionado). & Financeiro, \emph{Marketing} e Vendas, \gls{rh}/Profissionais \\
	\hline
	RN11 & O salão possui horário fixo de funcionamento (definido pela gerente). & Atendimento, \gls{rh}/Profissionais \\
	\hline
	RN12 & Clientes devem informar como conheceram o salão. & Atendimento, Clientes \\
	\hline
	RN13 & Programa de indicação para indicar 3 pessoas e ganhar um procedimento grátis. & Clientes, Financeiro, \emph{Marketing} e Vendas \\
	\hline
	RN14 & Clientes aniversariantes devem receber mensagens de aniversário e descontos. & Clientes, \emph{Marketing} e Vendas \\
	\hline
	RN15 & Clientes, profissionais e a gerente devem receber notificações referentes a agendamentos. & Agenda, Atendimento, Clientes, RH/Profissionais \\
	\hline
	RN16 & \emph{E-mails} e telefones cadastrados por clientes e profissionais devem ser únicos. & Clientes, \gls{rh}/Profissionais \\
	\hline
	RN17 & Cada serviço possui um tempo de duração definido ao cadastrá-lo. & Agenda, Atendimento \\
	\hline
	RN18 & Não é possível agendar serviços para datas passadas. & Agenda, Atendimento \\
	\hline
	RN19 & Serviços devem estar vinculados a uma ou mais profissionais especializadas. & Atendimento, \gls{rh}/Profissionais \\
	\hline
	RN20 & Gerente e profissionais devem receber relatórios diversos. & \gls{rh}/Profissionais \\
	\hline
	RN21 & O histórico de avaliações e agendamentos deve ser mantido. & Atendimento \\
	\hline
	RN22 & Clientes devem agendar com base no serviço ou profissional. & Agenda, Clientes \\
	\hline
	RN23 & A quantidade de clientes atendidos por uma profissional depende do tempo de cada procedimento. & Agenda, Atendimento, \gls{rh}/Profissionais \\
	\hline
	RN24 & Atualmente, o salão oferece serviços de depilação, unhas, estética corporal e facial, sobrancelhas, remoção e cílios. & Agenda, Atendimento \\
	\hline
	RN25 & Profissionais podem criar seus próprios serviços (com limitações de acesso). & Agenda, \gls{rh}/Profissionais \\
	\hline
	RN26 & A gerente deve aprovar os novos serviços que forem criados. & Agenda, \gls{rh}/Profissionais. \\
	\hline
	
\end{longtable}

\subsection{Requisitos Funcionais}

Os requisitos funcionais representam as funcionalidades que o sistema deve implementar a fim de garantir o cumprimento das regras de negócio definidas durante as reuniões entre a equipe de desenvolvimento e a gestora do estabelecimento. Esses requisitos foram organizados de forma a manter uma correspondência clara com as regras de negócio, permitindo maior rastreabilidade e coerência entre as decisões do domínio e a implementação técnica do sistema.

\begin{longtable}{|p{0.1\textwidth}|p{0.85\textwidth}|}
	\caption{Requisitos Funcionais} \label{tab:requisitos-funcionais}\\
	\hline
	\textbf{ID} & \textbf{Descrição do Requisito Funcional} \\
	\hline
	\endfirsthead
	
	\hline
	\textbf{ID} & \textbf{Descrição do Requisito Funcional} \\
	\hline
	\endhead
	
	% RN01
	\multicolumn{2}{|l|}{\textbf{RN01}} \\ \hline
	RF01-01 & O sistema não deve bloquear clientes que se atrasaram em agendamentos anteriores. \\ \hline
	RF01-02 & O sistema deve permitir cancelamentos de agendamentos sem aplicação de multas ou penalidades. \\ \hline
	
	% RN02
	\multicolumn{2}{|l|}{\textbf{RN02}} \\ \hline
	RF02-01 & O sistema deve permitir a seleção de múltiplos serviços em uma única operação de agendamento. \\ \hline
	RF02-02 & O sistema deve calcular o tempo total necessário para todos os serviços selecionados. \\ \hline
	RF02-03 & O sistema deve calcular o valor total de todos os serviços agendados. \\ \hline
	RF02-04 & O sistema deve permitir a visualização do resumo dos múltiplos serviços agendados. \\ \hline
	RF02-05 & O sistema deve verificar a disponibilidade de profissionais para realizar os serviços selecionados no período necessário. \\ \hline
	
	% RN03
	\multicolumn{2}{|l|}{\textbf{RN03}} \\ \hline
	RF03-01 & O sistema não deve solicitar informações de pagamento durante o agendamento \emph{on-line}. \\ \hline
	RF03-02 & O sistema deve gerar um registro de pagamento pendente para cada agendamento. \\ \hline
	RF03-03 & O sistema deve disponibilizar interface para registro de pagamentos apenas em terminais do salão. \\ \hline
	RF03-04 & O sistema deve permitir o registro de diferentes formas de pagamento presencial. \\ \hline
	
	% RN04
	\multicolumn{2}{|l|}{\textbf{RN04}} \\ \hline
	RF04-01 & O sistema deve solicitar avaliação após cada serviço concluído. \\ \hline
	RF04-02 & O sistema deve enviar notificação de avaliação pendente após o atendimento. \\ \hline
	RF04-03 & O sistema deve permitir a inclusão de comentários textuais nas avaliações. \\ \hline
	RF04-04 & O sistema deve compilar e exibir estatísticas de avaliações para gerentes e profissionais. \\ \hline
	
	% RN05
	\multicolumn{2}{|l|}{\textbf{RN05}} \\ \hline
	RF05-01 & O sistema deve verificar a disponibilidade do profissional antes de confirmar um agendamento. \\ \hline
	RF05-02 & O sistema deve impedir o agendamento quando houver conflito de horários para o mesmo profissional. \\ \hline
	RF05-03 & O sistema deve exibir alertas visuais quando tentativas de sobreposição forem detectadas. \\ \hline
	RF05-04 & O sistema deve considerar o tempo de duração de cada serviço para evitar sobreposições parciais. \\ \hline
	
	% RN06
	\multicolumn{2}{|l|}{\textbf{RN06}} \\ \hline
	RF06-01 & O sistema deve permitir à gerente definir o tempo mínimo de antecedência. \\ \hline
	RF06-02 & O sistema deve validar a antecedência mínima para todos os agendamentos. \\ \hline
	RF06-03 & O sistema deve bloquear tentativas de agendamento inferiores ao tempo de antecedência definido. \\ \hline
	RF06-04 & O sistema deve exibir mensagem informativa sobre a política de antecedência mínima. \\ \hline
	RF06-05 & O sistema deve calcular automaticamente os horários disponíveis considerando a antecedência mínima. \\ \hline
	
	% RN07
	\multicolumn{2}{|l|}{\textbf{RN07}} \\ \hline
	RF07-01 & O sistema deve restringir as operações de cadastro de funcionárias ao perfil de gerente. \\ \hline
	RF07-02 & O sistema deve restringir as operações de edição de dados de funcionárias ao perfil de gerente. \\ \hline
	RF07-03 & O sistema deve disponibilizar interface de gerenciamento de funcionárias apenas para usuários com perfil de gerente. \\ \hline
	RF07-04 & O sistema deve registrar \emph{log} de todas as operações de cadastro e edição de funcionárias. \\ \hline
	RF07-05 & O sistema deve notificar funcionárias quando seus dados forem alterados pela gerente. \\ \hline
	
	% RN08
	\multicolumn{2}{|l|}{\textbf{RN08}} \\ \hline
	RF08-01 & O sistema deve disponibilizar interface para profissionais definirem seus horários de trabalho. \\ \hline
	RF08-02 & O sistema deve permitir a configuração de horários diferentes para cada dia da semana. \\ \hline
	RF08-03 & O sistema deve validar os horários definidos conforme o horário de funcionamento do salão. \\ \hline
	RF08-04 & O sistema deve atualizar automaticamente a agenda de disponibilidade. \\ \hline
	
	% RN09
	\multicolumn{2}{|l|}{\textbf{RN09}} \\ \hline
	RF09-01 & O sistema deve permitir que profissionais bloqueiem intervalos específicos em sua agenda. \\ \hline
	RF09-02 & O sistema deve possibilitar a definição de motivo para o bloqueio (opcional). \\ \hline
	RF09-03 & O sistema deve impedir novos agendamentos nos horários bloqueados. \\ \hline
	RF09-04 & O sistema deve permitir o desbloqueio de horários previamente bloqueados. \\ \hline
	RF09-05 & O sistema deve exibir visualmente os horários bloqueados na interface de agendamento. \\ \hline
	
	% RN10
	\multicolumn{2}{|l|}{\textbf{RN10}} \\ \hline
	RF10-01 & O sistema deve permitir configuração de preços e duração de serviços conforme o modelo de trabalho da profissional. \\ \hline
	RF10-02 & O sistema deve restringir edição de preços e duração às profissionais autônomas ou à gerente, conforme o caso. \\ \hline
	RF10-03 & O sistema deve calcular automaticamente comissões para profissionais do modelo comissionado. \\ \hline
	RF10-04 & O sistema deve guardar histórico de alterações de preços e duração de serviços. \\ \hline
	
	% RN11
	\multicolumn{2}{|l|}{\textbf{RN11}} \\ \hline
	RF11-01 & O sistema deve permitir à gerente configurar os horários de funcionamento do salão para cada dia da semana. \\ \hline
	RF11-02 & O sistema deve impedir agendamentos fora do horário de funcionamento. \\ \hline
	RF11-03 & O sistema deve exibir os horários de funcionamento para clientes durante o processo de agendamento. \\ \hline
	RF11-04 & O sistema deve permitir configuração de horários especiais para datas comemorativas ou feriados. \\ \hline
	RF11-05 & O sistema deve adaptar a visualização da agenda conforme os horários de funcionamento. \\ \hline
	
	% RN12
	\multicolumn{2}{|l|}{\textbf{RN12}} \\ \hline
	RF12-01 & O sistema deve solicitar informação sobre como o cliente conheceu o salão durante o cadastro inicial. \\ \hline
	RF12-02 & O sistema deve oferecer opções predefinidas (redes sociais, indicação, busca \emph{on-line}, etc.). \\ \hline
	RF12-03 & O sistema deve permitir especificar o nome do indicador quando a opção for "indicação”. \\ \hline
	RF12-04 & O sistema deve gerar relatórios estatísticos sobre as fontes de captação de clientes. \\ \hline
	RF12-05 & O sistema deve permitir atualização dessa informação em caso de erro ou mudança. \\ \hline
	
	% RN13
	\multicolumn{2}{|l|}{\textbf{RN13}} \\ \hline
	RF13-01 & O sistema deve registrar e contabilizar indicações feitas por cada cliente. \\ \hline
	RF13-02 & O sistema deve confirmar que novos clientes indicados realizaram ao menos um serviço pago. \\ \hline
	RF13-03 & O sistema deve notificar automaticamente quando um cliente atingir 3 indicações confirmadas. \\ \hline
	RF13-04 & O sistema deve gerar um voucher/crédito para procedimento gratuito. \\ \hline
	RF13-05 & O sistema deve permitir a seleção e aplicação do procedimento gratuito durante o agendamento. \\ \hline
	RF13-06 & O sistema deve reiniciar a contagem de indicações após o uso do benefício. \\ \hline
		
	% RN14
	\multicolumn{2}{|l|}{\textbf{RN14}} \\ \hline
	RF14-01 & O sistema deve identificar automaticamente clientes aniversariantes do dia/semana/mês. \\ \hline
	RF14-02 & O sistema deve enviar automaticamente mensagem de aniversário na data correspondente. \\ \hline
	RF14-03 & O sistema deve gerar cupom de desconto para uso durante o mês de aniversário. \\ \hline
	RF14-04 & O sistema deve permitir a aplicação do desconto de aniversário durante o agendamento no mês correspondente. \\ \hline
	RF14-05 & O sistema deve registrar o uso do benefício para evitar duplicidades. \\ \hline
	
	% RN15
	\multicolumn{2}{|l|}{\textbf{RN15}} \\ \hline
	RF15-01 & O sistema deve enviar confirmação de agendamento para o cliente e profissional envolvidos. \\ \hline
	RF15-02 & O sistema deve enviar lembretes 24h antes do horário agendado. \\ \hline
	RF15-03 & O sistema deve notificar sobre cancelamentos ou alterações nos agendamentos. \\ \hline
	RF15-04 & O sistema deve permitir configuração dos tipos de notificação desejados (\emph{e-mail}, SMS, push). \\ \hline
	
	% RN16
	\multicolumn{2}{|l|}{\textbf{RN16}} \\ \hline
	RF16-01 & O sistema deve validar a unicidade de \emph{e-mails} durante o cadastro de usuários. \\ \hline
	RF16-02 & O sistema deve validar a unicidade de telefones durante o cadastro de usuários. \\ \hline
	RF16-03 & O sistema deve exibir alerta quando houver tentativa de cadastro com \emph{e-mail}/telefone já existente. \\ \hline
	RF16-04 & O sistema deve permitir a atualização de \emph{e-mail}/telefone com validação de unicidade. \\ \hline
	
	% RN17
	\multicolumn{2}{|l|}{\textbf{RN17}} \\ \hline
	RF17-01 & O sistema deve solicitar o tempo de duração durante o cadastro de cada serviço. \\ \hline
	RF17-02 & O sistema deve utilizar a duração para cálculo de disponibilidade na agenda. \\ \hline
	RF17-03 & O sistema deve permitir duração personalizada por profissional para o mesmo tipo de serviço. \\ \hline
	RF17-04 & O sistema deve considerar a duração para evitar sobreposições de agendamentos. \\ \hline
	RF17-05 & O sistema deve exibir o tempo de duração estimado durante o processo de agendamento. \\ \hline
	
	% RN18
	\multicolumn{2}{|l|}{\textbf{RN18}} \\ \hline
	RF18-01 & O sistema deve validar se a data e hora de agendamento são futuras em relação ao momento da operação. \\ \hline
	RF18-02 & O sistema deve bloquear e exibir mensagem de erro para tentativas de agendamento em datas/horários passados. \\ \hline
	RF18-03 & O sistema deve atualizar automaticamente os horários disponíveis conforme o avanço do tempo. \\ \hline
	RF18-04 & O sistema deve considerar o fuso horário do salão para validação das datas e horários. \\ \hline
	RF18-05 & O sistema deve desabilitar visualmente na interface os dias e horários que já passaram. \\ \hline
	
	% RN19
	\multicolumn{2}{|l|}{\textbf{RN19}} \\ \hline
	RF19-01 & O sistema deve permitir a vinculação de serviços a profissionais específicas durante o cadastro do serviço. \\ \hline
	RF19-02 & O sistema deve exibir apenas profissionais habilitadas para cada serviço durante o agendamento. \\ \hline
	RF19-03 & O sistema deve permitir que uma profissional seja vinculada a múltiplos serviços. \\ \hline
	RF19-04 & O sistema deve permitir à gerente atualizar as vinculações entre serviços e profissionais. \\ \hline
	RF19-05 & O sistema deve notificar profissionais quando forem vinculadas a novos serviços. \\ \hline
	
	% RN20
	\multicolumn{2}{|l|}{\textbf{RN20}} \\ \hline
	RF20-01 & O sistema deve gerar relatórios de produtividade por profissional (serviços realizados, valores, avaliações). \\ \hline
	RF20-02 & O sistema deve gerar relatórios financeiros (receita diária, semanal, mensal) para a gerência. \\ \hline
	RF20-03 & O sistema deve gerar relatórios de ocupação (horários mais procurados, serviços mais solicitados). \\ \hline
	RF20-04 & O sistema deve permitir a exportação dos relatórios em formatos diversos (\gls{pdf}, Excel, \gls{csv}). \\ \hline
	RF20-05 & O sistema deve gerar \emph{dashboard} com indicadores-chave de desempenho para gerência e profissionais. \\ \hline
	
	% RN21
	\multicolumn{2}{|l|}{\textbf{RN21}} \\ \hline
	RF21-01 & O sistema deve armazenar permanentemente todos os agendamentos realizados e suas informações. \\ \hline
	RF21-02 & O sistema deve manter registro completo de todas as avaliações de serviços feitas pelos clientes. \\ \hline
	RF21-03 & O sistema deve permitir consulta ao histórico de serviços e avaliações por cliente. \\ \hline
	RF21-04 & O sistema deve possibilitar consulta ao histórico de serviços realizados por profissional. \\ \hline
	RF21-05 & O sistema deve implementar política de retenção de dados conforme legislação vigente (LGPD). \\ \hline
	
	% RN22
	\multicolumn{2}{|l|}{\textbf{RN22}} \\ \hline
	RF22-01 & O sistema deve oferecer duas vias principais de agendamento: por serviço ou por profissional. \\ \hline
	RF22-02 & O sistema deve exibir lista de profissionais disponíveis após a seleção do serviço. \\ \hline
	RF22-03 & O sistema deve exibir lista de serviços disponíveis após a seleção da profissional. \\ \hline
	RF22-04 & O sistema deve permitir filtragem por data/horário em ambos os fluxos de agendamento. \\ \hline
	RF22-05 & O sistema deve exibir informações sobre a profissional (experiência, avaliações) durante o agendamento por serviço. \\ \hline
	RF22-06 & O sistema deve exibir informações sobre o serviço (duração, preço, descrição) durante o agendamento por profissional. \\ \hline
	
	% RN23
	\multicolumn{2}{|l|}{\textbf{RN23}} \\ \hline
	RF23-01 & O sistema deve calcular automaticamente a disponibilidade da profissional com base na duração dos serviços agendados. \\ \hline
	RF23-02 & O sistema deve bloquear horários subsequentes conforme a duração do serviço agendado. \\ \hline
	RF23-03 & O sistema deve considerar intervalos entre atendimentos quando configurados. \\ \hline
	RF23-04 & O sistema deve atualizar a agenda em tempo real conforme novos agendamentos são realizados. \\ \hline
	RF23-05 & O sistema deve permitir visualização da taxa de ocupação diária da profissional. \\ \hline
	
	% RN24
	\multicolumn{2}{|l|}{\textbf{RN24}} \\ \hline
	RF24-01 & O sistema deve permitir cadastro e categorização dos serviços por tipo (depilação, unhas, estética corporal, etc.). \\ \hline
	RF24-02 & O sistema deve possibilitar a organização hierárquica dos serviços (categorias e subcategorias). \\ \hline
	RF24-03 & O sistema deve exibir serviços agrupados por categoria na interface de agendamento. \\ \hline
	RF24-04 & O sistema deve permitir busca de serviços por nome ou categoria. \\ \hline
	RF24-05 & O sistema deve exibir informações detalhadas sobre cada serviço (descrição, duração, valor). \\ \hline
	RF24-06 & O sistema deve permitir adição de imagens ilustrativas para cada categoria de serviço. \\ \hline
	
	% RN25
	\multicolumn{2}{|l|}{\textbf{RN25}} \\ \hline
	RF25-01 & O sistema deve disponibilizar interface para profissionais cadastrarem novos serviços. \\ \hline
	RF25-02 & O sistema deve implementar fluxo de aprovação pela gerência para novos serviços criados. \\ \hline
	RF25-03 & O sistema deve permitir definição de parâmetros específicos para serviços personalizados (preço, duração, descrição). \\ \hline
	RF25-04 & O sistema deve restringir a visualização e agendamento dos serviços personalizados conforme configuração da profissional. \\ \hline
	RF25-05 & O sistema deve notificar clientes elegíveis sobre novos serviços personalizados quando configurado. \\ \hline
	RF25-06 & O sistema deve fazer com que novos serviços sejam criados com o status de "Pendente de Aprovação”. \\ \hline
	RF25-07 & O sistema deve impedir que serviços com status "Pendente" sejam disponibilizados para agendamento. \\ \hline
	RF25-08 & O sistema deve gerar automaticamente uma solicitação de aprovação quando um novo serviço for criado. \\ \hline
	
	% RN26
	\multicolumn{2}{|l|}{\textbf{RN26}} \\ \hline
	RF26-01 & O sistema deve notificar a gerente imediatamente quando um novo serviço for submetido para aprovação. \\ \hline
	RF26-02 & O sistema deve notificar a profissional sobre o status da avaliação (aprovado, rejeitado, ou solicitação de alterações). \\ \hline
	RF26-03 & O sistema deve permitir que a gerente inclua comentários justificando aprovação ou rejeição. \\ \hline
	RF26-04 & O sistema deve disponibilizar uma área exclusiva para a gerente visualizar todos os serviços pendentes de aprovação. \\ \hline
	RF26-05 & O sistema deve exibir todas as informações do serviço proposto de forma clara e organizada. \\ \hline
	RF26-06 & O sistema deve permitir que a gerente aprove, rejeite ou solicite modificações nos serviços. \\ \hline
	RF26-07 & O sistema deve exibir informações sobre a profissional que criou o serviço. \\ \hline
	RF26-08 & O sistema deve impedir que profissionais modifiquem serviços que estão em processo de aprovação. \\ \hline
	RF26-09 & O sistema deve manter os seguintes status para serviços: "Rascunho", "Pendente de Aprovação", "Aprovado", "Rejeitado", "Requer Modificações”. \\ \hline
	RF26-10 & O sistema deve permitir que serviços rejeitados sejam editados e ressubmetidos para nova avaliação. \\ \hline
	RF26-11 & O sistema deve automaticamente disponibilizar serviços aprovados na lista de agendamento. \\ \hline
	

\end{longtable}



\subsection{Requisitos Não Funcionais}

A seguir, são apresentados os requisitos não funcionais, relativos à qualidade e à experiência do sistema, garantindo aspectos essenciais como usabilidade, segurança e design.

\begin{longtable}{|p{0.1\textwidth}|p{0.85\textwidth}|}
	\caption{Requisitos Não Funcionais} \label{tab:requisitos-nao-funcionais}\\
	\hline
	\textbf{ID} & \textbf{Descrição do Requisito Não Funcional} \\
	\hline
	\endfirsthead
	
	\hline
	\textbf{ID} & \textbf{Descrição do Requisito Não Funcional} \\
	\hline
	\endhead
	
	RNF01 & O sistema deve ser intuitivo e fácil de usar. \\ \hline
	RNF02 & O sistema deve estar constantemente disponível. \\ \hline
	RNF03 & O sistema deve proporcionar segurança aos usuários. \\ \hline
	RNF04 & O sistema deve ser responsivo, adaptando-se a diferentes dispositivos. \\ \hline
	RNF05 & O sistema deve ter um design simples, minimalista e chique. \\ \hline
	RNF06 & O sistema deve ter processo de cadastro simplificado. \\ \hline
\end{longtable}

