\section{Escopo do Projeto}


Esta seção apresenta o escopo do sistema desenvolvido, com o objetivo de delimitar suas funcionalidades, comportamentos esperados e restrições de operação. As definições aqui descritas foram elaboradas com base em reuniões realizadas com a gestora do salão de beleza, nas quais foram discutidas as necessidades do negócio e as principais demandas da operação cotidiana.

Com base nessas interações, foram estabelecidas as regras de negócio que orientam o funcionamento do sistema, os requisitos funcionais que especificam os serviços a serem oferecidos aos usuários e os requisitos não funcionais que tratam de aspectos como desempenho, segurança e usabilidade.

A definição clara desses elementos é essencial para garantir a coerência do sistema com as necessidades dos usuários e para orientar a equipe de desenvolvimento ao longo das etapas do projeto.

\subsection{Regras de Negócio}

As regras de negócio definem os comportamentos, restrições e condições que regem o funcionamento do sistema, conforme as particularidades da BS Beauty Academy, e representam diretrizes que devem ser respeitadas tanto no uso da aplicação quanto no desenvolvimento das funcionalidades.

A seguir, são listadas as principais regras de negócio identificadas:

\begin{longtable}{|p{1.2cm}|p{10cm}|p{4cm}|}
	\hline
	\textbf{ID} & \textbf{Descrição da Regra de Negócio} & \textbf{Área Impactada} \\
	\hline
	\endfirsthead
	
	\hline
	\textbf{ID} & \textbf{Descrição da Regra de Negócio} & \textbf{Área Impactada} \\
	\hline
	\endhead
	
	RN01 & Não há políticas ou penalidades referentes a cancelamentos ou atrasos. & Agenda \\
	\hline
	RN02 & Clientes podem agendar múltiplos serviços em um mesmo agendamento. & Agenda, Clientes \\
	\hline
	RN03 & Pagamentos são realizados só presencialmente. & Atendimento, Financeiro, Marketing e Vendas \\
	\hline
	RN04 & Clientes devem avaliar os serviços concluídos. & Atendimento, Clientes \\
	\hline
	RN05 & Não deve haver sobreposição de horários em agendamentos. & Agenda, Atendimento \\
	\hline
	RN06 & Agendamentos devem ser feitos com, no mínimo, 3 horas de antecedência. & Agenda \\
	\hline
	RN07 & O cadastro e edição de funcionárias é feito exclusivamente pela gerente. & RH/Profissionais \\
	\hline
	RN08 & Profissionais podem definir seus próprios horários de trabalho. & Agenda, RH/Profissionais \\
	\hline
	RN09 & Profissionais podem bloquear horários específicos na agenda. & Agenda, RH/Profissionais \\
	\hline
	RN10 & Preços e duração de serviços podem ser definidos pela profissional (autônoma) ou pela gerente (modelo comissionado). & Financeiro, Marketing e Vendas, RH/Profissionais \\
	\hline
	RN11 & O salão possui horário fixo de funcionamento (definido pela gerente). & Atendimento, RH/Profissionais \\
	\hline
	RN12 & Clientes devem informar como conheceram o salão. & Atendimento, Clientes \\
	\hline
	RN13 & Programa de indicação para indicar 3 pessoas e ganhar um procedimento grátis. & Clientes, Financeiro, Marketing e Vendas \\
	\hline
	RN14 & Clientes aniversariantes devem receber mensagens de aniversário e descontos. & Clientes, Marketing e Vendas \\
	\hline
	RN15 & Clientes, profissionais e a gerente devem receber notificações referentes a agendamentos. & Agenda, Atendimento, Clientes, RH/Profissionais \\
	\hline
	RN16 & E-mails e telefones cadastrados por clientes e profissionais devem ser únicos. & Clientes, RH/Profissionais \\
	\hline
	RN17 & Cada serviço possui um tempo de duração definido ao cadastrá-lo. & Agenda, Atendimento \\
	\hline
	RN18 & Não é possível agendar serviços para datas passadas. & Agenda, Atendimento \\
	\hline
	RN19 & Serviços devem estar vinculados a uma ou mais profissionais especializadas. & Atendimento, RH/Profissionais \\
	\hline
	RN20 & Gerente e profissionais devem receber relatórios diversos. & RH/Profissionais \\
	\hline
	RN21 & O histórico de avaliações e agendamentos deve ser mantido. & Atendimento \\
	\hline
	RN22 & Clientes devem agendar com base no serviço ou profissional. & Agenda, Clientes \\
	\hline
	RN23 & A quantidade de clientes atendidos por uma profissional depende do tempo de cada procedimento. & Agenda, Atendimento, RH/Profissionais \\
	\hline
	RN24 & Atualmente, o salão oferece serviços de depilação, unhas, estética corporal e facial, sobrancelhas, remoção e cílios. & Agenda, Atendimento \\
	\hline
	RN25 & Profissionais podem criar seus próprios serviços (com limitações de acesso). & Agenda, RH/Profissionais \\
	\hline
	
\end{longtable}

\subsection{Requisitos Funcionais}
\subsection{Requisitos Não Funcionais}