\section{Testes e Manutenibilidade}
Nesta seção, apresenta-se os mecanismos e ferramentas adotados para garantir a qualidade de software do projeto ao longo do desenvolvimento.
Será abordado o plano de testes, assim como cada teste que está incluso. Além disso, assuntos como infraestrutura de testes e convenções de código (\textit{coding convention})
serão detalhados, evidenciando práticas que promovem a manutenibilidade, padronização e confiabilidade do sistema ao longo de sua evolução.

\subsection{Plano de Testes}
O plano de testes define a estratégia adotada para garantir a qualidade e confiabilidade da aplicação. 
Ele inclui os tipos de testes aplicados, as ferramentas utilizadas, o escopo das validações, e os critérios de sucesso e falha.

Tendo em vista que a arquitetura do \textit{back-end} é constituída por \textit{controllers}, \textit{services} e \textit{repositories} usando o \emph{framework} \textit{Express}, 
é necessário garantir um ótimo funcionamento da comunicação entre essas camadas. Portanto, conforme os recursos da \textit{RESTful \gls{api}} são desenvolvidos (agendamentos, serviços de beleza, clientes, funcionários, etc) com as respectivas camadas que foram comentadas, urge-se a demanda de serem testadas em paralelo, cobrindo os cenários possíveis de sucesso/falha. 

Pretende-se atingir, no mínimo, 80\% de cobertura nos testes unitários e integrados aplicados sobre as camadas de \textit{controllers} e \textit{services} da \gls{api}. Essa métrica será obtida por meio de ferramentas integradas ao ambiente de testes, como o \textit{framework} \textbf{\textit{Vitest}}~\cite{vitest-2025} com suporte a geração de relatórios. Embora a cobertura de testes não garanta por si só a ausência de falhas, ela serve como um forte indicador de que a maior parte da lógica de negócio está sendo exercitada e validada durante a execução dos testes. Essa prática contribui diretamente para a robustez do sistema e facilita a detecção precoce de regressões ao longo do desenvolvimento.

O mesmo propósito de cobertura de testes é válido para o \textit{front-end} desenvolvido em \textit{React}. Como essa tecnologia adota um princípio de componentização de elementos da interface,
é interessante que as páginas com cenários lógicos mais críticos (como as telas de agendamento) sejam validadas de forma precisa. Assim, garantindo que os elementos da interface estejam atendendo o comportamento esperado.

Quanto à cobertura de testes do \textit{front-end}, é tido como propósito, realizar um plano que envolva todos os processos que foram mapeados no escopo do projeto.

Ademais, os testes desenvolvidos nas duas divisões do sistema terão foco, inicialmente, nas funcionalidades essenciais para o funcionamento do sistema, tais como autenticação, criação e gerenciamento de agendamentos, relatórios, serviços e ofertas. Essa abordagem inicial permite validar as principais regras de negócio desde as primeiras entregas, contribuindo para uma base de código mais robusta e confiável à medida que novas funcionalidades forem sendo integradas.

Os arquivos contendo as classes/funções de testes devem estar localizados em diretórios específicos, adotando uma nomenclatura compreensível como 
\texttt{analytics.\allowbreak controller.\allowbreak spec.\allowbreak unit.ts} (referenciando um teste unitário) e \texttt{analytics.controller.spec.}%
\linebreak%
{\ttfamily integration.ts} referenciando um teste integrado).

\subsection{Análise Estáticas}
A análise estática de código é realizada utilizando as ferramentas \textbf{\textit{SonarQube}}~\cite{sonarqube-2025} e \textbf{\textit{ESLint}}~\cite{eslint-2025}, 
que permitem detectar problemas de qualidade, como vulnerabilidades, \textit{bugs} e código duplicado, sem a necessidade de executar a aplicação. Essa etapa ajuda a manter o código limpo, seguro e sustentável ao longo do tempo.

\subsection{Testes funcionais}
Os testes funcionais baseiam-se nas especificações dos requisitos do sistema. O principal propósito, portanto, é validar se uma determinada funcionalidade solicitada foi desenvolvida com êxito, e está atendendo as expectativas esperadas dentro do sistema. Dessa forma, serão abordados nessa subseção os testes unitários, de integração e \textit{end-to-end}. 

\subsubsection{Testes Unitários}
Os testes unitários servem para validar o comportamento de funções e componentes isolados utilizando \textit{mocks} de outras partes que interagem com o que está sendo testado.

Os arquivos de testes unitários são organizados em diretórios específicos com uma nomenclatura padronizada. Um exemplo de nome de arquivo seria \texttt{appointments.\allowbreak controller.\allowbreak spec.\allowbreak unit.ts}, indicando que se trata de um teste unitário do módulo de agendamentos. Essa convenção visa facilitar a identificação, localização e manutenção dos testes ao longo do tempo.

\subsubsection{Testes de Componente}
Os testes de componente têm como objetivo validar o comportamento isolado de componentes funcionais da aplicação, garantindo que cada unidade seja capaz de cumprir sua função específica conforme o esperado. No contexto do projeto, esse tipo de teste está especialmente presente no \textit{front-end}, desenvolvido em \textit{React}, onde a arquitetura baseada em componentes torna natural a aplicação dessa abordagem.

Assim como os demais testes, os arquivos que contêm testes de componentes seguem uma convenção de nomenclatura voltada à organização e fácil localização no projeto. Um exemplo seria \texttt{bookingForm.\allowbreak spec.\allowbreak component.\allowbreak tsx}, indicando que se trata de um teste de componente da interface de agendamento. Essa padronização é importante para manter a coesão do código de testes e permitir sua escalabilidade conforme o sistema cresce.

\subsubsection{Testes de Integração}
Os testes de integração verificam a interação entre módulos e componentes da aplicação de acordo com uma determinada funcionalidade que precisa ser desenvolvida.

Assim como os testes unitários, os testes integrados seguem uma convenção de nomenclatura para manter a organização do projeto. Um exemplo de nome de arquivo seria \texttt{services.\allowbreak controller.\allowbreak spec.\allowbreak integration.ts} que sinaliza tratar-se de um arquivo de testes vinculado ao módulo de serviços. Essa padronização contribui para a clareza estrutural do projeto de testes.


\subsubsection{Testes \textit{end-to-end}}
Os testes \textit{end-to-end} representam provações feitas através de cenários que validam o sistema completamente desde a interface gráfica até a lógica interna da \gls{api}.

No atual momento do desenvolvimento, esse tipo de teste será usado para validar, especialmente, os processos levantados nas premissas do projeto, devido à alta complexidade de configuração e lentidão de execução que possui.


\subsection{Testes não funcionais}
Testes não funcionais estão fora do escopo do projeto no atual desenvolvimento.
\subsubsection{Testes de performance}
Testes de performance estão fora do escopo do projeto no atual desenvolvimento.
\subsubsection{Testes de carga}
Testes de carga estão fora do escopo do projeto no atual desenvolvimento.
\subsubsection{Testes de configuração}
Testes de configuração estão fora do escopo do projeto no atual desenvolvimento.

\subsection{Testes automatizados}
Os testes automatizados desempenham um papel fundamental na garantia da qualidade do sistema ao longo de seu ciclo de vida. Eles são responsáveis por verificar continuamente o correto funcionamento das funcionalidades implementadas, promovendo confiança no código e permitindo a identificação precoce de regressões.

A execução dos testes automatizados está ligada ao fluxo de integração contínua por meio da ferramenta \textbf{\textit{GitHub Actions}}~\cite{githubactions-2025}. Essa integração visa garantir que o código entregue atenda a padrões mínimos de qualidade e estabilidade em todas as etapas do desenvolvimento.  A cada \textit{push} ou \textit{pull request}, fluxos automatizados são executados para validar o código por meio de testes automatizados, análise estática, e verificação de cobertura. Esse processo assegura que apenas alterações estáveis e em conformidade com os padrões de qualidade sejam incorporadas à base de código principal, promovendo entregas seguras e contínuas ao longo do ciclo de desenvolvimento.

\subsection{\textit{Logs}}

\subsection{\textit{Code Convention}}
Para garantir a legibilidade e padronização do código, são adotadas convenções definidas com base em boas práticas da comunidade \textit{JavaScript}/\textit{TypeScript}.
Essas diretrizes ajudam a manter o código uniforme entre os diferentes desenvolvedores do time, reduzindo ambiguidades e facilitando o entendimento do sistema como um todo.

As principais práticas adotadas incluem:

\begin{itemize}
  \item \textbf{Ferramentas de \textit{Linting} e Formatação:}
  \begin{itemize}
    \item Utilização do \textbf{\textit{ESLint}}~\cite{eslint-2025} para garantir padrões de estilo e detectar possíveis erros ou práticas inadequadas de codificação.
    \item Uso do \textbf{\textit{Prettier}}~\cite{prettier-2025} para formatação automática do código, assegurando que todos os arquivos mantenham a mesma estrutura visual (espaçamento, quebras de linha, indentação, etc).
  \end{itemize}

  \item \textbf{Padrões de Nomenclatura:}
  \begin{itemize}
    \item Uso de \texttt{camelCase} para variáveis e funções.
    \item Uso de \texttt{PascalCase} para componentes e classes.
    \item Uso de \texttt{UPPER\_SNAKE\_CASE} para constantes globais.
  \end{itemize}

  \item \textbf{Organização do Código:}
  \begin{itemize}
    \item Estrutura modular com separação clara entre camadas (\textit{controllers}, \textit{services}, \textit{repositories}).
    \item Agrupamento de arquivos por domínio funcional.
  \end{itemize}

  \item \textbf{Boas Práticas:}
  \begin{itemize}
    \item Escrita de código limpo e legível, evitando duplicações.
    \item Utilização de comentários apenas quando necessário, priorizando nomes autoexplicativos.
  \end{itemize}

  \item \textbf{Revisões e Padronização em Equipe:}
  \begin{itemize}
    \item Adoção de \textit{pull requests} com as devidas descrições das funcionalidades desenvolvidas.
    \item Documentação e comunicação clara de decisões técnicas relevantes.
  \end{itemize}
\end{itemize}
