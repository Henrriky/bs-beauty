\section{Testes e Manutenibilidade}
Nesta seção, apresenta-se os mecanismos e ferramentas adotados para garantir a qualidade de software do projeto ao longo do desenvolvimento.
Será abordado o plano de testes, assim como cada teste que está incluso. Além disso, assuntos como infraestrutura de testes e convenções de código (\textit{coding convention})
serão detalhados, evidenciando práticas que promovem a manutenibilidade, padronização e confiabilidade do sistema ao longo de sua evolução.

\subsection{Plano de Testes}
Tendo em vista que a arquitetura do \textit{back-end} é constituída por \textit{controllers}, \textit{services} e \textit{repositories} usando o \emph{framework} \textit{Express}, 
é necessário garantir um ótimo funcionamento da comunicação entre essas camadas. Portanto, conforme os recursos da \textit{RESTful \gls{api}} são desenvolvidos (agendamentos, serviços de beleza, clientes, profissionais etc) com as respectivas camadas que foram comentadas, urge-se a demanda de serem testadas em paralelo, cobrindo os cenários possíveis de sucesso/falha. 

Pretende-se atingir, no mínimo, 70\% de cobertura nos testes unitários aplicados sobre as camadas de \textit{controllers} e \textit{services} da \gls{api}. 

As métricas de testes para o \textit{back-end} são obtidas por meio de ferramentas integradas ao ambiente de testes, como o \textit{framework} \textbf{\textit{Vitest}}~\cite{vitest-2025} com suporte a geração de relatórios. Embora a cobertura de testes não garanta por si só a ausência de falhas, ela serve como um forte indicador de que a maior parte da lógica de negócio está sendo exercitada e validada durante a execução dos testes. Essa prática contribui diretamente para a robustez do sistema e facilita a detecção precoce de regressões ao longo do desenvolvimento.

Quanto aos testes do \textit{front-end}, o foco principal do plano é realizar testes de usabilidade tanto em dispositivos de menor resolução (celulares e tablets) quanto dispositivos grandes (desktops e notebooks). 

Ademais, os testes desenvolvidos nas duas divisões do sistema terão foco, inicialmente, nas funcionalidades essenciais para o funcionamento do sistema, tais como autenticação, criação e gerenciamento de agendamentos, relatórios, serviços e ofertas. Essa abordagem inicial permite validar as principais regras de negócio desde as primeiras entregas, contribuindo para uma base de código mais robusta e confiável à medida que novas funcionalidades forem sendo integradas.

\begin{quadro}[htbp]
	\centering
	\caption{Planejamento de Testes por Fase Funcional}
	\label{quad:planejamento-testes}
	\begin{tabularx}{\textwidth}{|p{5cm}|X|X|}
		\hline
		\textbf{Fase} & 
		\textbf{Funcionalidades Prioritárias} & 
		\textbf{Tipos de Testes Aplicados} \\
		\hline
		Fase 1 – Preparação da Esteira CI/CD e Ambiente de Testes & 
		Configuração iniciais para a varredura de todos os tipos de testes que serão realizados &
		Todos os testes automatizados (unitários e integrados)
		\\
		\hline
		Fase 2 – Funcionalidades Essenciais & 
		Autenticação e Agendamentos & 
		Testes unitários e de integração (back-end e front-end) \\
		\hline
		Fase 3 – Monitoramento de qualidade e Medidas de comunicação & 
		Relatórios, Avaliações e Notificações & 
		Testes unitários e de integração (back-end e front-end) \\
		\hline
		Fase 4 – Dados Operacionais & 
		Gerenciamento de Funcionários, Serviços e Ofertas & 
		Testes unitários e de integração (back-end e front-end) \\
		\hline
		Fase 5 – Estabilização e Qualidade Final & 
		Reexecução dos testes automatizados, análise de cobertura, correções finais & 
		Todos os tipos de teste relacionados ao projeto \\
		\hline
	\end{tabularx}
	\fonte{Produzido pelos autores}
\end{quadro}


\subsection{Análise Estática}
A análise estática de código é realizada utilizando as ferramentas \textbf{\textit{SonarQube}}~\cite{sonarqube-2025} e \textbf{\textit{ESLint}}~\cite{eslint-2025}, 
que permitem detectar problemas de qualidade, como vulnerabilidades, \textit{bugs} e código duplicado, sem a necessidade de executar a aplicação. Essa etapa ajuda a manter o código limpo, seguro e sustentável ao longo do tempo.

\subsection{Testes funcionais}
Os testes funcionais baseiam-se nas especificações dos requisitos do sistema. O principal propósito, portanto, é validar se uma determinada funcionalidade solicitada foi desenvolvida com êxito, e está atendendo as expectativas esperadas dentro do sistema. Dessa forma, serão abordados nessa subseção os testes unitários e de integração.

\subsubsection{Testes Unitários}
Os arquivos de testes unitários são organizados em diretórios específicos com uma nomenclatura padronizada. Um exemplo de nome de arquivo seria \texttt{appointments.\allowbreak controller.\allowbreak spec.ts}, indicando que se trata de um teste unitário do módulo de agendamentos. Essa convenção visa facilitar a identificação, localização e manutenção dos testes ao longo do tempo.

\subsubsection{Testes de Componente}
Os testes de componente não serão desenvolvidos neste estágio do projeto, 
tendo em vista a complexidade envolvida na configuração de ambientes de teste para o \textit{front-end} e o tempo reduzido disponível no cronograma de desenvolvimento.
Além disso, a cobertura pretendida pelos testes unitários e de integração já contempla a maior parte das interações entre os módulos do sistema, 
tornando desnecessário, neste momento, o esforço adicional de implementação de testes específicos de componente. 
Esses testes poderão ser incorporados futuramente, conforme o sistema amadureça e novas versões do \textit{front-end} sejam disponibilizadas.

\subsubsection{Testes de Integração}
Assim como os testes unitários, os testes integrados seguem uma convenção de nomenclatura para manter a organização do projeto. Um exemplo de nome de arquivo seria \texttt{services.\allowbreak integration.\allowbreak spec.ts} que sinaliza tratar-se de um arquivo de testes vinculado ao módulo de serviços. Essa padronização contribui para a clareza estrutural do projeto de testes.


\subsubsection{Testes \textit{end-to-end}}
Os testes \textit{end-to-end} não serão desenvolvidos no momento inicial do projeto devido ao alto custo de configuração e manutenção dessa categoria de testes, 
que exige a simulação completa do ambiente e da integração entre todas as camadas do sistema. 
Dada a fase atual do desenvolvimento, o foco está em garantir a estabilidade das funcionalidades essenciais por meio de testes unitários e de integração, 
que já permitem validar os principais fluxos do sistema com menor complexidade. 
A implementação de testes \textit{end-to-end} está prevista para fases futuras, quando o produto atingir maior estabilidade e o ciclo de entregas estiver consolidado.

\subsection{Testes não funcionais}

\subsubsection{Testes de performance}
Testes de performance estão fora do escopo do projeto neste momento em razão da ausência de requisitos formais de desempenho e da necessidade de concentrar esforços nas funcionalidades básicas do sistema.
A configuração de um ambiente adequado para esse tipo de teste demandaria recursos e tempo adicionais não compatíveis com o cronograma atual. 
Em futuras versões, quando houver maior volume de dados e tráfego de usuários, serão aplicadas medições específicas de latência, throughput e tempo de resposta para assegurar a escalabilidade do sistema.
\subsubsection{Testes de carga}
Os testes de carga não serão realizados na fase atual, pois o sistema ainda se encontra em estágio inicial e não há volume de usuários suficiente para justificar esse tipo de avaliação. 
Esses testes requerem simulações realistas de uso simultâneo e infraestrutura específica para coleta de métricas de desempenho, o que elevaria significativamente o custo de desenvolvimento. 
Sua execução será considerada em fases posteriores, após a consolidação da versão estável do sistema.
\subsubsection{Testes de configuração}
Os testes de configuração não serão desenvolvidos devido à maturidade ainda em evolução dos ambientes de implantação e integração contínua. 
Como o projeto encontra-se em estágio inicial, com constantes ajustes nas variáveis de ambiente e parâmetros de execução, 
optou-se por validar essas configurações manualmente durante o processo de integração e entrega contínua (\textit{CI/CD}). 
Assim que os ambientes estiverem estabilizados, pretende-se automatizar parte dessas verificações para garantir maior consistência entre os diferentes estágios de desenvolvimento.

\subsection{Testes automatizados}
Os testes automatizados agrupam os dois tipos de testes que são usados no desenvolvimento do projeto: unitários e integrados, que são executados de forma automática por ferramentas de validação contínua.

A execução dos testes automatizados está ligada ao fluxo de integração contínua por meio da ferramenta \textbf{\textit{GitHub Actions}}~\cite{githubactions-2025}. Essa integração visa garantir que o código entregue atenda a padrões mínimos de qualidade e estabilidade em todas as etapas do desenvolvimento.  A cada \textit{push} ou \textit{pull request}, fluxos automatizados são executados para validar o código por meio de testes automatizados, análise estática, e verificação de cobertura. Esse processo assegura que apenas alterações estáveis e em conformidade com os padrões de qualidade sejam incorporadas à base de código principal, promovendo entregas seguras e contínuas ao longo do ciclo de desenvolvimento.

\subsection{\textit{Logs}}
O sistema adota uma estratégia de registro de eventos por meio de mecanismos de \textit{logging} que visam fornecer visibilidade sobre o comportamento da aplicação durante sua execução. Esses registros são essenciais para atividades de depuração, monitoramento e manutenção.

No \textit{back-end} e \textit{front-end}, os logs são implementados utilizando funcionalidades nativas do Node.js, como o uso de \texttt{console.log}, \texttt{console.error}, \texttt{console.warn} e \texttt{console.info}.

Os principais pontos de geração de logs incluem:
\begin{itemize}
	\item A entrada e saída de requisições HTTP (rotas, métodos, status).
	\item Erros em operações internas, como falhas de banco de dados ou validações.
\end{itemize}

Além disso, é tido um cuidado com as informações exibidas nos \textit{logs}. Em suma, é priorizado que as mensagens exibidas sejam curtas e objetivas com a cautela de não fornecer dados sensíveis.

\subsection{\textit{Code Convention}}
Para garantir a legibilidade e padronização do código, são adotadas convenções definidas com base em boas práticas da comunidade \textit{JavaScript}/\textit{TypeScript}.
Essas diretrizes ajudam a manter o código uniforme entre os diferentes desenvolvedores do time, reduzindo ambiguidades e facilitando o entendimento do sistema como um todo.

As principais práticas adotadas incluem:

\begin{itemize}
  \item \textbf{Ferramentas de \textit{Linting} e Formatação:}
  \begin{itemize}
    \item Utilização do \textbf{\textit{ESLint}}~\cite{eslint-2025} para garantir padrões de estilo e detectar possíveis erros ou práticas inadequadas de codificação.
    \item Uso do \textbf{\textit{Prettier}}~\cite{prettier-2025} para formatação automática do código, assegurando que todos os arquivos mantenham a mesma estrutura visual (espaçamento, quebras de linha, indentação, etc).
  \end{itemize}

  \item \textbf{Padrões de Nomenclatura:}
  \begin{itemize}
    \item Uso de \textit{\texttt{camelCase}} para variáveis e funções.
    \item Uso de \textit{\texttt{PascalCase}} para componentes e classes.
    \item Uso de \textit{\texttt{UPPER\_SNAKE\_CASE}} para constantes globais.
  \end{itemize}

  \item \textbf{Organização do Código:}
  \begin{itemize}
    \item Estrutura modular com separação clara entre camadas (\textit{controllers}, \textit{services}, \textit{repositories}).
    \item Agrupamento de arquivos por domínio funcional.
  \end{itemize}

  \item \textbf{Boas Práticas:}
  \begin{itemize}
    \item Escrita de código limpo e legível, evitando duplicações.
    \item Utilização de comentários apenas quando necessário, priorizando nomes autoexplicativos.
  \end{itemize}

  \item \textbf{Revisões e Padronização em Equipe:}
  \begin{itemize}
    \item Adoção de \textit{pull requests} com as devidas descrições das funcionalidades desenvolvidas.
    \item Documentação e comunicação clara de decisões técnicas relevantes.
  \end{itemize}
\end{itemize}

\subsection{Cobertura de testes}

A cobertura de testes indica a porcentagem do código-fonte exercitada pela suíte durante a execução. A cobertura é usada como \textbf{indicador de lacunas}, não como garantia de qualidade absoluta.

\subsubsection{Meta institucional (back-end).}
A equipe deverá atingir \textbf{70\% de cobertura geral no back-end} considerando somente \textbf{testes unitários}. Essa meta é agregada no nível do serviço (projeto inteiro), e \textbf{não} é exigida por módulo.

\subsubsection{Como a cobertura será medida.}
\begin{itemize}
	\item \textbf{Métrica oficial:} cobertura de \textbf{instruções/linhas} (statement/line).
	\item \textbf{Acompanhamento complementar:} cobertura de \textbf{ramos} (branch) será reportada, sem meta numérica. Serve para revelar decisões não exercitadas.
	\item \textbf{Escopo do cálculo:} entram apenas arquivos de código de produção do back-end.
	\item \textbf{Exclusões:} arquivos gerados, migrações, \textit{scripts} utilitários e código de teste não entram no denominador.
\end{itemize}

\subsubsection{Unitários x integrados.}
\begin{itemize}
	\item \textbf{Unitários:} base da meta de 70\%. Devem cobrir regras de negócio, validações e tratamentos de erro.
	\item \textbf{Integrados:} não possuem percentual mínimo. Serão priorizados \textbf{cenários críticos} do domínio.
\end{itemize}

\paragraph{Boas práticas adotadas.}
\begin{itemize}
	\item Escrever testes que verifiquem \textbf{resultados e efeitos colaterais} (asserts úteis), não apenas “executar linhas”.
	\item Dar atenção a \textbf{ramos e exceções}: caminhos de erro, validações e \textit{edge cases}.
	\item Para integrações (BD/filas), usar \textbf{\textit{test doubles}} quando possível e poucos testes integrados focados em cenários críticos.
\end{itemize}
