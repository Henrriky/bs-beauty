\section{Tecnologias e Ferramentas}

Nesta seção, são apresentadas as principais tecnologias e ferramentas utilizadas no desenvolvimento do sistema. Elas foram escolhidas com base em critérios de desempenho, escalabilidade, facilidade de manutenção e ampla adoção pela comunidade de desenvolvedores. As ferramentas foram organizadas em quatro categorias: \emph{Front-End}, \emph{Back-End}, \emph{Infraestrutura} e \emph{Qualidade de Software e Testes}.

\subsection{\emph{Front-End}}

O \emph{Front-End} é responsável pela interface do usuário e pela interação com o sistema. As tecnologias selecionadas visam oferecer uma experiência fluida e responsiva, com código limpo e de fácil manutenção.

\subsubsection{\emph{React}}
O \emph{React} \cite{react} é uma biblioteca \emph{JavaScript} para construção de interfaces de usuário. Atualmente, se coloca como uma das ferramentas mais populares nesse aspecto \cite{state-of-front-end}. Sua utilização se foca na criação de componentes encapsulados, reutilizáveis e que gerenciam seus próprios estados.

\subsubsection{\emph{TailwindCSS}}
\emph{Framework} de \gls{css} utilitário que permite a criação rápida de \emph{layouts} e estilizações diretamente nas classes \gls{html}. Parte da premissa do desenvolvedor não deixar o arquivo \gls{html} para estilizar a página com \gls{css}, embutindo as duas tecnologias em um único arquivo e também removendo código \gls{css} inútil, diminuindo o tamanho do arquivo final enviado ao usuário final \cite{tailwind}.

\subsubsection{\emph{Redux} e \emph{RTK Query}}
Biblioteca que facilita o gerenciamento de estados no \emph{React} e outras bibliotecas de \emph{interface }\cite{redux}. Já o \emph{RTK Query} oferece funcionalidades de cache e requisições assíncronas de forma otimizada \cite{rtk-query}. Ele facilita a criação de código para requisição de dados, evitando sua escrita manual, que torna-se muito repetitiva no desenvolvimento de aplicações.

\subsection{\emph{Back-End}}

O \emph{Back-End} compreende os componentes responsáveis pela lógica de negócio, persistência de dados e comunicação com o \emph{Front-End}. Para isso, foram utilizadas ferramentas que oferecem alta performance e simplicidade no desenvolvimento.

\subsubsection{\emph{NodeJS}}
Ambiente de execução \emph{JavaScript} no lado do servidor, baseado no motor V8 do navegador \emph{Chrome}. É ideal para a construção de aplicações escaláveis e que funcionem em tempo real \cite{node}. O fato de funcionar independentemente do navegador torna-o performático e atraente para os desenvolvedores.

\subsubsection{\emph{Express}}
\emph{Framework} de roteamento minimalista para Node.js que torna a criação de \gls{api}s e rotas \gls{http} simples e eficiente. Ele possui uma enxuta gama de ferramentas e recursos, que providenciam uma forma facilitada para criação de aplicações sem comprometer a já aclamada performance do Node \cite{express}, no entanto, suas funcionalidades ainda podem ser ampliadas pelos módulos de \emph{middleware}.

\subsection{Infraestrutura}

A infraestrutura do sistema foi projetada para garantir portabilidade, escalabilidade e facilidade de implantação. Sendo assim, foram empregadas tecnologias modernas de contêinerização, hospedagem em nuvem e gerenciamento de bancos de dados.

\subsubsection{\emph{Docker}}
Plataforma aberta de contêineres que permite empacotar aplicações e suas dependências de forma isolada e reprodutível. Essa divisão entre infraestrutura e aplicação culmina na entrega mais veloz de \emph{software}, e o encapsulamento de aplicações elimina os problemas que surgem por diferenças em relação a \emph{hardware} ou sistema operacional. Um contêiner \emph{Docker} funciona para qualquer pessoa da mesma forma \cite{docker}.

\subsubsection{\emph{Amazon Web Services} (\gls{aws})}
É uma plataforma que provê computação em nuvem. Esses serviços são utilizados para hospedagem, armazenamento, orquestração de infraestrutura e muito mais \cite{aws}.

\subsubsection{Banco de Dados MariaDB}
Notável e popular Sistema de gerenciamento de banco de dados relacional \emph{Open Source} usado para armazenamento persistente e estruturado de dados. Trata-se de um \emph{fork} do \emph{MySQL}, feito pelos seus desenvolvedores originais após a aquisição deste último pela \emph{Oracle} \cite{mariadb}.

\subsection{Qualidade de software e testes}

Para garantir a qualidade do código e a confiabilidade das funcionalidades desenvolvidas, foram aplicadas ferramentas de análise estática e testes automatizados.

\subsubsection{\emph{SonarQube}}
Ferramenta para análise contínua da qualidade do código, identificando problemas como \emph{bugs}, vulnerabilidades e \emph{code smells} \cite{sonarqube}.

\subsubsection{\emph{Vitest}}
\emph{Framework} moderno de testes para aplicações \emph{JavaScript} e \emph{TypeScript}, com integração nativa ao ecossistema do Vite \cite{vitest-2025}.
