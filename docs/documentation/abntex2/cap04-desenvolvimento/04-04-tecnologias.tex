\section{Tecnologias e Ferramentas} 

\subsection{\emph{Front-End}} 

\subsubsection{\emph{React}} 
O \emph{React} \cite{react} é uma biblioteca \emph{JavaScript} para construção de interfaces de usuário. Atualmente, se coloca como uma das ferramentas mais populares nesse aspecto \cite{state-of-front-end}. Sua utilização se foca na criação de componentes encapsulados, reutilizáveis e que gerenciam seus próprios estados.

\begin{figure}[htb]
	\centering
	\includegraphics[width=5cm]{cap04-desenvolvimento/images/4-4-1-1-react}
	\caption{Logo do \emph{React}}
	\label{fig:logo_react}
	\fonte{\cite{react-logo}}
\end{figure}
\FloatBarrier

\subsubsection{\emph{TailwindCSS}} 
\emph{Framework} de \gls{css} utilitário que permite a criação rápida de \emph{layouts} e estilizações diretamente nas classes \gls{html}. Parte da premissa do desenvolvedor não deixar o arquivo \gls{html} para estilizar a página com \gls{css}, embutindo as duas tecnologias em um único arquivo e também removendo código \gls{css} inútil, diminuindo o tamanho do arquivo final enviado ao usuário final \cite{tailwind}.

\begin{figure}[htb]
	\centering
	\includegraphics[width=5cm]{cap04-desenvolvimento/images/4-4-1-2-tailwindcss-logo.png}
	\caption{Logo do \emph{Tailwind}}
	\label{fig:logo_tailwind}
	\fonte{\cite{tailwind-logo}}
\end{figure}
\FloatBarrier

\subsubsection{\emph{Redux} e \emph{RTK Query}} 
Biblioteca que facilita o gerenciamento de estados no \emph{React} e outras bibliotecas de \emph{interface }\cite{redux}. Já o \emph{RTK Query} oferece funcionalidades de cache e requisições assíncronas de forma otimizada \cite{rtk-query}. Ele facilita a criação de código para requisição de dados, evitando sua escrita manual, que torna-se muito repetitiva no desenvolvimento de aplicações. 

\begin{figure}[htb]
	\centering
	\includegraphics[width=5cm]{cap04-desenvolvimento/images/4-4-1-3-redux-logo.png}
	\caption{Logo do \emph{Redux}}
	\label{fig:redux-logo}
	\fonte{\cite{redux-logo}}
\end{figure}
\FloatBarrier

\subsection{\emph{Back-End}} 

\subsubsection{\emph{NodeJS}} 
Ambiente de execução \emph{JavaScript} no lado do servidor, baseado no motor V8 do navegador \emph{Chrome}. É ideal para a construção de aplicações escaláveis e que funcionem em tempo real \cite{node}. O fato de funcionar independentemente do navegador torna-o performático e atraente para os desenvolvedores.

\begin{figure}[htb]
	\centering
	\includegraphics[width=5cm]{cap04-desenvolvimento/images/4-4-2-1-node-logo.png}
	\caption{Logo do \emph{Node}}
	\label{fig:node-logo}
	\fonte{\cite{node-logo}}
\end{figure}
\FloatBarrier

\subsubsection{\emph{Express}} 
\emph{Framework} de roteamento minimalista para Node.js que torna a criação de \gls{api}s e rotas \gls{http} simples e eficiente. Ele possui uma enxuta gama de ferramentas e recursos, que providenciam uma forma facilitada para criação de aplicações sem comprometer a já aclamada performance do Node \cite{express}, no entanto, suas funcionalidades ainda podem ser ampliadas pelos módulos de \emph{middleware}.

\begin{figure}[htb]
	\centering
	\includegraphics[width=5cm]{cap04-desenvolvimento/images/4-4-2-2-express-logo.png}
	\caption{Logo do \emph{Express}}
	\label{fig:express-logo}
	\fonte{\cite{express-logo}}
\end{figure}
\FloatBarrier

\subsection{Infraestrutura} 

\subsubsection{\emph{Docker}}
Plataforma aberta de contêineres que permite empacotar aplicações e suas dependências de forma isolada e reprodutível. Essa divisão entre infraestrutura e aplicação culmina na entrega mais veloz de \emph{software}, e o encapsulamento de aplicações elimina os problemas que surgem por diferenças em relação a \emph{hardware} ou sistema operacional. Um contêiner \emph{Docker} funciona para qualquer pessoa da mesma forma \cite{docker}.

\begin{figure}[htb]
	\centering
	\includegraphics[width=5cm]{cap04-desenvolvimento/images/4-4-3-1-docker-logo.png}
	\caption{Logo do \emph{Docker}}
	\label{fig:docker-logo}
	\fonte{\cite{docker-logo}}
\end{figure}
\FloatBarrier

\subsubsection{\emph{Amazon Web Services} (\gls{aws})} 
É uma plataforma que provê computação em nuvem. Esses serviços são utilizados para hospedagem, armazenamento, orquestração de infraestrutura e muito mais \cite{aws}.

\begin{figure}[htb]
	\centering
	\includegraphics[width=5cm]{cap04-desenvolvimento/images/4-4-3-2-aws-logo.png}
	\caption{Logo do \emph{AWS}}
	\label{fig:aws-logo}
	\fonte{\cite{aws-logo}}
\end{figure}
\FloatBarrier

\subsubsection{Banco de Dados MariaDB}
Notável e popular Sistema de gerenciamento de banco de dados relacional \emph{Open Source} usado para armazenamento persistente e estruturado de dados. Trata-se de um \emph{fork} do \emph{MySQL}, feito pelos seus desenvolvedores originais após a aquisição deste último pela \emph{Oracle} \cite{mariadb}.

\begin{figure}[htb]
	\centering
	\includegraphics[width=5cm]{cap04-desenvolvimento/images/4-4-3-3-mariadb-logo.png}
	\caption{Logo do \emph{MariaDB}}
	\label{fig:mariadb-logo}
	\fonte{\cite{mariadb-logo}}
\end{figure}
\FloatBarrier

\subsection{Qualidade de Software e Testes} 

\subsubsection{\emph{SonarQube}} 
Ferramenta para análise contínua da qualidade do código, identificando problemas como \emph{bugs}, vulnerabilidades e \emph{code smells} \cite{sonarqube}.

\begin{figure}[htb]
	\centering
	\includegraphics[width=5cm]{cap04-desenvolvimento/images/4-4-4-1-sonarqube-logo.png}
	\caption{Logo do \emph{SonarQube}}
	\label{fig:sonarqube-logo}
	\fonte{\cite{sonarqube-logo}}
\end{figure}
\FloatBarrier

\subsubsection{\emph{Vitest}}
\emph{Framework} moderno de testes para aplicações \emph{JavaScript} e \emph{TypeScript}, com integração nativa ao ecossistema do Vite \cite{vitest-2025}.

\begin{figure}[htb]
	\centering
	\includegraphics[width=5cm]{cap04-desenvolvimento/images/4-4-4-2-vitest-logo.png}
	\caption{Logo do \emph{Vitest}}
	\label{fig:vitest-logo}
	\fonte{\cite{vitest-logo}}
\end{figure}
\FloatBarrier
