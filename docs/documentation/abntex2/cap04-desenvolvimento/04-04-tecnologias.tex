\section{Tecnologias e Ferramentas} 

\subsection{\emph{Front-End}} 

\subsubsection{\emph{React}} 
Biblioteca \emph{JavaScript} para construção de interfaces de usuário com foco em componentes reutilizáveis e reatividade.

\begin{figure}[htb]
	\centering
	\includegraphics[width=6cm]{cap04-desenvolvimento/images/4-4-1-1-react}
	\caption{Logo do \emph{React}}
	\label{fig:logo_react}
	\fonte{\cite{react-logo}}
\end{figure}


\subsubsection{\emph{TailwindCSS}} 
\emph{Framework} de \gls{css} utilitário que permite a criação rápida de \emph{layouts} e estilizações diretamente nas classes \gls{html}.

\subsubsection{\emph{Redux} e \emph{RTK Query}} 
Biblioteca para gerenciamento de estado global no \emph{React}, com \emph{RTK Query} oferecendo funcionalidades de cache e requisições assíncronas de forma otimizada.

\subsection{\emph{Back-End}} 

\subsubsection{\emph{NodeJS}} 
Ambiente de execução \emph{JavaScript} no lado do servidor, baseado no motor V8 do \emph{Chrome}, ideal para aplicações escaláveis e em tempo real.

\subsubsection{\emph{Express}} 
\emph{Framework} minimalista para Node.js que facilita a criação de \gls{api}s e rotas \gls{http} de maneira simples e eficiente.

\subsection{Infraestrutura} 

\subsubsection{\emph{Docker}}
Plataforma de contêineres que permite empacotar aplicações e suas dependências de forma isolada e reprodutível.

\subsubsection{\emph{Amazon Web Services} (\gls{aws})} 
Conjunto de serviços de computação em nuvem utilizados para hospedagem, armazenamento, e orquestração da infraestrutura.

\subsubsection{Banco de Dados MariaDB}
Sistema de gerenciamento de banco de dados relacional, compatível com \emph{MySQL}, usado para armazenamento persistente e estruturado de dados.

\subsection{Qualidade de Software e Testes} 

\subsubsection{\emph{SonarQube}} 
Ferramenta para análise contínua da qualidade do código, identificando problemas como \emph{bugs}, vulnerabilidades e \emph{code smells}.

\subsubsection{\emph{Vitest}}
\emph{Framework} moderno de testes para aplicações \emph{JavaScript} e \emph{TypeScript}, com integração nativa ao ecossistema do Vite.
