%este arquivo contém todas as seções do capítulo de introdução

\section{Objetivos}
% ----------------------------------------------------------
\section{Problema e Solução Proposta}
% ----------------------------------------------------------
\section{Justificativa}
graficos com numeros, expor a relevância da solução - extensao e importancia
% ----------------------------------------------------------
\section{Análise da Concorrência}
% Aqui você pode começar a parte do texto da análise da concorrência

\subsection{Concorrente 1}
 exemplos de concorrentes:\\
 
 Trinks
 • Sede: Rio de Janeiro – RJ
 • Visão geral: Maior player SaaS para beleza no Brasil (aproximadamente 25 mil salões).
 • Funcionalidades-chave:
 – Agenda multi-profissional, confirmação por WhatsApp/SMS
 – PDV completo (NF-e, SAT, TEF, PIX) e split de comissão
 – Relatórios de caixa, estoque e folha de pagamento
 – Marketplace B2C (app Trinks) que envia clientes para o salão
 • Modelos de preço: plano grátis (até 3 colaboradores) e assinaturas de RS 59 a RS 299/mês.
 • Pontos fortes: conformidade fiscal, volume de usuários, integração nativa de pagamentos.
 • Possíveis brechas: interface carregada, pouco foco em “coworking” (locação de estações).
 
 Wedy (ex-BeautyDate)
 • Sede: Curitiba – PR
 • Visão geral: Sistema de gestão + marketplace, comprado pelo Grupo Stone em 2022.
 • Funcionalidades-chave:
 – Calendário on-line multi-colaborador
 – Repasse de comissão automático nos pagamentos via WedyPay
 – Painel financeiro com DRE simplificado, metas de faturamento
 – Campanhas de marketing (push, e-mail, SMS)
 • Preço: planos a partir de RS 79/mês ou grátis (apenas marketplace).
 • Pontos fortes: gateway de pagamento próprio, UX moderna.
 • Brechas: relatórios mais básicos que Trinks; não possui gerenciamento de aluguel de cadeiras.
 
 Belasis
 • Sede: Recife – PE
 • Visão geral: ERP cloud voltado a salões, clínicas de estética e spas.
 • Funcionalidades-chave:
 – Agenda, pagamento recorrente, clube de assinatura
 – Controle de estoque com curva ABC, inventário em lote
 – Gestão de comissões flexível (fixo, % variável, prêmio)
 – Dashboard para franqueadoras (múltiplas unidades)
 • Preço: de RS 69 a RS 229/mês (por número de usuários).
 • Pontos fortes: módulo financeiro robusto, multi-unidade.
 • Brechas: UI menos “mobile-first”; sem marketplace externo.
 
 SalãoVIP (Vip Infor)
 • Sede: Blumenau – SC
 • Visão geral: software tradicional (desktop + nuvem), muito usado por redes de médio porte.
 • Funcionalidades-chave:
 – Agenda com cores por colaborador e integração com totens de senha
 – PDV completo com cupom fiscal (ECF/SAT/ NFC-e)
 – Relatórios de conta corrente de profissional (comissão vs. aluguel)
 – Módulo de fidelidade e cartão-presente
 • Preço: licença mensal ~ RS 120 + módulos extras.
 • Pontos fortes: recursos fiscais avançados e personalização.
 • Brechas: experiência web/mobile inferior; implantação mais demorada.
 
 Vaniday Business
 • Sede: SP (filial de grupo germano-singapurense, mas operação 100 % BR)
 • Visão geral: começou como marketplace, evoluiu para sistema completo.
 • Funcionalidades-chave:
 – Agenda on-line, confirmação automática, overbooking
 – App “Vaniday Manager” para colaboradores verem agenda própria
 – Split de pagamentos com taxa fixa; adiantamento de recebíveis
 – Anúncio no marketplace Vaniday (captação de novos clientes)
 • Preço: 8 % por reserva via marketplace + assinatura para usar apenas o backoffice.
 • Pontos fortes: geração de demanda, app de gestão enxuto.
 • Brechas: dependência de taxas de marketplace, menos relatórios financeiros detalhados.
 
 AgendaPro (forte no Brasil desde 2020)
 • Origem: Chile, mas equipe comercial, suporte e fiscal 100 % brasileiros.
 • Funcionalidades-chave:
 – Agenda web + app, confirmação automática pelo WhatsApp oficial
 – Caixa, fluxo de caixa e contas a pagar/receber
 – Módulo de aulas/serviços recorrentes (útil p/ barbearia + cursos)
 – Ferramentas de marketing por funil (come-back de clientes)
 • Preço: RS 69 a RS 199/mês.
 • Pontos fortes: usabilidade, relatórios gráficos.
 • Brechas: documentos fiscais brasileiros mais limitados que Trinks/SalãoVIP.

\subsection{Concorrente 2}


\subsection{Concorrente 3}
% Conteúdo do subtópico 1.4.3

\subsection{Quadro comparativo}
% Conteúdo do subtópico 1.4.4
exemplo de quadro:
\begin{quadro}[htb]
	\caption{\label{quadro_exemplo}Exemplo de quadro}
	\begin{tabular}{|c|c|c|c|}
		\hline
		\textbf{Pessoa} & \textbf{Idade} & \textbf{Peso} & \textbf{Altura} \\ \hline
		Marcos & 26    & 68   & 178    \\ \hline
		Ivone  & 22    & 57   & 162    \\ \hline
		...    & ...   & ...  & ...    \\ \hline
		Sueli  & 40    & 65   & 153    \\ \hline
	\end{tabular}
	\fonte{Autor.}
\end{quadro}