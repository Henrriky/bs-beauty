\begin{table}[htb]  
	% Inicia o ambiente flutuante para tabelas
	% Parâmetros indicam onde a tabela pode ser posicionada:
	% h = aqui, t = topo da página, b = base da página
	
	\centering  
	% Centraliza horizontalmente o conteúdo dentro do ambiente "table"
	
	\caption{Custos estimados do projeto}  
	% Insere a legenda numerada da tabela, que pode aparecer acima ou abaixo dependendo da classe do documento
	
	\label{tab:custos}  
	% Cria um rótulo para referência cruzada dentro do documento (com \ref{tab:custos})
	
	\begin{tabular}{|l|r|r|}  
		% Começa o ambiente que cria a estrutura da tabela propriamente dita
		% Define três colunas:  
		% 'l' = alinha texto à esquerda,  
		% 'r' = alinha números/dados à direita,  
		% '|' = desenha linhas verticais nas fronteiras e entre as colunas
		
		\hline  
		% Desenha uma linha horizontal no topo da tabela (e serve para outras linhas)
		
		Descrição & Quantidade & Preço (R\$) \\  
		% Primeira linha da tabela (cabeçalho), separa colunas com '&' e termina linha com '\\'
		% '\$' é para mostrar o símbolo de cifrão literal
		
		\hline  
		% Linha horizontal separando cabeçalho do corpo da tabela
		
		Material A & 100 & 1500 \\  
		\hline 
		Material B & 50 & 800 \\  
		\hline 
		Mão de obra & - & 2000 \\  
		% Linhas com os dados da tabela. O símbolo '-' indica campo vazio ou não aplicável
		
		\hline  
		% Linha horizontal no final da tabela, fechando a grade
		
	\end{tabular}  
	% Fim do ambiente tabular (estrutura da tabela)
	
	\vspace{0.5em}  % Espaço vertical extra entre a tabela e o texto da fonte
	{\footnotesize Fonte: Dados da empresa XYZ.\par}  
	% Texto logo abaixo da tabela, em fonte menor (footnotesize), para indicar a fonte dos dados
	% '\par' finaliza o parágrafo, garantindo espaçamento correto
	
\end{table}  
% Fecha o ambiente flutuante table
