	\begin{resumo}
	Este projeto integrado de extensão apresenta o desenvolvimento da aplicação \emph{web} BS \emph{Beauty}, com o objetivo de otimizar a gestão e o agendamento de serviços de beleza no ambiente \emph{coworking}, sob responsabilidade da gestora, nossa parceira de extensão.
	
	Como parte de uma iniciativa competitiva, a solução digital desenvolvida tem como propósito melhorar o atendimento e fidelizar clientes, reduzindo o tempo gasto em tarefas administrativas e garantindo uma interface de usuário intuitiva. Para isso, o sistema centraliza agendas, previne conflitos de horário e fornece notificações automáticas, relatórios financeiros e \emph{dashboards} de desempenho.
	
	A fim de atender as demandas da nossa parceira, a comunicação constante foi essencial para o levantamento de requisitos, análise de concorrentes e definição de regras de negócio. Para a gestão das etapas do projeto, foi adotado o framework ágil \emph{Scrum}, formalizando o planejamento e controle de tarefas no \emph{ProjectLibre}, um software de gestão de projetos que permite gerenciar cronogramas e alocar recursos para as tarefas definidas.
	
	A arquitetura da aplicação foi idealizada em camadas, sendo detalhada em diagramas de componentes e de implantação. Paralelamente, elaborou-se o plano de testes, padronizou-se a documentação e avaliou-se a viabilidade financeira em cenários realistas, otimistas e pessimistas.
	
	Como resultado, a aplicação desenvolvida fortalece conhecimentos teóricos do curso de graduação, aproxima-os das demandas de mercado e promove inovação tecnológica no setor de beleza apoiando o modelo de \emph{coworking}.

\textbf{Palavras-chave:} aplicação \emph{web}. agendamento online. \emph{coworking} de beleza. gestão de serviços. \emph{Scrum}.

	\end{resumo}