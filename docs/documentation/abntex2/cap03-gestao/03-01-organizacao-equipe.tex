\section{Organização da Equipe}

A equipe do presente projeto é composta por seis docentes do curso de graduação Superior de Tecnologia em Análise e Desenvolvimento de Sistemas do \gls{ifsp}, a saber:

\begin{itemize}
	\item Alyson César Fumagalli dos Santos Júnior
	\item Bruno de Almeida Fischer
	\item Eliel da Silva
	\item Giovanna Camille Silva Carvalho
	\item Henrique Santiago Pires
	\item Henrriky Jhonny de Oliveira Bastos
\end{itemize}

O grupo de trabalho foi formado logo no início da primeira disciplina de \gls{pie} por estudantes que já haviam realizado diversos outros trabalhos em conjunto e, portanto, estavam acostumados a trabalhar em equipe.

Visando uma transmissão de informações clara e centralizada, utilizou-se o \emph{Discord} \cite{discord-2025} como ferramenta de comunicação para realizar reuniões assíncronas; e o \emph{Taiga} \cite{taiga-2025} para organizar documentos, atribuir tarefas e monitorar o andamento do projeto (juntamente com o \emph{ProjectLibre} \cite{projectlibre-2025}).

\subsection{Responsabilidades / Papéis / Atividades}
\label{subsec:papeis-equipe}

Para cada integrante da equipe foi definido um papel contendo responsabilidades e atividades definidas com base em suas respectivas proficiências em diferentes áreas, a fim de distribuir as tarefas do projeto de maneira eficiente.

O Quadro \ref{frame:distribuicao-papeis} descreve a distribuição dos membros do grupo com base em seus respectivos papéis de uma forma mais geral.

\begin{quadro}[ht]
	\setlength{\tabcolsep}{3pt}
	\begin{center}
		\caption{\label{frame:distribuicao-papeis}Membros e seus respectivos papéis}
		\begin{tabular}{|l|c|c|m{3cm}|c|}
			\hline
			\textbf{Membro} & \textbf{Gestor} & \textbf{\textit{Tech Lead}} & \centering\textbf{Desenvolvedor \textit{Full Stack}} & \textbf{Analista de Documentação} \\
			\hline
			Alyson & \checkmark &  & \centering\checkmark &  \\
			\hline
			Bruno &  &  & \centering\checkmark &  \\
			\hline
			Eliel &  &  & \centering\checkmark &  \\
			\hline
			Giovanna &  &  &  & \checkmark \\
			\hline
			Henrique &  &  & \centering\checkmark &  \\
			\hline
			Henrriky &  & \checkmark & \centering\checkmark &  \\
			\hline
		\end{tabular}
		\fonte{Produzido pelos autores}
	\end{center}
\end{quadro}

\indent Assim, constata-se que a equipe conta com 1 (um) gestor, responsável pelo gerenciamento de todo o projeto; 1 (um) \textit{tech lead}, encarregado de guiar a equipe de desenvolvimento; 5 (cinco) desenvolvedores \textit{full stack} (com 2 deles desempenhando outros papéis paralelos) incumbidos por desenvolver o sistema em todas as suas etapas e 1 (uma) analista de documentação para supervisionar as documentações do projeto. 

O Quadro \ref{quad:membros-atividades} apresenta as atividades desempenhadas pelos membros da equipe nas diversas áreas que contemplaram o desenvolvimento do projeto.

\begin{quadro}[h]
	\setlength{\tabcolsep}{2pt}
	\begin{center}
		\caption{\label{quad:membros-atividades}Membros e suas atividades}
		\begin{tabular}{|l|c|m{2.5cm}|m{2cm}|c|m{2cm}|c|}
			\hline
			\textbf{Membro} & \textbf{\emph{Taiga}} & \centering\textbf{\textit{Front-End Back-End}} & \centering\textbf{Banco de Dados} & \textbf{Documentação} & \centering\textbf{Diário de Bordo} & \textbf{\emph{ProjectLibre}}\\
			\hline
			Alyson & \checkmark & \centering\checkmark & \centering\checkmark & \checkmark & \centering\checkmark & \checkmark \\
			\hline
			Bruno & \checkmark & \centering\checkmark & \centering\checkmark & \checkmark & & \checkmark\\
			\hline
			Eliel & \checkmark & \centering\checkmark & \centering\checkmark & \checkmark & & \checkmark\\
			\hline
			Giovanna & \checkmark &  & \centering\checkmark & \checkmark & \centering\checkmark & \checkmark\\
			\hline
			Henrique & \checkmark & \centering\checkmark & \centering\checkmark & \checkmark & & \\
			\hline
			Henrriky & \checkmark & \centering\checkmark & \centering\checkmark & \checkmark & & \\
			\hline
		\end{tabular}
		\fonte{Produzido pelos autores}
	\end{center}
\end{quadro}