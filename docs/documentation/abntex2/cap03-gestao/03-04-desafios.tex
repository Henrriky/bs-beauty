\section{Desafios e Soluções}

Durante a execução do projeto, a equipe enfrentou como principal desafio a limitação de tempo para conciliar as atividades acadêmicas e o desenvolvimento das funcionalidades planejadas. Essa restrição impactou diretamente o ritmo de entrega das tarefas e exigiu uma reorganização das prioridades no backlog. 

Para superar esse obstáculo, foram adotadas estratégias de gestão ágil, como a redefinição de escopo em cada sprint e a divisão mais clara das responsabilidades entre os membros. Além disso, a comunicação constante por meio de reuniões breves e o uso de ferramentas colaborativas, como o Taiga e o GitHub, contribuíram para manter o alinhamento do time e garantir a entrega dos principais requisitos dentro do prazo estabelecido. Essa experiência reforçou a importância do planejamento contínuo e da adaptação frente a imprevistos no contexto de projetos reais.
