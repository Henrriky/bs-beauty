\section{Metodologias de Gestão e Desenvolvimento}

\label{section:metolodogia-gestao}

Para o projeto, foi adotada a metodologia Scrum \cite{scrum-2024} voltada tanto para a gestão quanto para o desenvolvimento do sistema.

\subsection{Scrum}

O \textit{framework} Scrum foi escolhido por ter sido bastante estudado em disciplinas anteriores e também devido à equipe já ter uma certa familiaridade em trabalhar com ele. A metodologia foi adaptada de forma a incorporar alguns elementos do \emph{Kanban} \cite{kanban-2023} a fim de ter um sistema visual para monitorar as atividades em andamento.

Tendo em mente as responsabilidades provenientes do Scrum, o Quadro \ref{frame:papeis-scrum} descreve como os membros do grupo foram distribuídos seguindo os papéis da metodologia.

\begin{quadro}[ht]
	\setlength{\tabcolsep}{3pt}
	\begin{center}
		\caption{\label{frame:papeis-scrum}Papéis dos integrantes com base no Scrum}
		\begin{tabular}{|l|c|c|c|}
			\hline
			\textbf{Membro} & \textbf{\textit{Product Owner}} & \textbf{\textit{Scrum Master}} & \textbf{Equipe de Desenvolvimento} \\
			\hline
			Alyson &  & \checkmark & \checkmark \\
			\hline
			Bruno &  &  & \checkmark \\
			\hline
			Eliel &  &  & \checkmark \\
			\hline
			Giovanna &  &  & \checkmark \\
			\hline
			Henrique &  &  & \checkmark \\
			\hline
			Henrriky & \checkmark &  & \checkmark \\
			\hline
		\end{tabular}
		\fonte{Produzido pelos autores}
	\end{center}
\end{quadro}

A plataforma \emph{Notion} \cite{notion-2025} teve um papel significativo na implementação do Scrum: por meio dela, criou-se um espaço de trabalho para registrar o \textit{product backlog}, os \textit{sprints} e suas respectivas tarefas.

Com essa ferramenta, ainda foi possível atribuir informações --- como prioridade, \textit{status}, prazos e responsáveis --- para cada atividade, além de criar visualizações para simular as colunas e cartões do \emph{Kanban}, criando uma estrutura organizada e informativa que facilitou o gerenciamento do projeto.

\subsubsection{\emph{Sprints}}

Para o desenvolvimento do sistema, foram determinados \textit{sprints} quinzenais que se iniciam na segunda-feira da primeira semana do \textit{sprint} e terminam na sexta-feira da segunda semana. Com essas duas semanas garante-se que a equipe tenha tempo suficiente para desenvolver as funcionalidades do sistema, permitindo entregas incrementais significativas.

Dessa forma, os \textit{sprints} e suas tarefas específicas foram definidos ao longo do desenvolvimento com base nos itens do \textit{product backlog}, que por sua vez foram estabelecidos a partir das histórias de usuário levantadas junto à entidade parceira do projeto.

Ademais, as reuniões características dos \textit{sprints} da metodologia Scrum foram marcadas para ocorrer presencialmente nos dias referentes à disciplina de \gls{pie}, permitindo os membros se organizarem para definir quais itens seriam trabalhados no próximo \textit{sprint}, bem como os prazos, tarefas, prioridades e responsáveis. 
