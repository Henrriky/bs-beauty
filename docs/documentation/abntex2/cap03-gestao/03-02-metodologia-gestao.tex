\section{Metodologias de Gestão e Desenvolvimento}

\label{section:metolodogia-gestao}

Para o projeto, foi adotada a metodologia Scrum \cite{scrum-2024} voltada tanto para a gestão quanto para o desenvolvimento do sistema.

\subsection{Scrum}

O \textit{framework} Scrum foi escolhido por ter sido bastante estudado em disciplinas anteriores e também devido à equipe já ter uma certa familiaridade em trabalhar com ele. Além disso, para possibilitar a aplicação da metodologia, foi utilizado o Taiga \cite{taiga-2025}, uma ferramenta \textit{open-source} voltada justamente para gerenciamento de projetos.

Com o Taiga, foi possível aplicar os elementos do Scrum, como a definição de papéis dos integrantes do grupo, a criação e refinamento do \textit{product backlog} e o estabelecimento dos \textit{sprints} a serem trabalhados.

O Taiga também conta com elementos do \emph{Kanban} \cite{kanban-2023} que, nesse projeto, foram incorporados à metodologia Scrum. Esses recursos incluem:

\begin{itemize}
	\item Criação de quadros separados para cada sprint voltados ao monitoramento das tarefas definidas;
	\item Estabelecimento de colunas dentro dos quadros para indicar os estágios das tarefas (“a fazer”, “em andamento”, “concluído” ou “em revisão”);
	\item Representação de tarefas na forma de cartões, contendo informações como o responsável pela tarefa, seu prazo de conclusão e descrição, que são transitados entre as colunas dos quadros;
	\item Aplicação de \textit{swimlanes} nos quadros para agrupar tarefas relacionadas de acordo com as \textit{user stories}.
\end{itemize}

Dessa forma, a integração entre \emph{Kanban} e \emph{Scrum} foi benéfica para a equipe, haja vista que permitiu uma melhor visualização do fluxo de trabalho e andamento do \textit{sprint} trabalhado, além de possibilitar a identificação de dependências entre as tarefas.

Tendo em mente as responsabilidades provenientes do Scrum, o Quadro \ref{frame:papeis-scrum} descreve como os membros do grupo foram distribuídos seguindo os papéis da metodologia.

\begin{quadro}[ht]
	\setlength{\tabcolsep}{3pt}
	\begin{center}
		\caption{\label{frame:papeis-scrum}Papéis dos integrantes com base no Scrum}
		\begin{tabular}{|l|c|c|c|}
			\hline
			\textbf{Membro} & \textbf{\textit{Product Owner}} & \textbf{\textit{Scrum Master}} & \textbf{Equipe de Desenvolvimento} \\
			\hline
			Alyson &  & \checkmark & \checkmark \\
			\hline
			Bruno &  &  & \checkmark \\
			\hline
			Eliel &  &  & \checkmark \\
			\hline
			Giovanna &  &  & \checkmark \\
			\hline
			Henrique &  &  & \checkmark \\
			\hline
			Henrriky & \checkmark &  & \checkmark \\
			\hline
		\end{tabular}
		\fonte{Produzido pelos autores}
	\end{center}
\end{quadro}

\subsubsection{\emph{Sprints}}

Para o desenvolvimento do projeto, foram determinados \textit{sprints} semanais que se iniciam na quarta-feira e terminam na terça-feira da outra semana. Com esse período de tempo, garante-se que a equipe realize entregas incrementais continuamente.

Dessa forma, os \textit{sprints} e suas tarefas específicas foram definidos ao longo do desenvolvimento com base nos itens do \textit{product backlog}, que por sua vez foram estabelecidos a partir das histórias de usuário levantadas junto à entidade parceira do projeto.

Ademais, as reuniões características dos \textit{sprints} da metodologia Scrum foram marcadas para ocorrer presencialmente nos dias referentes à disciplina de \gls{pie}, permitindo os membros se organizarem para definir quais itens seriam trabalhados no próximo \textit{sprint}, bem como os prazos, tarefas, prioridades e responsáveis. 
