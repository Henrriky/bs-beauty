\section{Metodologias de Gestão e Desenvolvimento}

Para o projeto, foi adotada a metodologia Scrum \cite{scrum-2024} tanto voltada para a gestão quanto para o desenvolvimento do sistema.

\subsection{Scrum}

O \textit{framework} Scrum foi escolhido por ter sido bastante estudado em disciplinas anteriores e também devido à equipe já ter uma certa familiaridade em trabalhar com ele. A metodologia foi adaptada de forma a incorporar alguns elementos do Kanban \cite{kanban-2023} a fim de ter um sistema visual para monitorar as atividades em andamento.

Tendo em mente as responsabilidades provenientes do Scrum, o Quadro \ref{frame:papeis-scrum} descreve como os membros do grupo foram distribuídos seguindo os papéis da metodologia.

\begin{quadro}[ht]
	\setlength{\tabcolsep}{3pt}
	\begin{center}
		\caption{\label{frame:papeis-scrum}Papéis dos integrantes com base no Scrum}
		\begin{tabular}{|l|c|c|c|}
			\hline
			\textbf{Membro} & \textbf{\textit{Product Owner}} & \textbf{\textit{Scrum Master}} & \textbf{Equipe de Desenvolvimento} \\
			\hline
			Alyson &  & \checkmark & \checkmark \\
			\hline
			Bruno &  &  & \checkmark \\
			\hline
			Eliel &  &  & \checkmark \\
			\hline
			Giovanna &  &  & \checkmark \\
			\hline
			Henrique &  &  & \checkmark \\
			\hline
			Henrriky & \checkmark &  & \checkmark \\
			\hline
		\end{tabular}
		\fonte{Produzido pelos autores}
	\end{center}
\end{quadro}

A plataforma Notion \cite{notion-2025} teve um papel significativo na implementação do Scrum: por meio dela, criou-se um espaço de trabalho para registrar o \textit{product backlog}, os \textit{sprints} e suas respectivas tarefas.

Com essa ferramenta, ainda foi possível atribuir informações --- como prioridade, \textit{status}, prazos e responsáveis --- para cada atividade, além de criar visualizações para simular as colunas e cartões do Kanban, criando uma estrutura organizada e informativa que facilitou o gerenciamento do projeto.

\subsubsection{Sprints}

Os \textit{sprints} do projeto possuem duração de 2 semanas para garantir que a equipe tenha tempo suficiente para implementar os requisitos definidos para um determinado incremento.

Dessa forma, estabeleceu-se os \textit{sprints} com base nos itens do \textit{product backlog}, que por sua vez foram definidos tendo em mentes as histórias de usuário levantadas junto à entidade parceira do projeto.

Ademais, as reuniões características da metodologia Scrum foram marcadas para ocorrer presencialmente nos dias referentes às disciplinas de Projeto Integrado de Extensão, nas quais os membros poderiam se juntar e organizar para definir quais itens seriam trabalhados no próximo \textit{sprint}, bem como os prazos, tarefas, prioridades e responsáveis. 
