

A aplicação web \emph{BS Beauty} representa uma contribuição relevante para o mercado de serviços de beleza, sobretudo no modelo de \emph{coworking}. Ao centralizar e digitalizar o agendamento e a gestão do fluxo de trabalho de diferentes profissionais autônomos, a aplicação reduz erros administrativos e melhora a experiência do cliente. Considerando que, segundo estudo do \cite{senac2022}, até 30\% do tempo dos pequenos empreendedores é consumido em tarefas manuais, a \emph{BS Beauty} oferece um diferencial competitivo, permitindo que gestores e profissionais dediquem mais tempo ao atendimento do que às operações administrativas.

Durante o desenvolvimento da aplicação, a escolha da metodologia Scrum foi fundamental para o planejamento e a entrega do projeto. Os ciclos de \emph{sprints} permitiram validar rapidamente cada funcionalidade junto à nossa parceira de extensão, garantindo flexibilidade na evolução de requisitos. Durante discussões sobre os requisitos, o módulo de pagamento \emph{on-line} foi estrategicamente descartado do sistema, uma vez que a gestora optou por manter os pagamentos apenas de forma presencial. Além disso, o fluxo de agendamento foi inicialmente pensado como horário$\to$profissional, porém, após sugestões do orientador, foi dividido em três caminhos distintos: agendamento apenas por horário, apenas por profissional ou de forma combinada. Essas mudanças só foram possíveis graças à flexibilidade do Scrum.

A comunicação com a gestora Bruna e a coordenação interna da equipe, apesar de bem-sucedidas, representaram desafios significativos. A necessidade de validações constantes das regras de negócio, aliada a conflitos de agenda, impossibilitou reuniões presenciais com todos os \emph{stakeholders}. Por isso, grande parte das interações foi conduzida por mensagens de texto ou ligações. Ferramentas como \emph{Discord} e \emph{Notion} foram essenciais para alinhar demandas, formalizar decisões e manter a coesão no código e na documentação.

Espera-se que, após a implantação, o sistema elimine conflitos de agenda, reduza a taxa de não-comparecimento por meio de lembretes automáticos e aumente a receita mensal em virtude da otimização da ocupação das estações, da clareza nos relatórios e da fidelização de clientes. Além disso, ao disponibilizar \emph{dashboards} de performance e indicadores de satisfação, a \emph{BS Beauty} criará uma base de dados estratégica para decisões futuras de \emph{marketing} e expansão.

Em síntese, este projeto de extensão uniu teoria e prática, contribuindo não apenas para a formação de profissionais capacitados, mas também para a criação de um produto de alto impacto para o mercado de beleza, apoiando a consolidação e o crescimento dos \emph{coworkings} de beleza por meio de uma solução digital eficiente.
