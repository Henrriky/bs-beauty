\section{Histórico de sistemas gerenciadores de agendamento}

O ato de agendar serviços é uma prática antiga, e o gerenciamento desses agendamentos sempre foi um processo trabalhoso e passível de erros. No entanto, à medida que a tecnologia avança, surgem também ferramentas que facilitam esse processo, diminuindo a chance de falhas e perda de informações.

Nos primórdios, o agendamento de qualquer serviço era feito apenas presencialmente, dada a falta de tecnologias de comunicação à distância. Consequentemente, o controle financeiro e de clientes era realizado manualmente, o que gerava a necessidade de contratar outras pessoas para auxiliar neste processo, causando mais gastos. Com o advento do telefone, o acesso dos clientes tornou-se mais fácil. Porém, o trabalho de gestão ainda permanecia manual, exceto nos casos em que se adquiriam soluções de gestão em \gls{dvd}s, que eram pouco personalizadas para o negócio específico e não integradas aos agendamentos. Além disso, mantinha-se a necessidade de alguém disponível para atender às chamadas ou para controlar a correlação das agendas com as informações do \gls{dvd}.

Posteriormente, com o surgimento da internet e das redes sociais, a maioria dos empresários prestadores de serviços aproveitou a oportunidade para concentrar seus agendamentos em mensagens de texto. Essa abordagem eliminava a necessidade de alguém estar sempre disponível para responder e confirmar, além de permitir a comunicação paralela com clientes. Ademais, a gestão já podia ser mais integrada a calendários virtuais (como o \emph{Google Calendar}), aos do próprio \emph{smartphone}, ou mesmo a planilhas digitais. Contudo, o processo de agendamento ainda dependia de uma ferramenta de comunicação que exigia intervenção humana: uma pessoa precisava estar envolvida na conversa para anotar o serviço agendado em outra ferramenta e controlar as finanças do negócio – tarefas ainda manuais ou realizadas com um sistema à parte, o que poderia levar à perda de informações. Esse excesso de ferramentas e a comunicação fragmentada consumiam tempo e podiam resultar na desistência de clientes.

Diante desse cenário, surgiu a necessidade de um sistema de agendamento automático, que dispensasse a comunicação direta e já integrasse o processo de gestão de clientes e finanças na mesma plataforma. Portanto, surgiram os sistemas de gestão de agendamento \emph{on-line}, com a promessa de reduzir gastos, minimizar processos manuais e aumentar a produtividade. Além disso, a possibilidade de fazer reservas a qualquer hora ajuda a atrair mais clientes e a mantê-los \cite{reservio}.

Por fim, os sistemas de gestão e agendamento de serviços evoluíram rapidamente, tornando-se cada vez mais personalizados para diferentes setores ou até mesmo para empresas distintas dentro do mesmo setor. Atualmente, é possível contratar facilmente uma instância de aplicação específica para salões de beleza, adaptada ao próprio negócio, dada a vasta quantidade de soluções já existentes no mercado.